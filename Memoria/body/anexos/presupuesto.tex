\chapter{Anexo II - Presupuesto}

% En este anexo se quiere hacer una aproximación del presupuesto que tendría el proyecto llevado a cabo. Para ello, debemos hacer recuentro de cuantas horas efectivas se han dedicado al proyecto, dado que, si bien es cierto que el TFM se ha extendido el doble del tiempo que se había estimado, no se ha dedicado el doble de semanas para la realización del mismo, dado que se ha ido compaginando con otros proyectos en paralelo. También hay que tener en cuenta los medios que se han utilizado para el desarrollo, clasificándolos en elementos \textit{hardware} y en elementos \textit{software}.

% \section{Duración del proyecto}

% %Con el propósito de obtener el número de horas de trabajo por semana del proyecto en \textbf{promedio}, se va a realizar un diagrama de Gantt. De este modo se podrá apreciar la distribución de tareas a lo largo del \gls{tfm} y así ser capaces de obtener un número de horas de trabajo aproximado por semana.

% Según se ha comentado anteriormente, con el propósito de obtener el número de horas \textbf{efectivas} dedicadas al proyecto, se va a realizar un diagrama de Gantt. De esta forma se pretende aclarar cuantas semanas se han dedicado a cada etapa del proyecto, y de este modo, poder llegar a estimar un número de horas efectivas de trabajo.\\
% \\
% Se quiere aclarar, que la duración del proyecto se ha extendido en el tiempo casi un año más de lo previsto, sin embargo, esta extensión temporal no ha sido completamente efectiva sobre el proyecto, dado que se ha ido compaginando con otros proyectos en paralelo. Se quiere de esta forma obtener, aproximadamente, una estimación temporal en semanas de cuanto ha llevado cada etapa del \gls{tfm}.\\
% \\
% Como se puede ver en la figura \ref{gantt}, la etapa crítica de este proyecto ha sido el aprendizaje del funcionamiento interno del \gls{bofus} y el desarrollo de las modificaciones sobre el mismo. Si bien es cierto que la curva de aprendizaje del \gls{bofus} es complicada, una de las grandes causas en la duración de dichas etapas ha sido la paralelización de este trabajo con otros proyectos, gastando bastante tiempo en los cambios de contexto entre ellos.\\

% \begin{sidewaysfigure}[ht!]
% 	\begin{center}
% 		\begin{tikzpicture}
% 			\begin{ganttchart}[
% 					hgrid,
% 					vgrid,
% 					y unit chart=.75cm,
% 					bar/.append style={fill=cyan!50},
% 					expand chart=\textwidth
% 				]{1}{30}
% 				\gantttitle{Quincenas}{30} \\
% 				\gantttitlelist{1,...,30}{1} \\
% 				\ganttbar{Documentación y estudio}{1}{6.5} \\
% 				\ganttbar{Planificación}{1}{6.5} \ganttnewline
% 				\ganttbar{Diseno del entorno}{3}{13} \ganttnewline
% 				\ganttbar{Instalación del entorno}{6.5}{16} \ganttnewline
% 				\ganttbar{Aprendizaje de las herramientas}{6.5}{18} \ganttnewline
% 				\ganttbar{Desarrollo}{15}{25} \ganttnewline
% 				\ganttbar{Despliegue del entorno}{22}{25} \ganttnewline
% 				\ganttbar{Valoración del desarrollo}{25}{26} \\
% 				\ganttbar{Informe final}{26}{30}
% 			\end{ganttchart}
% 		\end{tikzpicture}
% 	\end{center}
% 	\caption{Diagrama de Gantt del proyecto}
% 	\label{gantt}
% \end{sidewaysfigure}

% \begin{table}[ht!]
% 	\centering
% 	\begin{tabular}{|c|c|c|}
% 		\hline
% 		\rowcolor[HTML]{EFEFEF}
% 		\textbf{Número de horas totales} & \textbf{Horas efectivas} & \textbf{Horas efectivas por semana} \\ \hline
% 		2000h                            & $\approx$ 1200 - 1000h   & $\approx$ 20h                       \\ \hline
% 	\end{tabular}
% 	\caption{Promedio de horas de trabajo }
% 	\label{dig:horasTrabajadas}
% \end{table}

% \section{Costes del proyecto}

% Para realizar el cálculo de los costes del proyecto, se llevará a cabo una diferenciación previa en términos de \textit{hardware}, \textit{software} y mano de obra. Esta metodología permitirá un desglose detallado de los costes, lo que brindará una mayor claridad en cuanto a la cuantía total del proyecto.  En primer lugar, se considerarán los costes relacionados con el \textit{hardware}. Esto implica evaluar los gastos asociados a la adquisición de equipos, dispositivos y componentes físicos necesarios para el desarrollo y funcionamiento del proyecto. En segundo lugar, se analizarán los costes de \textit{software}. Esto involucra evaluar los gastos relacionados con las licencias de software, el desarrollo de aplicaciones personalizadas, la adquisición de paquetes de software especializados y los costes de mantenimiento y actualizaciones de los programas utilizados en el proyecto. Finalmente, se tendrán en cuenta los costes de mano de obra. Esto incluirá los gastos relacionados con los recursos humanos involucrados en el proyecto, como los salarios de los empleados. Es importante destacar que al desglosar los costes del proyecto de esta manera, se proporcionará una visión más completa y detallada de los recursos financieros requeridos en cada área. Esto permitirá una mejor planificación, seguimiento y control de los costes de cara a futuro en el desarrollo de un proyecto de mismas características.

% \begin{table}[ht]
% 	\centering
% 	\begin{tabular}{|c|r|}
% 		\hline
% 		\rowcolor[HTML]{EFEFEF}
% 		\textbf{Producto (IVA incluido)}           & \multicolumn{1}{c|}{\cellcolor[HTML]{EFEFEF}\textbf{Valor (€)}} \\ \hline
% 		Ordenador portátil Asus zeenbook           & 1560,00                                                         \\ \hline
% 		Servidor  A                                & 2100,00                                                         \\ \hline
% 		Servidor  B                                & 1700,00                                                         \\ \hline
% 		Pantalla Lenovo L27i                       & 130,00                                                          \\ \hline
% 		Pantalla Benq 21"                          & 80,00                                                           \\ \hline
% 		RPi 4 (2 uds)                              & 250,00                                                          \\ \hline
% 		Periféricos                                & 300,00                                                          \\ \hline
% 		Infraestructura de Red (APs y Router Asus) & 250,00                                                          \\ \hline
% 	\end{tabular}
% 	\caption{Presupuesto desglosado del Hardware para el \glsentryshort{tfm}}
% 	\label{tab:costesHardware}
% \end{table}

% Las licencias de software generalmente se venden por años, o por meses. Por ello, se ha calculado el precio equivalente asociado a la duración del \gls{tfm}.
% \begin{table}[ht]
% 	\centering
% 	\begin{tabular}{|c|r|}
% 		\hline
% 		\rowcolor[HTML]{EFEFEF}
% 		\textbf{Producto (IVA incluido)} & \multicolumn{1}{c|}{\cellcolor[HTML]{EFEFEF}\textbf{Valor (€)}} \\ \hline
% 		Microsoft Office                 & 300,00                                                          \\ \hline
% 		Adobe Suite                      & 500,00                                                          \\ \hline
% 		Overleaf                         & 40,00                                                           \\ \hline
% 		G suite                          & 120,00                                                          \\ \hline
% 	\end{tabular}
% 	\caption{Presupuesto desglosado del Software  para el \glsentryshort{tfm}}
% 	\label{tab:costesSoftware}
% \end{table}

% \vspace{0.5cm}

% Para determinar los costes de mano de obra, se han utilizado como referencia los honorarios de un ingeniero senior, los cuales ascienden a 30€ por hora. Estos honorarios se aplicarán en función de la cantidad de horas efectivas dedicadas al proyecto. En cuanto a los costes relacionados con el \textit{hardware} y el \textit{software}, se han agrupado como un único elemento dentro del presupuesto. Se ha considerado el valor total resultante del desglose de los productos indicados en el análisis. Al agregar estos costes de \textit{hardware} y \textit{software} al presupuesto, se obtiene el valor total necesario para cubrir estos componentes esenciales del proyecto. Es importante destacar que estos cálculos se basan en las referencias proporcionadas y están sujetos a ajustes según las necesidades específicas del proyecto y las tarifas y precios aplicables en cada caso particular. El desglose y la inclusión de estos costes en el presupuesto garantizan una consideración adecuada de los recursos necesarios para el éxito y la ejecución del proyecto.
% \vspace{0.5cm}

% \begin{table}[ht]
% 	\centering
% 	\begin{tabular}{|c|r|r|r|}
% 		\hline
% 		\rowcolor[HTML]{EFEFEF}
% 		\textbf{Descripción (IVA incluido)} & \multicolumn{1}{c|}{\cellcolor[HTML]{EFEFEF}\textbf{Unidades}}    & \multicolumn{1}{c|}{\cellcolor[HTML]{EFEFEF}\textbf{Coste unitario (€)}} & \multicolumn{1}{c|}{\cellcolor[HTML]{EFEFEF}\textbf{Coste total (€)}} \\ \hline
% 		Material Hardware                   & 1                                                                 & 6370,00                                                                  & 6370,00                                                               \\ \hline
% 		Material Software                   & 1                                                                 & 960,00                                                                   & 960,00                                                                \\ \hline
% 		Mano de obra                        & 1000                                                              & 30,00                                                                    & 30000,00                                                              \\ \hline
% 		Costes fijos (Luz, Internet)        & 1                                                                 & 650,00                                                                   & 650,00                                                                \\ \hline
% 		\rowcolor[HTML]{FFFFC7}
% 		\textbf{TOTAL}                      & \multicolumn{3}{r|}{\cellcolor[HTML]{FFFFC7}\textbf{37.980,00 €}}                                                                                                                                                    \\ \hline
% 	\end{tabular}
% 	\caption{Presupuesto total con IVA}
% 	\label{tab:budget}
% \end{table}