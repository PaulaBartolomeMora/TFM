%%%%%%%%%%%%%%%%%%%%%%%%%%%%%%%%%%%%%%%%%%%%%%%%%%%%%%%%%%%%%%%%%%
\section{Machine Learning/IA}
\label{sec:ml}


%ver Smart Grid Stability Prediction Model Using Neural Networks to Handle Missing Inputs
%figura interesante



%hablar sobre ml
%relacionar ml con sg
\cite{impact}


%ver figuras 4y 5 de deep learning in smart grids


%aqui hablar de ml para eficiencia de consumo


%ver figura pag6 cite stab -> esquema acciones desarrollo






%cite stab (para que se usan los algoritmos)

% Los algoritmos de predicción se utilizan para estimar la oferta y la demanda.
% de la red (Wang et al., 2018). En Singh y Yassine (2018), el
% Se abordó el análisis de datos de medidores inteligentes y se propusieron tres principales
% Aplicaciones: análisis de carga, previsión de carga y gestión de carga. El
% Las técnicas clave para estas aplicaciones incluyen series temporales, agrupación de datos, reducción de dimensionalidad, clasificación de datos, detección de valores atípicos,
% matriz de bajo rango y aprendizaje en línea. En Kezunovic et al. (2013), un
% Se propuso un modelo de minería de datos inteligente que se puede utilizar para analizar,
% predecir y visualizar patrones de consumo de series temporales de energía. A
% La parte clave de este modelo fue el uso de "minería de patrones frecuentes"; frecuente
% Los patrones son conjuntos de elementos que aparecen en un conjunto de datos con una frecuencia igual.
% o más que un umbral especificado por el usuario. La minería de patrones frecuente es una
% Herramienta de minería de datos esencial para procesar y analizar big data.





