
%%%%%%%%%%%% Dedicatoria %%%%%%%%%%%%%%%%%%%%%%%%%%%%%%%%%%%%%%%%%%%%%%%%

\cleardoublepage %salta a nueva página impar
% Aquí va la dedicatoria si la hubiese. Si no, comentar la(s) linea(s) siguientes
\chapter*{}
\setlength{\leftmargin}{0.5\textwidth}
\setlength{\parsep}{0cm}
\addtolength{\topsep}{0.5cm}
\begin{flushright}
	\small\em{
		A mis padres siempre.
	}
\end{flushright}

%%%%%%%%%%%%%%%%%%%%%%%%%%%%%%%%%%%%%%%%%%%%%%%%%%%%%%%%%%%%%%%%%%%%%%%%
 

%%%%%%%%%%%% Agradecimientos %%%%%%%%%%%%%%%%%%%%%%%%%%%%%%%%%%%%%%%%%%%

\chapter*{Agradecimientos}

\thispagestyle{empty}
\vspace{1cm}

Me gustaría agradecer

\cleardoublepage %salta a nueva página impar

%%%%%%%%%%%%%%%%%%%%%%%%%%%%%%%%%%%%%%%%%%%%%%%%%%%%%%%%%%%%%%%%%%%%%%%%


%%%%%%%%%%%%  Resumen corto  %%%%%%%%%%%%%%%%%%%%%%%%%%%%%%%%%%%%%%%%%%%
\chapter{Resumen}
\thispagestyle{empty}
En este \gls{tfm} se presenta el diseño y desarrollo de diferentes modelos de \gls{ml} y \gls{dl} para la detección y predicción de errores en entornos de \textit{Smart Grids} (SG).

\vspace{3mm}

En este proceso se utilizan datos de una implementación de \gls{sg} real, los cuales requieren un procesamiento exhaustivo. Posteriormente, se generan una serie de escenarios y topologías aleatorias con la herramienta \acrshort{brite} y se aplican modificaciones en el funcionamiento del algoritmo \acrshort{den2ne}. Se pretende detectar y predecir errores sobre las rutas que se obtienen en el mismo y, por ello, se ejecutan múltiples simulaciones que permitan identificar los patrones de error sobre los que entrenar los modelos de forma precisa. El proyecto concluye con el análisis y evaluación de los resultados obtenidos de los modelos para definir el que proporciona un rendimiento óptimo.

\vspace{1cm}

\textbf{Palabras clave}: 
\href{https://scholar.google.com/scholar?q=smartgrids}{Smart Grids}; 
\href{https://scholar.google.com/scholar?hl=es&as_sdt=0,5&q=machine+learning}{Machine Learning};
\href{https://scholar.google.com/scholar?hl=es&as_sdt=0%2C5&q=deep+learning&btnG=}{Deep Learning}; 
\href{https://scholar.google.com/scholar?hl=es&as_sdt=0%2C5&q=big+data&btnG=}{Big Data};
\href{https://scholar.google.com/scholar?hl=es&as_sdt=0%2C5&q=prosumidor&btnG=}{Prosumidor}; 
\href{https://scholar.google.com/scholar?hl=es&as_sdt=0%2C5&q=iot&btnG=}{IoT}; 

\cleardoublepage %salta a nueva página impar

%%%%%%%%%%%%%%%%%%%%%%%%%%%%%%%%%%%%%%%%%%%%%%%%%%%%%%%%%%%%%%%%%%%%%%%%


%%%%%%%%%%%%  Resumen corto - Inglés %%%%%%%%%%%%%%%%%%%%%%%%%%%%%%%%%%%
\chapter{Abstract}
\thispagestyle{empty}

In this Master's Thesis (TFM), we present the design and development of various Machine Learning (ML) and Deep Learning (DL) models for detecting and predicting errors in Smart Grid (SG) environments. 

\vspace{3mm}

This process involves using real implementation data, which requires comprehensive preprocessing. Subsequently, a series of scenarios and random topologies are generated using the \acrshort{brite} tool, and modifications are applied to the \acrshort{den2ne} algorithm's operation. The aim is to detect and predict errors in the routes obtained, and therefore, multiple simulations are run to identify error patterns to accurately train the models. The project concludes with the analysis and evaluation of the results obtained from the models to determine the one that offers optimal performance.

\vspace{1cm}

\textbf{Keywords}: 
\href{https://scholar.google.com/scholar?q=smartgrids}{Smart Grids}; 
\href{https://scholar.google.com/scholar?hl=es&as_sdt=0,5&q=machine+learning}{Machine Learning};
\href{https://scholar.google.com/scholar?hl=es&as_sdt=0%2C5&q=deep+learning&btnG=}{Deep Learning}; 
\href{https://scholar.google.com/scholar?hl=es&as_sdt=0%2C5&q=big+data&btnG=}{Big Data};
\href{https://scholar.google.com/scholar?hl=es&as_sdt=0%2C5&q=prosumer&btnG=}{Prosumer}; 
\href{https://scholar.google.com/scholar?hl=es&as_sdt=0%2C5&q=iot&btnG=}{IoT}; 

\cleardoublepage %salta a nueva página impar

%%%%%%%%%%%%%%%%%%%%%%%%%%%%%%%%%%%%%%%%%%%%%%%%%%%%%%%%%%%%%%%%%%%%%%%%


%%%%%%%%%%%%%%%%%%%%%%%%%%%%  Cita   %%%%%%%%%%%%%%%%%%%%%%%%%%%%%%%%%%%
% Aquí va la cita célebre si la hubiese. Si no, comentar la(s) linea(s) siguientes
\chapter*{}
\setlength{\leftmargin}{0.5\textwidth}
\setlength{\parsep}{0cm}
\addtolength{\topsep}{0.5cm}
\begin{flushright}
	\small\em{
		"Cualquier tecnología lo suficientemente avanzada es indistinguible de la magia"
	}
\end{flushright}
\begin{flushright}
	\small{
		Arthur C. Clarke.
	}
\end{flushright}

\cleardoublepage %salta a nueva página impar

%%%%%%%%%%%%%%%%%%%%%%%%%%%%%%%%%%%%%%%%%%%%%%%%%%%%%%%%%%%%%%%%%%%%%%%%