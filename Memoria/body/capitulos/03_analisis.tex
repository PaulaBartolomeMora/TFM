\chapter{Diseño y análisis de datos}
\label{cha:analisis}

Este Capítulo abordará la fase del \gls{tfm} que se centra en el análisis, diseño y procesado de las fuentes de datos para lograr el objetivo principal de generar un conjunto de datos final que suponga la base del entrenamiento y el desarrollo de los modelos de \gls{ml} que se plantearán en el Capítulo \ref{cha:desarrollo}. Se abrirá el Capítulo con una Sección enfocada al estudio de la disponibilidad de fuentes de datos de implementaciones reales de \gls{sg}s. Por ello, será imprescindible llevar a cabo una investigación exhaustiva y un posterior análisis de su utilidad en referencia a las necesidades que presenta este \gls{tfm} (ver Sección \ref{sec:sustdata}).

\vspace{3mm}

Por consiguiente, se añadirá una Sección en referencia al estudio de la generación energética (ver Sección \ref{sec:simuprod}). Esta vendrá justificada por la necesidad de realizar un análisis sobre algunas herramientas de simulación existentes que proporcionan datos de potencia. Este proceso posibilitará obtener información adicional sobre los recursos solares y las condiciones climáticas de una ubicación determinada. Una vez expuestas y seleccionadas las fuentes de datos que servirán como base para el desarrollo de este \gls{tfm}, se añadirá una Sección dedicada a la definición detallada de las acciones que serán necesarias para llevar a cabo un procesamiento exhaustivo de toda la información recopilada (ver Sección~\ref{sec:preprocesado}). 

\vspace{3mm}

Después, siguiendo las especificaciones expuestas anteriormente en la Sección \ref{sec:brite}, se plantearán una serie de escenarios mediante la generación de múltiples topologías con la herramienta BRITE (ver Sección \ref{sec:ejebrite}). Los resultados que se obtengan a partir de ella se constituirán como la base sobre la que se realizarán las simulaciones en el algoritmo \gls{den2ne}. En esta Sección (ver Sección \ref{sec:cambiosden2ne}) será preciso detallar las modificaciones y ajustes a implementar en el algoritmo. Esto permitirá extraer a la salida del mismo el conjunto de datos final sobre el que se entrenarán los modelos de \gls{ml} y \gls{dl}.

\vspace{3mm}

% %%%%%%%%%%%%%%%%%%%%%%%%%%%%%%%%%%%%%%%%%%%%%%%%%%%%%%%%%%%%%%%%%%%%%%%%%%%%%%%%%%%%%%%%%%%%%%%%%
%%%%%%%%%%%%%%%%%%%%%%%%%%%%%%%%%%%%%%%%%%%%%%%%%%%%%%%%%%%%%%%%%%
\section{Análisis de fuentes de datos}
\label{sec:sustdata}

%aqui explicar los datasets que se han visto
%enfocarlo mas a todos lo datasets o a sustdata?
%aqui habria que explicar tambien la creacion de los datos de produccion 

% %%%%%%%%%%%%%%%%%%%%%%%%%%%%%%%%%%%%%%%%%%%%%%%%%%%%%%%%%%%%%%%%%%%%%%%%%%%%%%%%%%%%%%%%%%%%%%%%%
%%%%%%%%%%%%%%%%%%%%%%%%%%%%%%%%%%%%%%%%%%%%%%%%%%%%%%%%%%%%%%%%%%
\section{Simulación de datos de producción}
\label{sec:simuprod}

Como se ha expuesto en el Apartado anterior de conclusiones del dataset (ver Sección \ref{sec:conclusionessustdata}), en la siguiente etapa de este \gls{tfm}, dedicada al preprocesamiento de los datos (ver Sección \ref{sec:preprocesado}), se requerirá evaluar la precisión y la utilidad de los datos de producción proporcionados por \textit{SustDataED}. Para ello, la presente Sección viene definida por la necesidad de obtener una fuente de datos de producción energética adicional, que permita contrastar la información adquirida del dataset anteriormente.

\vspace{3mm}

Atendiendo a este fin, se procederá a emplear varias herramientas de análisis y simulación de datos de producción energética, detallándose a su vez, sus características de funcionamiento y la configuración necesaria. Cabe destacar que el proceso de simulación de los datos que se va a llevar a cabo implica recrear un dataset que sea riguroso con la realidad.

\vspace{3mm}

Por lo tanto, en primera instancia, se deberá llevar a cabo un estudio y un análisis específico de las características geográficas y climáticas a largo plazo de la localización que se tomará como base. Por consiguiente, se cuantificarán los datos mediante una herramienta de simulación en función del estudio anterior. Finalmente, se analizarán los resultados y se evaluará su precisión, con el fin de procesarlos y combinarlos con los datos adquiridos de \textit{SustDataED} (ver Sección \ref{sec:procprod}). Es decir, será en esta fase de preprocesamiento donde se tome la decisión final de selección de la fuente de datos de producción más adecuada para el desarrollo de este \gls{tfm}.

\subsection{Estudio y análisis de la ubicación (\textit{Global Solar Atlas})}
\label{sec:global}

En cuanto a la ubicación, en este caso, será preciso basarse concretamente en Funchal, capital de la isla de Madeira (Portugal), ya que el análisis a realizar tiene que ser acorde a los datos recogidos en el dataset \textit{SustDataED}. 

\vspace{3mm}

Según las clasificaciones climáticas de Köppen \cite{koppen} o Trewartha \cite{wikitre}, la ciudad de Funchal se caracteriza por tener un clima mediterráneo con influencia oceánica o templado con verano seco (Csb) \cite{trewartha} \cite{wikimadeira} \cite{aemet}. Al localizarse en una zona subtropical, presenta oscilaciones diarias mínimas, lo que se traduce en escasos cambios de temperatura entre las diferentes estaciones del año. Por ello, los inviernos son suaves y con precipitaciones moderadas, mientras que los veranos tienden a ser ligeramente más cálidos y secos. 

\vspace{3mm}

Tomando en consideración lo anterior, para conocer el potencial de los recursos solares que presenta la ubicación se va a proceder al empleo del modelo solar \textit{Solargis}, proporcionado por la plataforma online \textit{Global Solar Atlas}\footnote{https://globalsolaratlas.info/map} \cite{globalsolar} \cite{energydata}. Esta plataforma es financiada por el \gls{esmap} y administrada por la organización de El Banco Mundial con el objetivo de mapear los recursos de energía renovable a nivel global, permitiendo el acceso a una gran cantidad mapas y de datos promediados a largo plazo y en tiempo real de cualquier punto de la Tierra. 

\vspace{3mm}

La información sobre los recursos solares y la cuantificación de la energía se suministran a través de esta plataforma siguiendo el estándar GIS ráster \cite{gis} o cuadriculado con formatos GeoTIFF \cite{geotiff} o AAIGRID \cite{aaigrid}. Para las diferentes capas de datos que se pueden determinar en función de los parámetros de radiación o del potencial fotovoltaico, se sigue una referencia espacial geográfica en base al código EPSG 4326 \cite{epsg}. Este código es asignado por la organización \gls{epsg} para identificar la proyección geográfica empleada, la cual en este caso hace referencia al sistema de coordenadas convencional (latitud-longitud) que se utiliza para la representación cartográfica de la Tierra. Por otro lado, los metadatos correspondientes a las características de cada capa se proveen en formato PDF o XML, siguiendo la estructura de datos geográficos definida por ISO 19115:2003/19139~\cite{globalsolar} \cite{globalsolarreport}.

\subsubsection{Identificación de capas de datos}

Considerando las motivaciones que se persiguen con el empleo de la plataforma \textit{Global Solar Atlas}, es preciso realizar un paso previo al análisis, basado en la identificación de los tipos de capas de datos que se pueden configurar \cite{globalsolarreport}:

\begin{itemize}    
    \item \gls{dni} (kWh/m²): El índice de radiación directa normal se define como la cantidad de radiación solar que llega perpendicularmente a la superficie de la placa fotovoltaica, sin tener en cuenta los posibles efectos atmosféricos de dispersión o absorción.
    \item \gls{dif} (kWh/m²): El índice de radiación difusa hace referencia a la porción de radiación dispersada por las nubes y los gases atmosféricos, lo que produce que provenga de todas las direcciones.
    \item \gls{ghi} (kWh/m²): El índice de radiación horizontal global viene dado por el sumatorio de la radiación solar directa (\gls{dni}) y la radiación solar difusa dispersada en consecuencia a los efectos de la atmósfera (\gls{dif}).
    \item \gls{gti} (kWh/m²): El índice de radiación global inclinada se refiere al total de radiación que incide en una superficie inclinada, que generalmente se encuentra ajustada un ángulo óptimo para maximizar la captación en términos anuales. 
    \item \gls{pvout} (kWh/kWp): El parámetro que mide el potencial energético de los sistemas fotovoltaicos de una ubicación determinada se cuantifica a partir de un sistema de referencia construido por módulos de silicio cristalino, de 1kWp (kilovatio pico) y que se encuentra inclinado un ángulo óptimo.
\end{itemize}

Es necesario indicar que todos los parámetros anteriores se proporcionan para cada ubicación como valores promedios anuales de los totales diarios. Adicionalmente, otras capas a tener en cuenta para el análisis de los datos podrían ser la temperatura del aire, que determina en gran medida el ambiente de operación de las placas fotovoltaicas, o la elevación del terreno, que puede convertirse en un factor limitante para la instalación de plantas solares. 

\vspace{3mm}

En las Figuras \ref{fig:dni} y \ref{fig:ghi}, se visualizan en formato GeoTIFF los resultados de aplicar las capas de los índices de radiación \gls{dni} y \gls{ghi} al mapa mundial a través de la plataforma \textit{Global Solar Atlas}. Por otro lado, en la Figura \ref{fig:photo}, se representa el potencial fotovoltaico dado por el parámetro \gls{pvout}. Como es de esperar, se puede verificar que existe una gran correlación entre los niveles de radiación recibidos con respecto a la cantidad de energía que se puede generar con una planta solar en una ubicación determinada. 

\vspace{3mm}

\begin{figure}[H]
    \centering
    \includegraphics[width=1\textwidth]{img/diseno/dni.png}
    \caption{Mapa mundial del índice de radiación directa normal (\acrshort{dni}) mundial \cite{globalsolar}}
    \label{fig:dni}
\end{figure}

\vspace{3mm}

\begin{figure}[H]
    \centering
    \includegraphics[width=1\textwidth]{img/diseno/ghi.png}
    \caption{Mapa del índice de radiación horizontal global (\acrshort{ghi}) mundial \cite{globalsolar}}
    \label{fig:ghi}
\end{figure}

\vspace{3mm}

\pagebreak

\begin{figure}[H]
    \centering
    \includegraphics[width=1\textwidth]{img/diseno/photovoltaic.png}
    \caption{Mapa del potencial de producción energética fotovoltaica (\acrshort{pvout}) mundial \cite{globalsolar}}
    \label{fig:photo}
\end{figure}

\subsubsection{Análisis de resultados de radiación}

De la misma forma, lo anterior se hace patente para el caso de la isla de Madeira, como se puede visualizar en las Figuras \ref{fig:madeiradni}, \ref{fig:madeiraghi} y \ref{fig:madeirapvout}. Desde la plataforma \textit{Global Solar Atlas} también se extraen y se recogen en la Tabla \ref{tab:global} los valores de los índices de radiación, configurados específicamente para la localización de la ciudad de Funchal. 

\vspace{5mm}

\begin{table}[h!]
    \centering
    \begin{tabular}{|c|c|c|}
    \hline
    \rowcolor[HTML]{AAAAAA} 
    \multicolumn{1}{|c|}{\cellcolor[HTML]{AAAAAA}Parámetro} & \multicolumn{1}{c|}{\cellcolor[HTML]{AAAAAA}kWh/m²/día} & kWh/m²/año \\ \hline
    \gls{dni} & 3,698 & 1349,9 \\ \hline
    \gls{dif} & 2,038 & 743,8 \\ \hline
    \gls{ghi} & 4,345 & 1586,1 \\ \hline
    \gls{gti} & 4,730 & 1726,3 \\ \hline
    \end{tabular}
    \caption{Tabla de valores extraídos para cada índice de radiación en la ciudad de Funchal~\cite{globalsolar}}
    \label{tab:global}
\end{table}

\vspace{3mm}

Por tanto, se puede comprobar a través de estos valores y de la leyenda proporcionada en las figuras anteriores, que Funchal se percibe como una ciudad con un potencial de recursos solares medio alto. Esto es coherente con su localización subtropical y con la elevación del terreno. Es decir, al encontrarse en una zona de costa, ronda en valores cercanos a los 100 metros (concretamente 137 metros en la ubicación seleccionada) y se reduce ligeramente el nivel de energía solar captable respecto a otras zonas de la isla más montañosas.

\pagebreak

\begin{figure}[H]
    \centering
    \includegraphics[width=1\textwidth]{img/diseno/madeiradni.png}
    \caption{Mapa del índice de radiación directa normal (\acrshort{dni}) de Madeira \cite{globalsolar}}
    \label{fig:madeiradni}
\end{figure}

\begin{figure}[H]
    \centering
    \includegraphics[width=1\textwidth]{img/diseno/madeiraghi.png}
    \caption{Mapa del índice de radiación horizontal global (\acrshort{ghi}) de Madeira \cite{globalsolar}}
    \label{fig:madeiraghi}
\end{figure}

\begin{figure}[H]
    \centering
    \includegraphics[width=1\textwidth]{img/diseno/madeirapvout.png}
    \caption{Mapa del potencial de producción energética fotovoltaica (\acrshort{pvout}) de Madeira \cite{globalsolar}}
    \label{fig:madeirapvout}
\end{figure}

\pagebreak

\subsubsection{Configuración y cuantificación del potencial energético de la ubicación}

Para calcular de forma aproximada la cantidad de energía que supondría la instalación de un sistema fotovoltaico en una vivienda ubicada en la ciudad de Funchal, es preciso configurar primero los siguientes parámetros a través de la plataforma:

\begin{itemize}
    \item Tipo y tamaño de sistema fotovoltaico: Se permite seleccionar entre un sistema enfocado a una vivienda, a un edificio comercial o a una planta solar de grandes dimensiones. En este caso, se toma como base el estudio de producción energética en un entorno residencial. Es preciso indicar que en la plataforma la configuración de un sistema fotovoltaico para un entorno residencial no considera la opción de almacenamiento de electricidad.
    \item Capacidad de instalación: Se toma una capacidad máxima de generación de 4kWp (kilovatios pico) para el sistema. En otros términos, este valor determina la cantidad de energía que producirían los paneles en condiciones óptimas.
    \item Azimut \cite{azimut}: Se determina como el ángulo de orientación horizontal y, en función de su valor, define la proyección de los paneles solares en dirección norte, sur, este u oeste. Se indica un valor de 180º para establecer una orientación hacia el sur. 
    \item Inclinación: La plataforma establece como ángulo óptimo para la localización de Funchal un ángulo de 26º, por lo que se configura la instalación de los paneles solares sobre rieles sujetos a un tejado con esta misma inclinación.
\end{itemize}

\subsubsection{Análisis de resultados de potencia}
\label{sec:analisisglobal}

Una vez configurados los parámetros expuestos, se obtienen como resultados la trayectoria solar diaria y los perfiles de radiación y generación fotovoltaica diarios y mensuales para la ubicación de Funchal. En la Figura \ref{fig:azimut} se representa la trayectoria solar diaria, en la que se puede visualizar la comparación de la elevación que percibe el sol en función del instante del año, siendo máxima durante el solsticio de junio, y mínima, durante el de diciembre. En el caso del equinoccio, se sigue una curva con valores promedios, al existir una igualdad entre las horas de día y las de noche. 

\vspace{3mm}

Por otro lado, en la Figura \ref{fig:average}, se modelan gráficamente los valores totales mensuales, tanto de producción energética, como de radiación directa normal \gls{dni} para visualizar la correlación existente entre los mismos en los distintos meses del año. El valor promedio máximo energético se alcanza en el mes de julio, con un valor de 565kWh, ocurriendo de la misma forma para el valor máximo de radiación, cuyo alcance es de 163,7kWh/m² para este mes.

\begin{figure}[H]
    \centering
    \includegraphics[width=0.9\textwidth]{img/diseno/azimut.png}
    \caption{Representación de la trayectoria solar percibida en la localización de la ciudad de Funchal \cite{globalsolar}}
    \label{fig:azimut}
\end{figure}

\vspace{3mm}

\begin{figure}[H]
    \centering    
    \begin{subfigure}{0.5\linewidth}
        \centering
        \includegraphics[width=\linewidth,height=5cm]{img/diseno/averagepvout.png}
        \label{fig:averagepvout}
    \end{subfigure}\hfill
    \begin{subfigure}{0.5\linewidth}
        \centering
        \includegraphics[width=\linewidth,height=5cm]{img/diseno/averagedni.png}
        \label{fig:averagedni}
    \end{subfigure}    
    \caption{Comparación de valores totales mensuales de producción energética fotovoltaica (\acrshort{pvout}) respecto a la radiación directa normal (\acrshort{dni}) \cite{globalsolar}}
    \label{fig:average}
\end{figure}

De la misma forma, se representan en la Figura \ref{fig:averagehour} los perfiles diarios propromedios de radiación y generación fotovoltaica en función del mes del año, en los cuales se encuentran valores máximos en las horas centrales de los meses de julio y agosto. 

\pagebreak

\begin{figure}[H]
    \centering    
    \begin{subfigure}{0.75\linewidth}
        \centering
        \includegraphics[width=\linewidth]{img/diseno/averagepvouthour.png}
        \label{fig:averagepvouthour}
    \end{subfigure}\hfill

    \begin{subfigure}{0.75\linewidth}
        \centering
        \includegraphics[width=\linewidth]{img/diseno/averagednihour.png}
        \label{fig:averagepdnihour}
    \end{subfigure}    
    \caption{Perfiles promedios de potencial de producción energética fotovoltaica (\acrshort{pvout}) y de radiación directa normal (\acrshort{dni}) \cite{globalsolar}}
    \label{fig:averagehour}
\end{figure}

\pagebreak

Por lo tanto, tomando en consideración los parámetros configurados y los resultados obtenidos, se puede cuantificar finalmente, que la instalación de un sistema fotovoltaico en un edificio residencial de Funchal proporcionaría en un año una generación de energía promedia igual a 5433kWh y un valor acumulado de \gls{dni} equivalente a 1719,4 kWh/m².

\vspace{3mm}

%ver pag 26-31 technical report -> losses


% Total photovoltaic power output and Global tilted irradiation (4)
% 0.015MWh per day 4.711kWh/m2 per day
% 5433kWh per year 1719.4 kWh/m2 per year

% \begin{figure}[H]
%     \centering
%     \includegraphics[width=0.8\textwidth]{img/diseno/orient.jpg}
%     \caption{Geometría solar para la instalación de paneles solares \cite{azimut}}
%     \label{fig:orient}
% \end{figure}

% •	Ver pdf PV Systems Performance in Madeira, gráficas interesantes, indica zonas de madeira con mayor producción
% •	Ver Solar Energy Resource in Madeira Islands, gráficas interesantes


\subsection{Creación del dataset de producción (\textit{PVWatts})}
\label{sec:pvw}

Una vez que se han concretado a modo de análisis las características geográficas, climáticas y energéticas de la localización seleccionada en la ciudad de Funchal, se procederá a realizar el paso dedicado a la obtención de los datos de producción fotovoltaica por medio de la simulación. Para ello, se hará uso de la herramienta online \textit{PVWatts}\footnote{https://pvwatts.nrel.gov/pvwatts.php} \cite{pvwatts}, proporcionada por el \gls{nrel}, que se trata del laboratorio nacional del Departamento de Energía de Estados Unidos. 

\vspace{3mm}

De la misma forma que se ha operado con la plataforma \textit{Global Solar Atlas} (ver Sección \ref{sec:global}), en el caso de \textit{PVWatts}, también se requiere indicar a la entrada las características relativas a la ubicación seleccionada y a la configuración del sistema de paneles fotovoltaicos que se desea. 

\vspace{3mm}

En cuanto a la primera, la herramienta \textit{PVWatts}, a diferencia de \textit{Global Solar Atlas}, provee un mapa mundial que se rige por celdas, en función de la localización de las bases de datos del \gls{nrel}. En el caso de este \gls{tfm}, se cuenta con la fortuna de que se dispone de una de estas bases en la misma ciudad de Funchal, concretamente en las coordenadas 32,65°N, 16,92°W. En este contexto, es preciso comentar que esta característica se tuvo en cuenta para definir la localización exacta de instalación del sistema fotovoltaico en la plataforma \textit{Global Solar Atlas} (ver Sección \ref{sec:global}). Por lo tanto, no se requieren ajustes adicionales en estos términos.  

\subsubsection{Configuración de los parámetros de entrada}

Con el objetivo de conseguir a la salida de la simulación unos datos que sean coherentes y acordes al análisis realizado anteriormente en la Sección \ref{sec:analisisglobal}, es imprescindible replicar y ajustar de la forma más precisa posible los valores de los parámetros de entrada de la herramienta: 

\pagebreak

\begin{itemize}
    \item Tamaño del sistema fotovoltaico y capacidad de instalación: Por defecto, \textit{PVWatts} determina una capacidad de 4kWp, por lo que coincide directamente con la definida en \textit{Global Solar Atlas}. Además, se especifica la relación existente entre este valor y el tamaño del sistema de la siguiente manera:
    \[\frac{4 \, \text{kW}}{1 \, \text{kW/m}^2 \times 0,16} = 25 \, \text{m}^2\]
    Donde se representa que, para obtener una eficiencia del 16\% del sistema (eficiencia por defecto), se requiere un área de 25 m² para los módulos solares. Es decir, este valor no representa el área total del sistema fotovoltaico, ya que para ello habría que añadir el cálculo del espacio que se necesita establecer entre cada uno de los paneles DC o el tamaño de los inversores AC que se instalarían.
    \item Índice de cobertura del suelo: Se define como la relación entre el área de superficie del módulo y el área del suelo (tejado en el caso de una vivienda) ocupado en total. El valor predeterminado es de 0,4, lo que supone que, teniendo un área efectiva de 25 m², se necesitará un área total de 62,5 m².
    \item Ratio DC/AC: Haciendo referencia al tamaño de los inversores AC, es preciso exponer que estos limitan la salida de la energía del conjunto para que haya un correcto funcionamiento. La herramienta cuantifica por defecto para la localización, una relación 1,2 entre el tamaño de array de paneles y los inversores de corriente alterna. Por lo tanto, en este caso, estos contarían con una capacidad teórica de 3,33kW. Si se pretendiera realizar una instalación a gran escala se necesitarían ratios de hasta 1,5.
    \item Azimut: De la misma forma que en la configuración en la plataforma \textit{Global Solar Atlas}, se define un azimut de 180º, orientando los paneles hacia el sur.
    \item Inclinación: \textit{Global Solar Atlas}, en el caso de localizar el sistema fotovoltaico en la ciudad de Funchal, establecía como ángulo óptimo de inclinación un valor de 26º.
    \item Tipo de módulos: Se selecciona el tipo estándar, basado en silicio cristalino y con una cobertura de vidrio con revestimiento antireflectante. Aproximadamente, cuenta con una eficiencia nominal del 19\%.
    \item Tipo de array: Se especifica de forma simplificada un tipo de array fijo, que no siga el movimiento del sol mediante ejes de rotación, sino que mantenga sus valores de inclinación y de azimut configurados. En la Figura \ref{fig:array}, se describe el funcionamiento de cada tipo de array.
\end{itemize}


\begin{figure}[H]
    \centering
    \includegraphics[width=0.9\textwidth]{img/diseno/array.png}
    \caption{Representación de las diferentes opciones de array que permite configurar la herramienta \textit{PVWatts} \cite{pvwatts}}
    \label{fig:array}
\end{figure}

\subsubsection{Configuración de las pérdidas del sistema real}

Una de las ventajas de uso que permite la herramienta \textit{PVWatts} es la posibilidad de estimar las pérdidas totales que tendría el sistema en la realidad. Esto es importante para que los resultados de la simulación que se va a realizar sean lo más parecidos a los que se obtendrían mediante la medición del sistema real. Para ello, se deben configurar cada una de las pérdidas parciales de la siguiente manera:

\begin{itemize}
    \item Suciedad: Debido a la escasez de precipitaciones y a la presencia de calima que se experimenta en la isla, se determina un valor de un 2\%.
    \item Sombreado: Se reduce la radiación solar que incide en los paneles si se producen sombras por objetos cercanos o por la propia de los módulos si se colocan en fila. Es decir, estos crean sombras sobre los de la fila adyacente. El cálculo se realiza a partir de la gráfica expuesta en la Figura \ref{fig:sombra}. Por lo tanto, como se ha definido un tipo de array fijo con una inclinación de 26º y un valor de índice de cobertura del suelo de 0,4, se determina aproximadamente un valor de pérdidas por sombra igual a un 3\%.
    \item Nieve: Teniendo en cuenta las características climáticas de la ciudad de Funchal, debido a su localización en una zona subtropical, se determinan pérdidas del 0\%.
    \item Discordancia: Son pérdidas eléctricas debido a las diferencias que se producen por imperfecciones en la fabricación de los módulos. Esto tiene como consecuencia que cada uno de ellos presente características I-V que varíen ligeramente entre sí. La herramienta proporciona un valor predeterminado del 2\%.
    \pagebreak
    \item Cableado y conexiones: Pérdidas resistivas en el cableado y en las conexiones entre módulos, inversores y otros elementos del sistema con un valor por defecto del 2\% y del 0,5\%, respectivamente.
    \item Degradación inducida por la luz: Se trata del efecto de reducción de potencia que se produce durante los primeros meses de funcionamiento del sistema fotovoltaico. El valor predeterminado es 1,5\%.
    \item Precisión del fabricante: Son pérdidas, generalmente bajas, que vienen definidas por la precisión del fabricante en el proceso de estudio de eficiencia de los paneles. Se determina un 1\%.
    \item Disponibilidad: Se añaden las pérdidas que se producen a partir de los mantenimientos que se requieren a lo largo del tiempo y que causan paradas en el funcionamiento del sistema. Se provee un valor por defecto del 3\%.
\end{itemize}

Finalmente, las pérdidas totales del sistema se pueden cuantificar a partir de la siguiente expresión:
    \[\textit{Losses} = (1-0,02) \times (1-0,03) \times (1-0,02) \times (1-0,02) 
    \times (1-0,005) \times (1-0,015) \times (1-0,01) \times (1-0,03)\]
    \[100\% \times (1-\textit{Losses}) = 14,08\%\]

\begin{figure}[H]
    \centering
    \includegraphics[width=0.85\textwidth]{img/diseno/sombra.png}
    \caption{Gráfica para el cálculo de las pérdidas por sombreado en la herramienta \textit{PVWatts}~\cite{pvwatts}}
    \label{fig:sombra}
\end{figure}

\subsubsection{Análisis de resultados de la simulación}
\label{sec:resultadossimu}

Una vez se ha concluido con la configuración necesaria, se procede a ejecutar la simulación en la herramienta y a adquirir a la salida los resultados de la misma. Se obtienen dos datasets que replican un año completo de mediciones: uno con los valores totales mensuales (ver Tabla \ref{tab:pvwattsdataset2}) y otro, con un conjunto de muestras que proporcionan información de cada hora (ver Tabla \ref{tab:pvwattsdataset}). 

\vspace{3mm}

Como se puede visualizar, en ambas tablas se proporcionan los valores de potencia DC y AC, que corresponden a la salida del módulo solar (\textit{DC Array Output}) y de los inversores (\textit{AC System Output}), respectivamente. En el caso de la Tabla (ver Tabla \ref{tab:pvwattsdataset}), se incluye de forma adicional toda la información climática y ambiental de cada instante.

\vspace{3mm}

\begin{table}[h!]
    \centering
    \begin{tabular}{|c|c|c|}
    \hline
    \rowcolor[HTML]{AAAAAA} 
    \multicolumn{1}{|c|}{\cellcolor[HTML]{AAAAAA}Campo} & \multicolumn{1}{c|}{\cellcolor[HTML]{AAAAAA}Descripción} & Unidades \\ \hline
    \textit{Month} & Mes de la muestra & - \\ \hline
    \textit{Daily POA Irradiance} & Índice de radiación en el plano del array & kWh/m2/day \\ \hline 
    \textit{DC Array Output} & Potencia de salida DC del array & kWh \\ \hline
    \textit{AC System Output} & Potencia de salida AC del sistema & kWh \\ \hline
    \end{tabular}
    \caption{Dataset con valores totales mensuales \cite{pvwatts}}
    \label{tab:pvwattsdataset2}
\end{table}

\vspace{1mm}

\begin{table}[h!]
    \centering
    \begin{tabular}{|c|c|c|}
    \hline
    \rowcolor[HTML]{AAAAAA} 
    \multicolumn{1}{|c|}{\cellcolor[HTML]{AAAAAA}Campo} & \multicolumn{1}{c|}{\cellcolor[HTML]{AAAAAA}Descripción} & Unidades \\ \hline
    \textit{Month} & Mes de la muestra & - \\ \hline
    \textit{Day} & Día de la muestra & - \\ \hline
    \textit{Hour} & Hora de la muestra & - \\ \hline
    %\textit{Beam Irradiance} & Índice de radiación directa normal (\gls{dni}) & W/m2 \\ \hline 
    \textit{Diffuse Irradiance} & Índice de radiación difusa (\gls{dif}) & W/m2 \\ \hline
    \textit{Plane of Array Irradiance} & Índice de radiación en el plano del array (\acrshort{poa}) & W/m2 \\ \hline 
    \textit{Ambient Temperature} & Temperatura ambiente & C \\ \hline
    \textit{Wind Speed} & Velocidad del viento & m/s \\ \hline
    \textit{Albedo} & Índice de reflectividad de la superficie & - \\ \hline
    \textit{Cell Temperature} & Temperatura de las células solares & C \\ \hline
    \textit{DC Array Output} & Potencia de salida DC del array & W \\ \hline
    \textit{AC System Output} & Potencia de salida AC del sistema & W \\ \hline
    \end{tabular}
    \caption{Dataset con muestras adquiridas por hora \cite{pvwatts}}
    \label{tab:pvwattsdataset}
\end{table}

\vspace{3mm}

Teniendo en cuenta las tablas anteriores, en este paso será imprescindible comprobar que los datos finales obtenidos presentan coherencia respecto al estudio previo y que encajan además, con el análisis de los resultados de la plataforma \textit{Global Solar Atlas}, realizado en la Sección \ref{sec:analisisglobal}. 

\vspace{3mm}

Por ello, en primera instancia, se va a proceder a poner el foco en los valores de radiación directa normal \gls{dni}. La herramienta \textit{PVWatts}, como se puede ver en la Tabla \ref{tab:pvwattsdataset2}, no proporciona directamente información mensual sobre este parámetro, sino sobre el índice de radiación en el plano del array (del inglés \gls{poa}), que se calcula a partir de la expresión \cite{poa}:

\begin{equation}
    \gls{poa} = \gls{dni} \cdot \cos(\theta) 
\end{equation}

    Donde:
\begin{itemize}
    \renewcommand{\labelitemi}{}
    \item \( \theta \) es el ángulo de inclinación configurado, que en este caso sería de 26º.
\end{itemize}

\vspace{3mm}

Por lo tanto, se pueden calcular los valores de \gls{dni} a partir de los de \gls{poa} dados por el dataset. Mediante la librería \textit{matplotlib} de \textit{Python}, se representa la Figura \ref{fig:averagereal} para visualizar los valores totales mensuales, tanto de producción energética, como de radiación directa normal \gls{dni}. El valor máximo energético se alcanza en el mes de julio, con un valor de 517,8kWh, ocurriendo de la misma forma para el valor promedio máximo de radiación, cuyo alcance es de 181,1kWh/m² para este mes. Por tanto, volviendo a la Sección \ref{sec:analisisglobal} y, específicamente a la \ref{fig:average}, se puede determinar que los valores obtenidos en la simulación son muy precisos respecto al estudio previo.

\vspace{3mm}

\begin{figure}[H]
    \centering    
    \begin{subfigure}{0.5\linewidth}
        \centering
        \includegraphics[width=\linewidth,height=6.5cm]{img/diseno/averagepvoutreal.png}
        \label{fig:averagepvoutreal}
    \end{subfigure}\hfill
    \begin{subfigure}{0.5\linewidth}
        \centering
        \includegraphics[width=\linewidth,height=6.5cm]{img/diseno/averagednireal.png}
        \label{fig:averagednireal}
    \end{subfigure}    
    \caption{Comparación de valores totales mensuales de producción energética fotovoltaica (\acrshort{pvout}) respecto a la radiación directa normal (\acrshort{dni}) a partir de los resultados de la simulación}
    \label{fig:averagereal}
\end{figure}

\pagebreak

Aparte de las gráficas realizadas, se expone la Tabla \ref{tab:globalvspvwatts} para llevar a cabo de una forma correcta la comparación entre los valores estudiados en \textit{Global Solar Atlas} y los resultantes de la simulación en \textit{PVWatts}. Como se puede ver, existe un menor error entre los valores respectivos a la radiación que entre los valores de potencia producida. Las diferencias percibidas provienen principalmente por las pérdidas configuradas, ya que al final los resultados de la simulación son mucho más cercanos al caso real que los percibidos teóricamente. 

\begin{sidewaystable}
    \centering 
    \begin{tabularx}{\textheight}{|c|XX|XX|}
        \hline
        \rowcolor[HTML]{EFEFEF} 
        \cellcolor[HTML]{C0C0C0} & \multicolumn{2}{c|}{\cellcolor[HTML]{C0C0C0}Producción energética fotovoltaica (PVOUT) [kWh]} & \multicolumn{2}{c|}{\cellcolor[HTML]{C0C0C0}Índice de Radiación Directa Normal (DNI) [kWhm2]} \\ \cline{2-5} 
        \rowcolor[HTML]{C0C0C0} 
        \multirow{-2}{*}{\cellcolor[HTML]{C0C0C0}Mes} & \multicolumn{1}{c|}{\cellcolor[HTML]{C0C0C0}Global Solar Atlas} & PVWatts & \multicolumn{1}{c|}{\cellcolor[HTML]{C0C0C0}Global Solar Atlas} & PVWatts \\ \hline
        \textit{Enero} & \multicolumn{1}{c|}{350,2} & \multicolumn{1}{c|}{247,1} & \multicolumn{1}{c|}{83,7} & \multicolumn{1}{c|}{81,1} \\ \hline
        \textit{Febrero} & \multicolumn{1}{c|}{374,0} & \multicolumn{1}{c|}{291,4} & \multicolumn{1}{c|}{85,5} & \multicolumn{1}{c|}{105,7} \\ \hline
        \textit{Marzo} & \multicolumn{1}{c|}{500,8} & \multicolumn{1}{c|}{401,9} & \multicolumn{1}{c|}{117,9} & \multicolumn{1}{c|}{132,8} \\ \hline
        \textit{Abril} & \multicolumn{1}{c|}{497,9} & \multicolumn{1}{c|}{441,3} & \multicolumn{1}{c|}{117,9} & \multicolumn{1}{c|}{155,0} \\ \hline
        \textit{Mayo} & \multicolumn{1}{c|}{526,5} & \multicolumn{1}{c|}{505,9} & \multicolumn{1}{c|}{137,3} & \multicolumn{1}{c|}{172,7} \\ \hline
        \textit{Junio} & \multicolumn{1}{c|}{502,5} & \multicolumn{1}{c|}{449,6} & \multicolumn{1}{c|}{135,6} & \multicolumn{1}{c|}{157,4} \\ \hline
        \textit{Julio} & \multicolumn{1}{c|}{565,5} & \multicolumn{1}{c|}{517,8} & \multicolumn{1}{c|}{163,7} & \multicolumn{1}{c|}{181,1} \\ \hline
        \textit{Agosto} & \multicolumn{1}{c|}{558,6} & \multicolumn{1}{c|}{505,5} & \multicolumn{1}{c|}{151,6} & \multicolumn{1}{c|}{176,1} \\ \hline
        \textit{Septiembre} & \multicolumn{1}{c|}{465,3} & \multicolumn{1}{c|}{417,4} & \multicolumn{1}{c|}{109,3} & \multicolumn{1}{c|}{148,7} \\ \hline
        \textit{Octubre} & \multicolumn{1}{c|}{429,0} & \multicolumn{1}{c|}{356,3} & \multicolumn{1}{c|}{100,2} & \multicolumn{1}{c|}{119,5} \\ \hline
        \textit{Noviembre} & \multicolumn{1}{c|}{343,7} & \multicolumn{1}{c|}{268,8} & \multicolumn{1}{c|}{80,2} & \multicolumn{1}{c|}{90,1} \\ \hline
        \textit{Diciembre} & \multicolumn{1}{c|}{318,6} & \multicolumn{1}{c|}{243,4} & \multicolumn{1}{c|}{76,9} & \multicolumn{1}{c|}{78,6} \\ \hline
    \end{tabularx}
    \caption{Tabla de comparación de los valores totales mensuales de producción energética fotovoltaica (\acrshort{pvout}) y de radiación directa normal (\acrshort{dni}) de las dos herramientas (valores teóricos y simulados)}
    \label{tab:globalvspvwatts}
\end{sidewaystable}


\vspace{3mm}

Por lo tanto, a través del sumatorio de todos los valores mensuales de producción energética, se puede cuantificar finalmente, que un sistema fotovoltaico real localizado en la ciudad de Funchal, proveería aproximadamente un total de 4417kWh al año. Este valor representa un 81,3\% del obtenido teóricamente (5433kWh) en la Sección \ref{sec:analisisglobal} y viene justificado en gran parte por las pérdidas introducidas. Por otro lado, en el caso del \gls{dni}, con la herramienta \textit{PVWatts} se obtiene un total anual equivalente a 1598,8 kWh/m², suponiendo un 92,9\% del valor teórico (1719,4 kWh/m²).

\vspace{3mm}

Es preciso tener en cuenta las diferencias que presentan las herramientas \textit{Global Solar Atlas} y \textit{PVWatts} en sus algoritmos de cálculo, ya que cada una de ellas asignará pesos diferentes a los parámetros de entrada, incidiendo parcialmente en los resultados. No obstante, a partir de los valores anteriores y de la Tabla \ref{tab:globalvspvwatts}, se puede determinar de forma concluyente que la simulación realizada es coherente con el estudio previo, cumpliendo así el objetivo principal de esta Sección.

\clearpage 

% %%%%%%%%%%%%%%%%%%%%%%%%%%%%%%%%%%%%%%%%%%%%%%%%%%%%%%%%%%%%%%%%%%%%%%%%%%%%%%%%%%%%%%%%%%%%%%%%%
%%%%%%%%%%%%%%%%%%%%%%%%%%%%%%%%%%%%%%%%%%%%%%%%%%%%%%%%%%%%%%%%%%
\section{Preprocesado de los datos}
\label{sec:preprocesado}

En las Secciones anteriores, se ha detallado el proceso de búsqueda y recolección de datos energéticos en entornos residenciales a partir de las fuentes públicas disponibles. Este proceso ha venido acompañado por un estudio y análisis en profundidad para exponer de forma justificada las razones de la selección de los conjuntos de datos que servirán como base de este \gls{tfm}. 

\vspace{3mm}

También, se ha llevado a cabo una investigación enfocada en dos herramientas de extracción de información geográfica, climática y energética. El fin de este proceso ha sido incorporar una nueva fuente de datos con parámetros relacionados con la producción de electricidad, la cual se ha construido a partir de la simulación de un caso real.

\vspace{3mm}

Entonces, se puede expresar que la ejecución de los procesos anteriores proporcionará múltiples ficheros de datos que soporten extensas cantidades de información. En consecuencia, estos ficheros se caracterizarán por su gran tamaño y su dificultad de manejo y empleo. Por ello, se introduce esta Sección con el fin principal de realizar un procesamiento exhaustivo y reducir los conjuntos de datos únicamente a las muestras que aporten información de utilidad para llevar a cabo el desarrollo posterior. Teniendo en cuenta el objetivo definido, se va a estructurar esta Sección en diferentes fases, en función de cada uno de los pasos que se deberán realizar para procesar los datos.

\vspace{3mm}

Antes de comenzar a describir este procesamiento, es imprescindible destacar que este vendrá determinado principalmente por la creación y diseño de código programado en \textit{Python} en diferentes \textit{notebooks} de \textit{Jupyter}, ya que es el entorno más adecuado en el ámbito de la ciencia de datos y de la aplicación de técnicas de \gls{ml}. Su empleo aportará la ventaja de poder depurar y comprobar paso a paso que se ejecutan de forma correcta cada una de las acciones que componen el procesamiento de los datos. Además, cabe destacar el uso de algunas librerías que serán imprescindibles para el manejo de los datos, como son \textit{pandas}, \textit{csv} o \textit{numpy}, entre otras.

\subsection{Datos de consumo}

\subsubsection{Análisis de la situación inicial y primeros pasos}
\label{sec:inicialproc}

El dataset seleccionado para este \gls{tfm}, \textit{SustDataED}, se detallaba en la Sección \ref{sec:sustdataed} como un conjunto de datos que abría multitud de posibilidades de implementación. Se puede expresar que esto venía justificado por el hecho de abarcar un extenso lapso temporal de medidas correspondientes a un gran número de viviendas. Además, estas medidas se caracterizaban por ser adquiridas a una frecuencia de muestreo de un minuto, lo que aportaba una buena resolución para analizar el comportamiento eléctrico de todos los usuarios implicados. 

\vspace{3mm}

Como también se exponía en la Sección comentada, especialmente en la Figura \ref{fig:despliegues}, el dataset \textit{SustDataED} se estructuraba en cuatro despliegues diferentes, cada uno con diferentes rangos temporales de medición y con un conjunto de viviendas determinado. Poniendo el enfoque en las medidas de consumo, esto supone que se termine coleccionando en conjunto hasta un total de 24.512.181 muestras, correspondientes a 1144 días de medición. A modo de síntesis, se aporta la Tabla \ref{tab:resumen} con la información respectiva a cada despliegue. \cite{sustdata}

\vspace{3mm}

\begin{table}[h!]
    \centering
    \begin{tabular}{c|c|c|cc}
    \hline
    \rowcolor[HTML]{C0C0C0} 
    \multicolumn{1}{|l|}{\cellcolor[HTML]{C0C0C0}Despliegue} & \multicolumn{1}{l|}{\cellcolor[HTML]{C0C0C0}Muestras} & \multicolumn{1}{l|}{\cellcolor[HTML]{C0C0C0}Días} & \multicolumn{1}{l|}{\cellcolor[HTML]{C0C0C0}Fecha de inicio} & \multicolumn{1}{l|}{\cellcolor[HTML]{C0C0C0}Fecha de fin} \\ \hline
    \multicolumn{1}{|c|}{1} & 3.474.557 & 123 & \multicolumn{1}{c|}{10/07/2010} & \multicolumn{1}{c|}{10/11/2010} \\ \hline
    \multicolumn{1}{|c|}{2} & 12.481.536 & 504 & \multicolumn{1}{c|}{25/11/2010} & \multicolumn{1}{c|}{20/04/2012} \\ \hline
    \multicolumn{1}{|c|}{3} & 5.671.576 & 298 & \multicolumn{1}{c|}{01/08/2012} & \multicolumn{1}{c|}{25/05/2013} \\ \hline
    \multicolumn{1}{|c|}{4} & 2.884.512 & 219 & \multicolumn{1}{c|}{31/07/2013} & \multicolumn{1}{c|}{10/03/2014} \\ \hline
    \multicolumn{1}{l|}{} & \multicolumn{1}{l|}{\cellcolor[HTML]{EFEFEF}24.512.181} & \multicolumn{1}{l|}{\cellcolor[HTML]{EFEFEF}1144} & \multicolumn{1}{l}{} & \multicolumn{1}{l}{} \\ \cline{2-3}
    \end{tabular}
    \caption{Resumen de los datos de consumo del dataset \textit{SustDataED} \cite{sustdata}}
    \label{tab:resumen}
\end{table}

\vspace{3mm}

Tomando lo anterior en consideración, en este paso, enfocado al análisis de la situación inicial del conjunto, es preciso tener el conocimiento de los ficheros de datos de consumo de los que se va a partir. Para ello, se añade la Tabla \ref{tab:fichconsumo}, la cual expone la información relativa a cada despliegue y a los tamaños de cada uno de estos ficheros. 

\vspace{3mm}

\begin{table}[h!]
    \centering
    \begin{tabular}{|c|c|c|}
    \hline
    \rowcolor[HTML]{C0C0C0}
    \multicolumn{1}{|l|}{\cellcolor[HTML]{C0C0C0}Despliegue} & Fichero de muestras & \multicolumn{1}{l|}{\cellcolor[HTML]{C0C0C0}Tamaño (MB)} \\
    \hline
    1 & power\_samples\_d1\_1 & 86.38 \\
      & power\_samples\_d1\_2 & 69.99 \\
    \hline
    2 & power\_samples\_d2\_1 & 289.81 \\
      & power\_samples\_d2\_2 & 265.13 \\
      & power\_samples\_d2\_3 & 283.27 \\
      & power\_samples\_d2\_4 & 311.89 \\
    \hline
    3 & power\_samples\_d3\_1 & 259.92 \\
      & power\_samples\_d3\_2 & 233.69 \\
      & power\_samples\_d3\_3 & 272.23 \\
    \hline
    4 & power\_samples\_d4\_1 & 250.12 \\
      & power\_samples\_d4\_2 & 252.84 \\
    \hline
    \end{tabular}
    \caption{Resumen de las características de los ficheros de los despliegues de \textit{SustDataED}}
    \label{tab:fichconsumo}
\end{table}

\vspace{3mm}

Como se puede comprobar, a priori la mayoría no precisan de un volumen manejable, por lo que la primera acción a realizar será dividir los mismos en nuevos ficheros que supongan un tamaño menor del dado. Esto sobre todo será importante a la hora de trabajar con el repositorio de GitHub, puesto que existe el factor limitante de que el tamaño máximo permitido por archivo es de 100MB. 

\vspace{3mm}

No obstante, antes de dividir los ficheros, se debe volver a la Tabla que hace referencia a las medidas de consumo energético (ver Tabla \ref{tab:consumo}) para observar que estos ficheros contienen una columna dedicada a la marca de tiempo (\textit{timestamp}) del instante en el que se adquirió la medida. A modo de simplificar esta información temporal, se crea el \textit{notebook} \textit{preprocessing\_nodes.ipynb}, que va a importa la librería \textit{time} para añadir tres nuevas columnas al dataset que separen los datos de la fecha (columna \textit{datetime}), el instante exacto de la muestra (hora, minuto y segundo) (columna \textit{hour}) y la hora como un valor entero (columna \textit{H}). En el caso de esta última columna, su utilidad vendrá descrita por los requerimientos de la siguiente Sección (ver Sección \ref{sec:datasamples}).

\vspace{3mm}

\begin{lstlisting}[style=Python-color, caption={Formato del timestamp}]
  df['tmstp'] = pd.to_datetime(df['tmstp']) # Conversión de la columna del dataframe a formato datetime
  df['datetime'] = df['tmstp'].dt.strftime('%Y-%m-%d') # Formateo de la fecha (año, mes, día)
  df['hour'] = df['tmstp'].dt.strftime('%H:%M:%S') # Formateo del instante (hora, minuto, segundo)
  df['H'] = df['tmstp'].dt.strftime('%H').astype(int) # Formateo de la hora como entero
\end{lstlisting}

\vspace{3mm}

De forma adicional, en este \textit{notebook} se comprueba también, si existen filas con valores \textit{NaN} que puedan perjudicar al análisis de los datos. Como ventaja, el dataset \textit{SustDataED} aporta la columna \textit{miss\_flag} (ver Tabla \ref{tab:consumo}) para determinar si para cierto instante temporal se ha producido la adquisición de forma incorrecta o con valores vacíos. Por tanto, la acción a desarrollar será comprobar en todo el dataframe cuándo este valor se encuentra activo y, en ese caso, eliminar la fila respectiva.

\vspace{3mm}

\begin{lstlisting}[style=Python, caption={Eliminación de valores NaN}]
  df.isnull().any() # Comprobación previa de existencia de valores NaN en el dataframe
  df = df[df['miss_flag'] != 1] # Aplicación de filtro al dataframe
  df = df.dropna(thresh=len(df.columns) - 15 + 1) # Eliminación de columnas con todos los elementos vacíos
  df.isnull().any() # Comprobación posterior de existencia de valores NaN en el dataframe
\end{lstlisting}

\vspace{3mm}

A modo de resumen, se puede determinar entonces, que se seguirá la siguiente secuencia de pasos en esta primera fase: crear las nuevas columnas temporales, eliminar las filas que estén completamente vacías y dividir los ficheros iniciales para ajustar su tamaño a un máximo de 100MB. Es obligatorio seguir este orden, puesto que no sería eficiente extraer los fragmentos del conjunto de datos y, después, añadir información nueva. En otros términos, de esta forma se estaría superando el tamaño máximo ajustado en los nuevos ficheros de salida.

\vspace{3mm}

Volviendo al proceso de división de los ficheros, se crea otro \textit{notebook} de \textit{Jupyter}, denominado como \textit{split.ipynb}. En primer lugar, se importa al mismo la librería de \textit{Python} \textit{csv}, enfocada al tratamiento de archivos con formato .csv. Después, se escribe una función que, a partir de un fichero de entrada, itere a través de sus filas de datos hasta llegar al volumen máximo determinado para obtener a la salida el nuevo archivo en la ruta indicada. Además, para cada nueva fragmentación de los datos se incluye el encabezado con los nombres de cada uno de los campos.

\vspace{3mm}
\begin{lstlisting}[style=Python, caption={Fragmentación de ficheros iniciales}]
  with open(input_file, 'r') as csv: # Apertura del fichero de entrada
    reader = csv.reader(csv) # Definición del reader
    header = next(reader)  

  for row in reader: # Iteración de las filas del fichero
      if current_row % split_size == 0:  # Condición de máximo de filas
          if output_file: # Cierre del fichero de salida 
              output_file.close()
          split = f'dataset/split_{current_split}.csv'
          output_name = input_name + split
          output_file = open(output_name, 'w', newline='') # Apertura del fichero de salida
          writer = csv.writer(output_file) # Definición del writer
          writer.writerow(header) # Escritura del encabezado
          current_split += 1
      writer.writerow(row) # Escritura de fila
      current_row += 1

if output_file: # Cierre del fichero de salida
  output_file.close()
\end{lstlisting}

\vspace{3mm}

Por tanto, tras realizar este procedimiento, se logra obtener un total de 28 nuevos ficheros en formato .csv de un tamaño reducido, permitiéndose así, una mejor gestión de los datos que contienen los mismos.

\subsubsection{Síntesis y reducción de la información}
\label{sec:datasamples}

La presente Sección se dedicará principalmente a la reducción de la información dada por \textit{SustDataED}. Como se ha comentado anteriormente, el hecho de que las muestras se hayan adquirido a una frecuencia de un minuto, proporciona un gran volumen de datos que debería sintentizarse. En el caso de este \gls{tfm}, con el fin de reducir el número de filas del dataset, se ha valorado como opción más apropiada realizar una cuantificación del promedio de las mediciones para cada hora. En otros términos, a partir de las muestras iniciales obtenidas cada minuto, se generaría un nuevo conjunto de medidas promediadas cada hora, resultando en una reducción de los datos en un factor 60.

\vspace{3mm}

En primera instancia, para simplificar la programación, se crea un primer \textit{notebook} básico que realice el procedimiento de reducción de datos para uno de los ficheros fragmentados que se han obtenido a partir de \textit{split.ipynb}. Este nuevo \textit{notebook} se denomina como \textit{datasamples\_onenode.ipynb} y aplica el promedio únicamente a las medidas de un nodo contenidas en una hora y fecha determinadas. Estos valores son determinados como parámetros de entrada y, en el caso de la hora, es preciso hacer uso de la columna \textit{H}, definida en la Sección anterior (ver Sección \ref{sec:inicialproc}) para determinar la hora a la que pertenece la medida como un valor entero.

\vspace{3mm}

Por tanto, una vez definidos los valores de los parámetros de entrada, se aplica un filtro al dataframe a partir de los mismos. En este momento, se comprueba que se ha realizado el procedimiento de forma correcta mediante la observación del número de filas que se extraen, que debería ser 60. 

\vspace{3mm}

Sin embargo, para llevar a cabo el cálculo del promedio, es preciso especificar las columnas sobre las que se va a aplicar la operación, que son las que contienen todos los valores que hacen referencia a los parámetros eléctricos. Finalmente, a partir de las 60 filas filtradas y, tomando en consideración estas columnas, se calcula el valor promedio de cada uno de los parámetros eléctricos y se proporcionan estos a la salida en una única fila.

\vspace{3mm}

\subsubsection{Automatización del proceso de síntesis y reducción de la información}
\label{sec:datasamples}

Como se ha expuesto en la Sección anterior (ver Sección \ref{sec:datasamples}), el proceso de cálculo del promedio de valores viene especificado por los parámetros de entrada definidos y se ejecuta únicamente para las 60 muestras que se han adquirido a partir de uno de los nodos en una hora y fechas determinadas. 

\vspace{3mm}

Esta característica supone que el proceso debe de ser automatizado.





\subsection{Datos de producción}
%meter graficas del paper de sustdata
%meter graficas de sustdata (ej. consumo de un hogar y tal, correlacion prod-clima)
%definir que es lo que se va a utilizar

%ESTO ES LO QUE ESTA PUESTO ARRIBA --> (PARA ACORDARME DE REFERENCIARLO AQUI Y DETALLARLO)
% El procesamiento que será requerido para estos datos se detallará en la Sección \ref{sec:preprocesado}. No obstante, es importante destacar las características de los datos de producción eléctrica dados por \textit{SustDataED}. Como se ha expuesto anteriormente en la Sección \ref{prodsustdata}, el dataset proporciona esta información en términos globales y después, desagrega los valores energéticos según su fuente de procedencia. Como se expondrá en la Sección \ref{sec:preprocesado}, será imprescindible determinar si estos datos son precisos en un entorno de \gls{sg}s como se requiere para cumplir con los objetivos de este \gls{tfm}


% 7 y 7.1 del AI-based FDI Countermeasure for IoE Smart Grids

%hacerme una idea de la estructura -> ver pag 37 AI-based-FDI-Countermeasure-for-IoE-Smart-Grids











%ESTO PARA CUANDO SE HABLE DE BRITE

%En la Sección \ref{sec:brite_eje}, referente a la ejecución de la herramienta \gls{brite} se había expuesto la posibilidad de generar un total de 1200 topologías para probar sobre el algoritmo \gls{den2ne}. Ahora, teniendo en cuenta también el número de nodos disponibles (25) y la cantidad de instantes temporales comprendidos en los datos (24*365=8760), existe la posibilidad de realizar hasta 262.800.000 simulaciones únicas.

% %%%%%%%%%%%%%%%%%%%%%%%%%%%%%%%%%%%%%%%%%%%%%%%%%%%%%%%%%%%%%%%%%%%%%%%%%%%%%%%%%%%%%%%%%%%%%%%%%
%%%%%%%%%%%%%%%%%%%%%%%%%%%%%%%%%%%%%%%%%%%%%%%%%%%%%%%%%%%%%%%%%%
\section{Planteamiento de escenarios y generación de topologías en \acrshort{brite}}
\label{sec:ejebrite}

La presente Sección está dedicada al planteamiento de una serie de escenarios de red y a la consecuente generación de topologías mediante el empleo de la herramienta \gls{brite}, cuyo funcionamiento ha sido descrito en la Sección \ref{sec:brite}. La importancia de esta fase del diseño viene dada por la necesidad de aplicar la operativa del algoritmo \gls{den2ne} sobre múltiples topologías para obtener, finalmente, el dataset sobre el que se entrenarán y desarrollarán los modelos de \gls{ml} y \gls{dl}. Se debe tener en cuenta que, para que este conjunto de datos final permita aportar suficiente información útil a los modelos, el número de topologías aleatorias a generar en \gls{brite} debe de ser relativamente elevado.

\vspace{3mm}

\subsection{Planteamiento de escenarios y configuración de \acrshort{brite}}
\label{sec:conftopo}

Para emplear la herramienta, se deben plantear los escenarios de red sobre los que se desea basar la generación de topologías aleatorias. Por ello, es preciso definir la configuración de los parámetros de entrada en el script \textit{autogenerador.sh}, tomando en consideración el requerimiento anterior. Como se especificaba en la Sección \ref{sec:brite_eje}, este script está dedicado a la automatización de la ejecución, tanto de la herramienta, como del \textit{parser}, para obtener a la salida los ficheros finales con las posiciones de los nodos en el plano y con la información relativa a las distancias y a los nodos que se interconectan con cada enlace (\textit{Nodos.txt} y \textit{Enlaces.txt}). 

\vspace{3mm}

Entonces, se puede expresar que el cálculo del número total de topologías que se pueden generar con \gls{brite} resultará de operar el producto de los siguientes parámetros de entrada configurados:

\begin{itemize}
    \item Número de modelos de topología a emplear: Como se había introducido en la Sección \ref{sec:modelostopos}, este \gls{tfm} se va a basar en el empleo de topologías aleatorias a nivel de router y, en particular, en los modelos Router Waxman y Router Barabasi-Albert. 
    \item Número de dimensiones: Para las pruebas a simular en el algoritmo \gls{den2ne}, se pretende manejar topologías con una gran cantidad de nodos, pero reduciendo los saltos de incremento del total. Por ello, se configura una generación desde 100 a 200 nodos con un incremento de 50 en 50, suponiendo la configuración de 3 dimensiones de topologías (100, 150 y 200 nodos). 
    \item Número de grados de conectividad: Este parámetro determina el número de enlaces por nodo y se especifican 3 grados \textit{m} diferentes en el script. Es preciso indicar que, como en \gls{den2ne} se tratarán los enlaces como bidireccionales, realmente cada topología tendrá un grado \textit{2m}.
    \item Número de semillas de generación: Es importante destacar que para la generación de topologías aleatorias se aplican 10 ficheros de semillas. Esto tiene como resultado la generación de 10 topologías diferentes para cada uno de los escenarios configurados o, en otros términos, para cada una de las configuraciones posibles de los parámetros de entrada.
\end{itemize}

\vspace{3mm}

Adicionalmente, se deben determinar otros parámetros que hacen referencia a los nodos, como son el modo de introducción al plano y su posicionamiento. Respectivamente, se establece una introducción incremental y un posicionamiento totalmente aleatorio alrededor del plano. También, los parámetros específicos del modelo Router Waxman, $\alpha$ y $\beta$, que toman valores de 0,2 y 0,15, respectivamente.

\vspace{3mm}

\begin{lstlisting}[language=bash, style=Consola, caption={Configuración de los parámetros de entrada en el script de automatización de \acrshort{brite}}]
topologia_rt_waxman=1  # Empleo de modelo RTWaxman
topologia_rt_barabasi=2  # Empleo de modelo RTBarabasi
nodos=$(seq 100 50 200) # Número de nodos por topología (dimensiones)
m=(1 2 3) # El grado de conectividad real es 2, 4, 6
n_topologias_distintasxnodo=10 # Número de topologías aleatorias en función del número de semillas

NodePlacement=1 # Posicionamiento aleatorio de los nodos
GrowthType=1 # Modo de introducción incremental

#parametros especificos de Waxman
alpha=0.2
beta=0.15
\end{lstlisting}

\subsection{Resultados de la generación de topologías aleatorias}
\label{sec:gentopo}

Considerando los parámetros de entrada configurados anteriormente, se puede cuantificar el número total de topologías que resultará de la ejecución del script \textit{autogenerador.sh} a partir de la siguiente expresión:

    \[\textit{N\_topos} = \textit{Nmodelos} \times \textit{Ndimensiones}
    \times \textit{Ngrados} \times \textit{Nseeds\_gen}\]
    \[\textit{N\_topos} = 2 \times 3 \times 3 \times 10 = 180\]

\vspace{3mm}

Es preciso indicar que en el caso de requerirse un mayor número de topologías a simular en \gls{den2ne}, bastaría únicamente con modificar el rango establecido para el número de nodos o el valor del incremento de los mismos. No obstante, puesto que se pretende tener en cuenta para la ejecución del algoritmo el dataset resultante de la etapa de procesamiento (ver Sección \ref{sec:combinacion}), no se precisa aumentar el número de topologías aleatorias a priori. Estos motivos vendrán justificados detalladamente en la Sección \ref{sec:confden2ne}, dedicada a la configuración de los parámetros de entrada de \gls{den2ne}.

\vspace{3mm}

De forma adicional, para observar gráficamente los resultados que se obtienen a la salida se representan varios ejemplos de topologías generadas en la Figura \ref{fig:grafbrite}. Respectivamente, las gráficas obtenidas de \textit{matlab} hacen referencia al modelo Barabasi-Albert y al Waxman y presentan una configuración de 100 nodos y un grado de conectividad \textit{m}=2, que al suponer enlaces direccionales realmente toma el valor de 4. Es importante indicar que, para apreciar las diferencias existentes entre ambos modelos, se selecciona el mismo fichero de semilla (\textit{seed\_4}) para la representación.

\vspace{3mm}

\begin{figure}[h!]
    \centering
    \begin{minipage}{0.5\textwidth}
      \centering
      \includegraphics[width=\linewidth]{img/diseno/britebarabasi2.png}
      \label{fig:grafbrite1}
    \end{minipage}\hfill
    \begin{minipage}{0.5\textwidth}
      \centering
      \includegraphics[width=\linewidth]{img/diseno/britewaxman2.png}
      \label{fig:grafbrite2}
    \end{minipage}\hfill
    \caption{Representación de diferencias de los modelos Router Barabasi-Albert y Waxman a partir de los mismos parámetros de entrada}
    \label{fig:grafbrite}
\end{figure}

% %%%%%%%%%%%%%%%%%%%%%%%%%%%%%%%%%%%%%%%%%%%%%%%%%%%%%%%%%%%%%%%%%%%%%%%%%%%%%%%%%%%%%%%%%%%%%%%%%
%%%%%%%%%%%%%%%%%%%%%%%%%%%%%%%%%%%%%%%%%%%%%%%%%%%%%%%%%%%%%%%%%%
\section{Simulación de las topologías en \acrshort{den2ne}}

Esta Sección viene dada por el objetivo de obtener el conjunto de datos final sobre el que se basará el desarrollo en el Capítulo \ref{cha:desarrollo}. Para lograrlo, se requiere llevar a cabo un gran número de  simulaciones en \gls{den2ne} a partir de las topologías generadas con la herramienta \gls{brite} (ver Sección \ref{sec:ejebrite}). 

\vspace{3mm}

No obstante, previamente a la ejecución del algoritmo, es imprescindible implementar una serie de modificaciones en el mismo para ajustar su funcionamiento para obtener resultados de utilidad para el presente \gls{tfm}. La secuencia de acciones que se llevará a cabo se detallará en la Sección \ref{sec:cambiosden2ne}. De la misma forma, se justificará el criterio escogido para determinar cúando se producen errores en el proceso de distribución energética. Esto servirá de base para después entrenar los modelos de \gls{ml} y \gls{dl}.

\vspace{3mm}

Por consiguiente, se añade la Sección \ref{sec:confden2ne}, donde se especificará la configuración de los parámetros de entrada de \gls{den2ne} y se cuantificará el número total de pruebas posibles que se podrían llegar a realizar a partir de las topologías generadas y de los datos reales procesados anteriormente. Se comprobará a través de diferentes pruebas que las modificaciones introducidas en el algoritmo \gls{den2ne} producen una operativa acorde a las necesidades de este \gls{tfm}. De forma concluyente, a partir de los resultados obtenidos en las simulaciones, se añadirá la Sección \ref{sec:datasetfinal}, en referencia a las características del dataset final que se utilizará para el entrenamiento de los modelos.

\subsection{Adaptación del algoritmo \acrshort{den2ne} a las pruebas}
\label{sec:cambiosden2ne}

Todas las modificaciones realizadas en el funcionamiento del algoritmo \acrshort{den2ne} han sido aplicadas a partir de los ficheros contenidos en el repositorio\footnote{https://github.com/NETSERV-UAH/den2ne-Alg} del equipo de investigación NetIS de la \gls{uah}.

\subsubsection{Importación de los perfiles de carga reales}

La primera modificación que se debe implementar en el funcionamiento de \gls{den2ne} consiste en la importación de los valores de carga reales, obtenidos como resultado de llevar a cabo la fase de procesamiento de los datos (ver Sección \ref{sec:combinacion}). La necesidad de aplicar este paso se debe a que el algoritmo, a partir de una topología dada a la entrada, establece una función de densidad de probabilidad para determinar de forma aleatoria un valor de carga para cada uno de los nodos de una topología. Para modificar este proceso se sigue la siguiente secuencia de pasos:

\begin{enumerate}
    \item Definición de función \textit{getLoads\_Config} en \textit{dataCollector.py}: Se añade una función de recolección de las configuraciones de carga reales, pasando como argumentos el directorio definido en \textit{path\_simtests} y el fichero de pruebas \textit{sim\_file}. Después, se inicializa una variable de diccionario y se almacenan los valores de potencia generada, consumida y la diferencia calculada de ambas. Estos valores se manejan en términos de kW, puesto que \gls{den2ne} está configurado para trabajar en esta unidad.
    \item Definición de funciones \textit{cargas} y \textit{cargas\_con\_limite} en \textit{brite\_intf.py}: Ambas funciones se encargan de aplicar los valores de carga reales, extraídos de la función anterior, a cada uno de los nodos de una topología determinada. En el caso de las simulaciones que se realizan en este \gls{tfm}, se hará uso de la primera, debido a que no se configuran límites de los valores de carga (ver Sección \ref{sec:confden2ne}). Por tanto, poniendo el foco en la función \textit{cargas}, se pasa como argumento el diccionario resultante de la función \textit{getLoads\_Config} y se extrae a partir de las claves del mismo el número de perfiles de carga (23) a manejar. De la misma forma, se pasa también como argumento el fichero de nodos, anteriormente generado con \gls{brite} y el \textit{parser}, para conocer las dimensiones que tiene la topología en cuestión. Con este dato se determina la cantidad de iteraciones que son necesarias para aplicar el valor de carga real a todos los nodos de una topología y, para cada uno de ellos, se determina de forma aleatoria uno de los 23 perfiles reales de carga. A la salida, se obtiene un nuevo diccionario con tamaño igual al número de nodos que hay en la topología y que viene constituido por la información del identificador del perfil real seleccionado y del valor de carga neta aplicada.
    \item Definición de la variable \textit{id\_orig} en \textit{node.py}: Se añade al constructor de la clase de nodo que tiene \gls{den2ne} un nuevo elemento, en relación al identificador del perfil de carga seleccionado para un nodo en cuestión.
    \item Introducción de la variable \textit{id\_orig} en la función \textit{buildGraph} en \textit{graph.py}: La nueva variable de nodo se incluye en el proceso de generación del grafo para mantener el conocimiento del perfil de carga que se ha seleccionado para cada nodo durante la ejecución completa del algoritmo.
    \item Modificaciones en \textit{test\_topo.py}: El fichero dedicado a las pruebas de \gls{den2ne} debe incluir las llamadas a las funciones creadas o modificadas anteriormente. También, requiere definir los nuevos parámetros \textit{path\_simtests} y \textit{sim\_file}, en referencia a los ficheros de prueba que contienen los valores de carga, extraídos del dataset para un instante temporal determinado (ver Sección \ref{sec:datasetfinal}).
\end{enumerate} 

\subsubsection{Introducción de la etiqueta de error}
Como se había introducido en el Capítulo \ref{ch:intro}, el objetivo de este \gls{tfm} viene dado por la necesidad de poder detectar y predecir los errores que se pueden producir durante el proceso de distribución energética que realiza \gls{den2ne}. Por ello, una vez que se aplican los pasos anteriores, se debe definir el criterio de error a partir del cual se van a etiquetar los resultados extraídos del algoritmo. 

\vspace{3mm}

Teniendo en cuenta que para las simulaciones se configura un escenario real con pérdidas y limitación de capacidad de los enlaces (ver Sección \ref{sec:confden2ne}), se toma la decisión de establecer la condición de fallo en base a la existencia de un exceso de capacidad en un intercambio energético entre dos nodos:

\begin{enumerate}
    \item Introducción de la variable \textit{link\_overflow} en la función \textit{globalBalance} en \textit{den2neALG.py}: Para cada intercambio energético que se realiza en el proceso de distribución, se comprueba que la carga direccionada desde el nodo de origen al nodo destino no sobrepasa la capacidad del enlace que los interconecta. En caso afirmativo, se activa la variable \textit{link\_overflow} y, por el contrario, se deja con valor nulo, indicando que el intercambio se ha producido sin fallos. 
    \item Modificación de la función \textit{getLosses} en \textit{link.py}: \gls{den2ne}, en su funcionamiento original, calcula las pérdidas del enlace a partir del valor de carga intercambiado o de la propia capacidad, en función de sobrepasarla o no. A modo de depuración y de simplificar su funcionamiento, se incluye como argumento la etiqueta \textit{link\_overflow} y se modifica la estructura de la función.
\end{enumerate} 

\subsubsection{Extracción de resultados}
Para extraer de la ejecución del algoritmo unos resultados que contengan toda la información útil sobre la que se pueda crear el dataset final, es preciso aplicar las siguientes modificaciones:

\begin{enumerate}
    \item Definición de la función \textit{getLinkDist} en \textit{graph.py}: Se incluye una función para obtener la distancia existente entre el nodo de origen y de destino.    
    \item Introducción de la variable \textit{data\_topo} en la función \textit{globalBalance} en \textit{den2neALG.py}: Se define el diccionario \textit{data\_topo} y, para cada intercambio energético, se almacena en el mismo la información del nodo origen y destino (etiquetas jerárquicas y longitudes de las mismas), los datos del enlace (distancia y capacidad), el valor de la etiqueta de error y la carga que se intercambia. En el caso de las longitudes de las etiquetas jerárquicas, se añaden las líneas de código necesarias en \textit{globalBalance} para extraer las mismas.
    Modificaciones en \textit{test\_topo.py}: Se cambia la llamada a la función \textit{globalBalance} para poder extraer el diccionario \textit{data\_topo} de la misma. También, se establece la nomenclatura que tendrán los ficheros de resultados. Como se expondrá más adelante en la Sección \ref{sec:datasetfinal}, esta nomenclatura se determina en base a la información dada por el instante temporal del fichero de test y por el resto de parámetros configurados en el script de automatización de pruebas \textit{auto\_run.sh}.
    \item Modificaciones en \textit{auto\_run.sh}: Se añaden los parámetros que hacen referencia al directorio donde se ubican los ficheros de test y a cada uno de los mismos para automatizar las pruebas (ver Sección \ref{sec:datasetfinal}).
\end{enumerate} 

\subsubsection{Configuración de la capacidad de los enlaces}
Como paso adicional, se modifican las capacidades de los enlaces que se proporcionan por defecto en el fichero \textit{links\_config.csv}. Esto se realiza con el objetivo de que el algoritmo configure enlaces de menor capacidad con la misma probabilidad que los de mayor, puesto que este proceso es aleatorio. A modo de simplificación, se establecen los tres tipos de enlaces definidos en la Tabla \ref{tab:links}

\begin{table}[h!]
    \centering
    \begin{tabular}{|c|c|c|}
    \hline
    \rowcolor[HTML]{AAAAAA}
    \multicolumn{1}{|c|}{\cellcolor[HTML]{AAAAAA}\textit{R (ohm/km)}} & \multicolumn{1}{c|}{\cellcolor[HTML]{AAAAAA}\textit{I max (A)}} & \textit{Sección (mm²)} \\ \hline
    0,272 & 185 & 70 \\ \hline
    0,78 & 100 & 25 \\ \hline
    1,91 & 53 & 10 \\ \hline
    \end{tabular}
    \caption{Configuraciones de enlaces de \textit{links\_config.csv}}
    \label{tab:links}
\end{table}

\subsection{Configuración de los parámetros de entrada}
\label{sec:confden2ne}

Para ejecutar \gls{den2ne}, es preciso definir la configuración de los parámetros de entrada en el script \textit{auto\_run.sh}. Este script se diseña con el objetivo de automatizar las simulaciones en el algoritmo y, como se ha expuesto en la Sección \ref{sec:cambiosden2ne}, ha sido necesario ajustar el mismo a los requerimientos de las pruebas de este \gls{tfm}. El cálculo de la cantidad total de simulaciones que se pueden ejecutar en \gls{den2ne} viene dado por el producto de todas los parámetros siguientes:

\begin{itemize}
    \item El número total de topologías generadas: En la Sección \ref{sec:gentopo} se especifica una cantidad total de 180 topologías obtenidas a la salida de la herramienta \gls{brite}.
    \item El número de instantes temporales del dataset: El conjunto de datos resultante de la etapa de procesamiento, detallado en la Sección \ref{sec:combinacion} y, específicamente, en la Tabla \ref{tab:datacombinacion}, abarca una cantidad de filas igual al total de instantes temporales que se han tomado en consideración. En otros términos, al tratarse de muestras tomadas cada hora durante un rango temporal que comprende un año completo, su valor se calcula como 24x365=8760 instantes.
    \item El número de criterios: Como se detallaba en la Sección \ref{sec:den2ne}, \gls{den2ne} presenta 6 criterios de selección de los mejores caminos hacia el nodo raíz. En este caso, se implementan a las pruebas los 6 tipos.
    \item El número de tipos de escenarios de red: En la Sección \ref{sec:den2ne}, también se exponían los 4 tipos de escenarios que permite configurar el algoritmo. Teniendo en cuenta que las simulaciones deben acercarse a un entorno de \gls{sg} real y que el objetivo se basa en encontrar las transacciones energéticas entre nodos que superen la capacidad del enlace, se determina únicamente el escenario de pérdidas y límite de capacidad.
    \item Modo de limitación de carga: Se especifica únicamente el modo que determina valores de carga ilimitados.
    \item El número de semillas de ejecución: De la misma manera que se ha especificado anteriormente para \gls{brite} (ver Sección \ref{sec:conftopo}), se aplica al algoritmo \gls{den2ne} un conjunto de archivos de semillas para obtener simulaciones diferentes a partir de una misma topología a la entrada. En este caso, debido a la cantidad de datos que ya se maneja y para no introducir latencias considerables en el lanzamiento de las pruebas, se especifican 5 semillas por topología.
\end{itemize}

Por otro lado, en el caso de los parámetros de entrada referentes a las topologías, como son el número de nodos por topología, los modelos a simular y el número de semillas de generación, se sigue la misma configuración especificada en la ejecución de \gls{brite} (ver Sección \ref{sec:conftopo}). No obstante, para el grado de conectividad se determinan directamente los valores reales, suponiendo enlaces bidireccionales.

\vspace{3mm}

\begin{lstlisting}[language=bash, style=Consola, caption={Configuración de los parámetros de entrada en el script de automatización de \acrshort{den2ne}}]
TOPO_CRITERIONS=(0 1 2 3 4 5) # Criterios de selección de IDs
TOPO_BEHAVIORAL=3 # Tipo de escenario de red = modo Losses and Capacity (3)         
TOPO_LOAD_LIMIT=0 # Sin limite de carga
TOPO_RUNS=5 # Seeds de ejecución de DEN2NE

TOPO_NAMES=('barabasi' 'waxman') # Empleo de modelos RTWaxman y RTBarabasi (igual que BRITE)
TOPO_NUM_NODES=$(seq 100 50 200) # Número de nodos por topología (igual que BRITE)
TOPO_DEGREES=(2 4 6) # El grado de conectividad real (en BRITE 1, 2, 3)
TOPO_SEEDS=(1 2 3 4 5 6 7 8 9 10) # Seeds de generación de BRITE
\end{lstlisting}

\vspace{3mm}

Por tanto, teniendo en cuenta la configuración implementada para los parámetros de entrada, se puede determinar el número total de simulaciones únicas posibles a partir de la siguiente expresión:

    \[\textit{Nsim} = \textit{Ntopos} \times \textit{Ninstantes} \times \textit{Ncriterios} 
    \times \textit{Nescenarios} \times \textit{NNlimit} \times \textit{Nsem\_ejec}\]
    \[\textit{Nsim} = 180 \times 8760 \times 6 \times 1 \times 1 \times 5 = 47304000\] 


\subsection{Creación del dataset final y conclusiones de las pruebas}
\label{sec:datasetfinal}

Teniendo en cuenta el número total de simulaciones únicas posibles calculado en la Sección \ref{sec:confden2ne}, se debe decidir qué pruebas se van a realizar en el algoritmo, puesto que es ineficiente e inviable ejecutar todas las posibilidades. Por ello, se escogen 12 instantes temporales, haciendo referencia a una hora determinada de un día al mes. Por simplificar, como el rango temporal de las muestras (ver Sección \ref{sec:rango}) comienza el 28 de noviembre de 2010, se seleccionan todos los días 28 de cada mes. De la misma forma, para la hora se escogen las 11:00. Esta decisión es debido a que a esta hora es cuando se aprecia una mayor producción energética de media, como se exponía en la Figura \ref{fig:average}. Por tanto, en los resultados de la ejecución de \gls{den2ne} se obtendrá un mayor número de intercambios energéticos con fallos, debido al exceso de capacidad del enlace. Esto es necesario para conseguir un dataset sobre el que se pueda entrenar de forma más precisa los modelos.

\vspace{3mm}

Con el objetivo de extraer de los ficheros finales de cargas netas \textit{load\_x.csv} (ver Sección \ref{sec:combinacion}) las filas de datos de los instantes temporales especificados, se introduce un nuevo \textit{notebook}, denominado como \textit{loadsamples.ipynb}.


%%%%%%%%
\vspace{3mm}

\begin{figure}[H]
    \centering
    \includegraphics[width=1\textwidth]{img/diseno/total2.png}
    \caption{Diagrama completo de diseño del dataset final}
    \label{fig:total2}
\end{figure}



%formato nomenclatura, comentar division de ficheros y directorios

%aquí va loadsamples.ipynb
  



Finalmente, se puede expresar que cada experimento o prueba realizada en \gls{den2ne}, derivará en la creación de un nuevo directorio con la fecha y hora seleccionadas (ej. \textit{dataset\_exp\_2011-04-28\_11}).



Como se podía apreciar en la Figura \ref{fig:dirpruebas}, cada experimento creaba un directorio con la fecha y hora determinadas (ej. \textit{dataset\_exp\_2011-04-28\_11}) y, en su interior, se generaban los diferentes ficheros de resultados, organizados por carpetas en función del modelo y el grado de conectividad al que hacían referencia (ej. \textit{barabasi-200-6}). 


\begin{figure}[H]
    \centering
    \includegraphics[width=0.5\textwidth]{img/diseno/dirpruebas.png}
    \caption{Nomenclatura de los directorios y ficheros de resultados}
    \label{fig:dirpruebas}
\end{figure}



%agrupación de ficheros en datafinal.ipynb
Teniendo en cuenta lo anterior, se decide crear un nuevo \textit{notebook}, \textit{datafinal.ipynb}, con el fin de conseguir a la salida un conjunto de datos final por cada instante temporal especificado en las pruebas. Como se podía apreciar en la Figura \ref{fig:dirpruebas}, cada experimento creaba un directorio con la fecha y hora determinadas (ej. \textit{dataset\_exp\_2011-04-28\_11}) y, en su interior, se generaban los diferentes ficheros de resultados, organizados por carpetas en función del modelo y el grado de conectividad al que hacían referencia (ej. \textit{barabasi-200-6}). 

\vspace{3mm}

Por ello, a modo de simplificar el tratamiento de los resultados del algoritmo, el primer paso que se implementa en el \textit{notebook} se basa en el agrupamiento en un mismo \textit{dataframe} de todos los ficheros que se han generado para un instante determinado. Después, es importante comprobar que el porcentaje total de error u \textit{overflow} dado para el instante en cuestión no es nulo. Es decir, para cada instante temporal que se pruebe, se debe conseguir un conjunto de datos que contenga algunas filas el valor \textit{overflow} activo para después, entrenar de forma precisa los modelos de \gls{ml} y poder predecir cuándo se producirán estos errores.

\vspace{3mm}

%%%%%REVISAR!!!!!!!!!!!!!!!!!!!!!!!!!!!
El siguiente paso consiste en combinar los datos obtenidos de los resultados con la información de las viviendas aportada en el fichero de test. Es importante recordar que este fichero de test (ej. \textit{test\_2010-11-28\_11.csv}) se creaba a partir de filtrar en el dataset resultante del procesamiento las medidas de las viviendas para un instante temporal determinado y se aplicaba a la entrada de \gls{den2ne} para definir los valores de carga (ver Sección \ref{sec:simtest}). Por ello, en este punto es combinar cada fila del \textit{dataframe} de resultados con la información real de cada vivienda en función del identificador del nodo al que se hace referencia. 

\vspace{3mm}

\begin{lstlisting}[style=Python, caption={Combinación de resultados y test}]
df_merged = pd.merge(df, df_sim, left_on='origen_id', right_on='iid', how='outer') # Operación de merge
\end{lstlisting}

\vspace{3mm}

Tras esta operación, el nuevo \textit{dataframe} se constituye por todas las columnas necesarias, aunque antes estas se deben revisar y eliminar si vienen replicadas, como ocurre en el caso del identificador y la fecha. 
Finalmente, se almacena el \textit{dataframe} limpio en un fichero con formato .csv, tratándose del dataset final que se empleará en la etapa de desarrollo. Este fichero contiene los campos definidos en la Tabla \ref{tab:datafinal} y sigue una nomenclatura formada por la marca de tiempo a la que hace referencia (ej. \textit{2010-12-28\_11.csv}). %%tabla para cada fecha

\vspace{3mm}

\begin{table}[h!]
    \centering
    \begin{tabular}{|c|c|c|}
    \hline
    \rowcolor[HTML]{AAAAAA} 
    \multicolumn{1}{|c|}{\cellcolor[HTML]{AAAAAA}Campo} & \multicolumn{1}{c|}{\cellcolor[HTML]{AAAAAA}Descripción} & Unidades \\ \hline
    \textit{timestamp} & Instante temporal de medida & datetime \\ \hline
    \textit{datetime} & Fecha del valor promedio & datetime \\ \hline
    \textit{H} & Hora del valor promedio & - \\ \hline
    \textit{overflow} & Etiqueta binaria de superación de carga & - \\ \hline
    \textit{cap} & Capacidad del enlace & kW \\ \hline
    \textit{load} & Carga neta (\textit{Dif}) & kW \\ \hline
    \textit{dist} & Distancia & m \\ \hline
    \textit{origen\_id} & Identificador de nodo origen & - \\ \hline
    \textit{dest\_id} & Identificador de nodo destino & - \\ \hline
    \textit{len\_origen\_tag} & Longitud de la etiqueta del nodo origen & - \\ \hline
    \textit{len\_dest\_tag} & Longitud de la etiqueta del nodo destino & - \\ \hline
    \textit{modelo} & Modelo de topología & - \\ \hline
    \textit{criterion} & Criterio de selección de IDs & - \\ \hline
    \textit{degree} & Grado de conectividad & - \\ \hline
    \textit{total\_balance} & Balance de carga global & - \\ \hline
    \textit{abs\_flux} & Flujo total de carga en el nodo raíz & - \\ \hline
    \textit{Diffuse Irradiance} & Índice de radiación difusa (\gls{dif}) & W/m2 \\ \hline
    \textit{Plane of Array Irradiance} & Índice de radiación en el plano del array (\acrshort{poa}) & W/m2 \\ \hline 
    \textit{Ambient Temperature} & Temperatura ambiente & C \\ \hline
    \textit{Cell Temperature} & Temperatura de las células solares & C \\ \hline
    \textit{DC Array Output} & Potencia de salida DC del array & W \\ \hline
    \textit{AC System Output} & Potencia de salida AC del sistema & W \\ \hline
    \textit{Pavg} & Potencia consumida & W \\ \hline
    \textit{Dif} & Carga neta calculada & W \\ \hline
    \end{tabular}  
    \caption{Dataset final obtenido por cada instante temporal probado en \acrshort{den2ne}}
    \label{tab:datafinal}
\end{table}

Con la obtención de este conjunto de datos se finaliza la etapa de diseño y, por ende, el presente Capítulo. A partir de los resultados obtenidos, se puede expresar de forma concluyente que se cumple el objetivo principal de generar el dataset final que suponga la base del entrenamiento y el desarrollo de los modelos de \gls{ml} que se plantearán en el Capítulo \ref{cha:desarrollo}. 

%Adicionalmente, a modo de simplificar la compresión de la secuencia de acciones que se han llevado a cabo durante este Capítulo, se representa la Figura \ref{fig:total}. En el diagrama se exponen los ficheros creados en cada paso, el tratamiento de los datos y los procesos ejecutados.
  
\pagebreak

% \begin{sidewaysfigure}
%     \centering
%     \includegraphics[width=1\textwidth]{img/diseno/total.png} 
%     \caption{Diagrama completo de diseño del dataset final}
%     \label{fig:total}
% \end{sidewaysfigure}

