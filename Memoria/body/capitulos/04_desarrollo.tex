\chapter{Desarrollo y evaluación del modelo}
\label{cha:desarrollo}


En este capítulo, se abordará el desarrollo de los diferentes modelos de predicción de errores y el consecuente entrenamiento de los mismos. Para ello, se tomará como base el conjunto de datos resultante del Capítulo \ref{cha:analisis} y, específicamente, de la Sección \ref{sec:datasetfinal} (ver Tabla \ref{tab:datafinal}).

\vspace{3mm}

La organización del presente capítulo se compone de dos partes diferenciadas en función de la naturaleza del aprendizaje que caracteriza a los modelos. Por un lado, se expondrán las técnicas de \gls{ml} empleadas para resolver la clasificación y predicción de errores (ver Sección \ref{sec:tecnicasml}). En el caso de este \gls{tfm}, se pone el foco en las técnicas \gls{rf} y \gls{svm}, cuyo funciomaniento fue descrito en las Secciones \ref{sec:mlrf} y \ref{sec:mlsvm}. Por otro lado, se realizará de la misma forma para las técnicas de \gls{dl} (ver Sección \ref{sec:tecnicasdl}). En particular, se detallará la configuración y la implementación de varias \gls{ann}s.

\vspace{3mm}

Ambas Secciones comentadas se dividirán internamente, en base a la secuencia de pasos que se requieren para llevar a cabo el desarrollo completo de cada una de las técnicas. Finalmente, se obtendrán una serie de resultados, que serán analizados y contrastados. En función de su precisión, se podrá determinar de forma concluyente qué modelo aporta una mayor efectividad en el diagnóstico y predicción de errores en una \gls{sg}.

%%%%%%%%%%%%%%%%%%%%%%%%%%%%%%%%%%%%%%%%%%%%%%%%%%%%%%%%%%%%%%
\section{Técnicas de \gls{ml}}
\label{sec:tecnicasml}

Teniendo en cuenta que los objetivos del presente \gls{tfm} se basan en la resolución de un problema de clasificación de errores, se toma la decisión de emplear \gls{rf} y \gls{svm}, como técnicas de aprendizaje supervisado. Para desarrollar diversos modelos en base a ambas, se requiere establecer la  secuencia de acciones a seguir: 

\begin{enumerate}
    \item Diseño y ejecución del modelo de prueba: Se desarrolla una primera versión por defecto de la técnica de clasificación para probar su funcionamiento. En este modelo de prueba se parte del conjunto de datos completo y de todas sus características, a excepción de las que hacen referencia a los valores de potencia, como se detallará más adelante (ver Sección \ref{sec:rf1}). 
    \item Puntuación de características: A partir de la ejecución del modelo de prueba, se puede apreciar qué características presentan mayor importancia en el proceso de clasificación, en base a la aplicación de varios métodos. 
    \item Optimización de hiperparámetros: Mediante la aplicación del método \textit{Grid Search}, se busca la mejor combinación de hiperparámetros de cada técnica para que los modelos resultantes proporcionen un rendimiento óptimo. 
    \item Selección de características: Una vez conocidos los hiperparámetros óptimos, se utilizan varias metodologías de selección de características para diseñar los nuevos modelos a partir de un número reducido de las mismas.
    \item Ejecución y validación de los modelos: Se extraen y se analizan los resultados de ejecución de todos los modelos anteriores. Se validan en base a la precisión obtenida mediante la creación de la matriz de confusión y de la aplicación de la técnica \textit{K-Fold Cross Validation}.
\end{enumerate}

Los pasos anteriores serán detallados posteriormente en las Secciones \ref{sec:rf} y \ref{sec:svm}. Es preciso indicar que se implementan dos nuevos \textit{notebooks}, \textit{rf.ipynb} y \textit{svm.ipynb}, para desarrollar los modelos de \gls{rf} y \gls{svm}, respectivamente. Principalmente, el diseño de los mismos tendrá como base el empleo de diferentes métodos proporcionados por la biblioteca de software de código abierto \textit{scikit-learn} o \textit{sklearn}, además de otras librerías imprescindibles para el manejo de los datos, como son \textit{pandas}, \textit{matplotlib} o \textit{seaborn}.


%%%%%%%%%%%%%%%%%%%%%%%%%%%%%%%%%%%%%%%%%%%%%%%%%%%%%%%%%%%%%%%%%%
\subsection{Random Forest (\acrshort{rf})}
\label{sec:rf}

\subsubsection{Puntuación de características}
\label{sec:rf1}

Se desarrolla una primera versión del clasificador basado en la técnica de \gls{rf}, \textit{RandomForestClassifier()} \cite{rfsklearn}, para visualizar las características que más repercuten en el proceso de clasificación de errores. En este modelo por defecto, se construyen un total de 10 estimadores o árboles y se determina el criterio de medición de la impureza a partir de la entropía (ver Sección \ref{sec:mlrf}). Después, se aplican dos métodos diferentes para puntuar la importancia que toman las características en el clasificador anterior. 

\vspace{3mm}

\begin{lstlisting}[style=Python, caption={Clasificador RF por defecto}]
  classifier = RandomForestClassifier(n_estimators = 10, criterion = 'entropy', random_state = 0) 
  classifier.fit(X_train, y_train)
\end{lstlisting}
  
\vspace{3mm}

Por un lado, se estima la importancia de las características del conjunto, a partir del atributo \textit{feature\_importances\_}, proporcionado por el clasificador. En este caso, el cálculo se realiza a partir de la media y de la desviación estándar de la disminución de la impureza que se produce dentro de cada árbol. Como se representa en la Figura \ref{fig:imp1}, se devuelve un array con los valores de importancia relativa asignados a las características y cuyo sumatorio es igual a 1. En este caso, se visualiza una gran incidencia de la distancia, seguida de los parámetros resultantes de \gls{den2ne}, \textit{total\_balance} y \textit{abs\_flux}, que hacen referencia a la carga que presenta el nodo \textit{gateway} tras el balance y al flujo total de recursos intercambiados en el proceso de distribución energética.

\pagebreak

Sin embargo, con este método surge cierto sesgo hacia las características que presentan alta cardinalidad, o en otros términos, una gran cantidad de valores únicos. Esto se debe a que generan nodos de división con mayor profundidad en los árboles, puesto que existen más opciones de separación del conjunto de datos. Por lo tanto, este tipo de características pueden recibir una puntuación más inflada.

\vspace{3mm}

\begin{figure}[H]
  \centering
  \includegraphics[width=1\textwidth]{img/desarrollo/rf/importance1.png}
  \caption{Puntuación de características del \acrshort{rf} mediante el atributo \textit{feature\_importances\_}}
  \label{fig:imp1}
\end{figure}

\vspace{3mm}

Teniendo esto en cuenta, se decide cuantificar la importancia también por el método de permutación (\textit{permutation\_importance}) \cite{importance}. Como se puede ver en la Figura \ref{fig:imp2}, en este caso, los parámetros resultantes de \gls{den2ne}, \textit{total\_balance} y \textit{abs\_flux}, siguen presentando puntuaciones altas, pero ahora también, las longitudes de las etiquetas de los nodos y la capacidad del enlace.

\vspace{4mm}

En este segundo método, los valores de las características se permutan una a una aleatoriamente y se evalúa en cada iteración los resultados del clasificador. Por tanto, cuando la variación de valores de una característica decrementa de forma considerable la precisión del modelo, puntúa una mayor importancia. Esta técnica es útil porque proporciona una evaluación imparcial de la importancia de las características. 

\newpage

\begin{figure}[H]
  \centering
  \includegraphics[width=1\textwidth]{img/desarrollo/rf/importance2.png}
  \caption{Puntuación de características del \acrshort{rf} mediante el método \textit{permutation\_importance}}
  \label{fig:imp2}
\end{figure}

\subsubsection{Optimización de hiperparámetros}
\label{rf2}

Los hiperparámetros de un modelo se definen como los parámetros de configuración o argumentos que son incluidos en el constructor de la clase del estimador o clasificador. En el caso del \gls{rf}, como ya se había introducido en el diseño del modelo por defecto, se pone el foco en dos hiperparámetros principales: el número de estimadores o árboles y el criterio de medición de la impureza. Para encontrar los valores que construyen el modelo óptimo de \gls{rf}, es preciso emplear la técnica \textit{Grid Search} \cite{gridsearch} y, específicamente, la clase \textit{GridSearchCV()} \cite{gridsearch2}. Esta realiza una búsqueda exhaustiva a partir de la definición de una cuadrícula de hiperparámetros para el estimador y extrae la combinación de valores que aporta una mayor precisión. Por ello, es necesario primero, definir en un diccionario una serie de valores para los dos hiperparámetros que se pretenden optimizar en el nuevo clasificador.

\begin{lstlisting}[style=Python, caption={Cuadrícula de parámetros RF}]
  param_grid = {
    'n_estimators': [10, 25, 50, 75, 100],
    'criterion': ['entropy', 'gini']
  }
\end{lstlisting}

\vspace{3mm}

Por consiguiente, se crea el objeto de la clase \textit{GridSearchCV()} y se determina el esquema de validación cruzada (\textit{cross validation}) de este mediante el parámetro \textit{cv}. En este caso, toma un valor de 5 y define el número de pliegues (\textit{folds}) que se van a utilizar para dividir el conjunto de datos y evaluar las combinaciones de los hiperparámetros. De la misma forma, se especifican el resto de parámetros, como la métrica para evaluar el rendimiento del modelo en cada iteración, que viene dada por la precisión global obtenida (\textit{accuracy}). 

\vspace{3mm}

Una vez creado el objeto, se ajusta al conjunto de datos de entrenamiento. Este proceso, a partir de los atributos \textit{best\_score\_} y \textit{best\_params\_}, aporta a la salida cuál es la mejor combinación de parámetros de todas las probadas y qué valor de precisión alcanza la misma. En este caso, se obtiene una precisión del 99,18\% con un clasificador basado en 100 estimadores y en el criterio de entropía. 

\vspace{3mm}

\begin{lstlisting}[style=Python, caption={Construcción del objeto \textit{GridSearchCV()}}]
  grid_search = GridSearchCV(estimator = classifier,
                            param_grid = param_grid,
                            scoring = 'accuracy',
                            cv = 5,
                            n_jobs = -1)

  grid_search.fit(X_train, y_train)
\end{lstlisting}

\vspace{3mm}

Sin embargo, para llevar a cabo un análisis en mayor profundidad de los resultados, se hace uso del atributo \textit{cv\_results\_}, que proporciona un diccionario con toda la información útil de la búsqueda. Principalmente, este análisis se centra en los valores de precisión obtenidos de todos los modelos y en los tiempos promedios que han sido necesarios para ajustar los mismos al conjunto de entrenamiento. Como se puede apreciar en la Tabla \ref{tab:rfgs}, se presentan variaciones muy pequeñas en los valores de precisión obtenida (\textit{mean\_test\_score}) para las diferentes combinaciones de parámetros.

\vspace{3mm}

No obstante, en el caso de los tiempos (\textit{mean\_fit\_time}), como es coherente, sí que se observan grandes diferencias. En la Tabla \ref{tab:rfgs2} se visualiza cómo se incrementa la duración de la búsqueda proporcionalmente al número de estimadores que se emplea. Esta métrica es importante también tenerla en cuenta para determinar el modelo óptimo, ya que a partir de la aplicación de 25 estimadores, la precisión no mejora de forma considerable. Por lo tanto, tras analizar los resultados, se puede expresar de forma concluyente que la mejor opción de modelo a emplear viene dada por un \gls{rf} basado en 25 estimadores o árboles y en el criterio de medición de la impureza a partir de la entropía.

\vspace{3mm}

\begin{table}[H]
  \centering
  \begin{subtable}{0.45\linewidth}
    \centering
    \begin{tabular}{|>{\columncolor[HTML]{EFEFEF}}c |c|c|}
      \hline
      \textit{\begin{tabular}[c]{@{}c@{}}Criterio /\\ Nº estimadores\end{tabular}} & \cellcolor[HTML]{EFEFEF}\textit{Entropía} & \cellcolor[HTML]{EFEFEF}\textit{Gini} \\ \hline
      10 & 99,07 & 99,02 \\ \hline
      25 & 99,16 & 99,14 \\ \hline
      50 & 99,16 & 99,15 \\ \hline
      75 & 99,17 & 99,17 \\ \hline
      100 & 99,18 & 99,16 \\ \hline
    \end{tabular}
    \caption{Precisión (\%) (\textit{mean\_test\_score})}
    \label{tab:rfgs}
  \end{subtable}
  \hfill
  \begin{subtable}{0.45\linewidth}
    \centering
    \begin{tabular}{|>{\columncolor[HTML]{EFEFEF}}c |c|c|}
      \hline
      \textit{\begin{tabular}[c]{@{}c@{}}Criterio /\\ Nº Estimadores\end{tabular}} & \cellcolor[HTML]{EFEFEF}\textit{Entropía} & \cellcolor[HTML]{EFEFEF}\textit{Gini} \\ \hline
      10 & 147 & 146 \\ \hline
      25 & 373 & 382 \\ \hline
      50 & 747 & 761 \\ \hline
      75 & 1072 & 1002 \\ \hline
      100 & 1439 & 987 \\ \hline
    \end{tabular}
    \caption{Tiempo (s) (\textit{mean\_fit\_time})}
    \label{tab:rfgs2}
  \end{subtable}
  \caption{Resultados extraídos del atributo \textit{cv\_results\_} del \textit{Grid Search} en el \acrshort{rf}}
  \label{tab:rfgs_combined}
\end{table}

\subsubsection{Selección de características}
\label{sec:rf3}

El proceso de selección de características viene dado por la necesidad de reducir las dimensiones del conjunto de datos y eliminar la información irrelevante o redundante que introduce ruido en el conjunto. En esta Sección, se expone el empleo de tres técnicas diferentes con el fin de realizar posteriormente un análisis comparativo de los resultados que se obtienen tras aplicar cada una de ellas al conjunto de datos (ver Sección \ref{sec:rf4}).

\vspace{3mm}

En primer lugar, se introduce la técnica de eliminación recursiva de características con validación cruzada (del inglés \gls{rfecv}) \cite{rfecv}. Esta técnica se basa en la ejecución de un proceso iterativo para ir desechando las características que tienen menor influencia en los resultados de precisión, hasta que el rendimiento del modelo deja de mejorar significativamente. Por este motivo, además de proveer a su salida la lista de características más importantes, es capaz de indicar cuál es el número óptimo de características que se debería aplicar para maximizar la precisión en el entrenamiento del modelo, a la vez que se minimiza el volumen de datos en el mismo. En este caso, el \gls{rfecv} se configura con 5 divisiones (\textit{cv}) del conjunto de datos para llevar a cabo el proceso de evaluación. Además, se determina la eliminación de una de las características disponibles en cada iteración (\textit{step}). 

\vspace{3mm}

\begin{lstlisting}[style=Python, caption={Aplicación del \acrshort{rfecv}}]
  rfecv = RFECV(estimator=classifier, step=1, cv=5, scoring='accuracy') 
  rfecv = rfecv.fit(X_train, y_train)
\end{lstlisting}

\vspace{3mm}

\begin{lstlisting}[language=bash, style=Python, caption={Resultados del \acrshort{rfecv}}]
  Nº óptimo de features : 8
  Selected features : Index(['cap', 'dist', 'origen_id', 'dest_id', 'len_origen_tag', 'len_dest_tag', 'total_balance', 'abs_flux'], dtype='object')
\end{lstlisting}

\vspace{3mm}

Se puede visualizar el proceso de evaluación del número de características gráficamente en la Figura \ref{fig:rfecv}, en la cual se representa cómo la precisión del modelo es máxima cuando se emplean 8. Por otro lado, en cuanto a la lista de características proporcionada por el \gls{rfecv}, es preciso llevar a cabo una comparación con las puntuaciones obtenidas en la Sección \ref{sec:rf1}. Volviendo a las Figuras \ref{fig:imp1} y \ref{fig:imp2}, se confirma que, exceptuando ligeras variaciones, la lista de características es coherente con las que presentan mayor grado de importancia.

\vspace{3mm}

\begin{figure}[H]
  \centering
  \includegraphics[width=0.9\textwidth]{img/desarrollo/rf/rfecv.png}
  \caption{Análisis de la precisión del modelo en función del número de características empleadas}
  \label{fig:rfecv}
\end{figure}

Por consiguiente, se entra en el funcionamiento de la segunda técnica a emplear, denominada como Selección Invariante de Características (del inglés \textit{Univariate feature selection}) o, también denominado, \textit{kBest} \cite{kbest}. En este caso, la operativa incluye un escalado previo de las características, para que sus valores se sitúen en un rango [0, 1]. 

\vspace{3mm}

Después, se declara el objeto de la clase especificada para la técnica en cuestión, \textit{SelectKBest()}, y se determina cuántas características se pretenden seleccionar del total que provee el conjunto. A diferencia del \gls{rfecv}, para conocer cuál es el número óptimo en la selección invariante de características, es preciso basarse en la prueba y error, puesto que no se proporciona este dato a la salida. Por ello, a modo comparativo, se produce el entrenamiento del conjunto de datos en base a k=5 y a k=8 características. 

\vspace{3mm}

\begin{lstlisting}[style=Python, caption={Aplicación del \textit{kBest}}]
  scaler = MinMaxScaler() # Escalado de las características en el rango [0, 1]
  kbest = SelectKBest(chi2, k=8)
  kbest = kbest.fit(scaler.fit_transform(X_train), y_train)
\end{lstlisting}

\vspace{3mm}

Una vez que se extraen aquellas características con mayor puntuación, a partir del atributo \textit{feature\_names\_in}, se procede a transformar los conjuntos de entrenamiento y de test. Si se compara la lista proporcionada por el atributo anterior con las puntuaciones de las Figuras \ref{fig:imp1} y \ref{fig:imp2}, se puede comprobar que esta técnica no es tan precisa como en el caso del \gls{rfecv}, puesto que se seleccionan algunas características que no presentan tan buenas puntuaciones. Con ello, se podría estimar, a priori, que la implementación de la selección invariante no mejorará los resultados de precisión que se esperan con \gls{rfecv}.

\vspace{3mm}

\begin{lstlisting}[language=bash, style=Python, caption={Resultados del \textit{kBest} para \textit{k=8}}]
  Selected features : Index(['cap', 'len_origen_tag', 'len_dest_tag', 'degree', 'abs_flux', 'Beam Irradiance (W/m2)', 'Plane of Array Irradiance (W/m2)', 'Cell Temperature (C)'], dtype='object')       
\end{lstlisting}

\vspace{3mm}

Por último, se introduce una técnica, que no es estrictamente de selección de características, pero que se incluye en esta Sección, puesto que el objetivo que se persigue viene dado por la reducción de la dimensionalidad del conjunto de datos. Se denomina como Análisis de Componentes Principales (del inglés \gls{pca}) \cite{pca} y se basa en la aplicación de la técnica matemática de Descomposición en Valores Singulares (del inglés \gls{svd}). Es decir, su funcionamiento consiste en la búsqueda de las direcciones principales de variación (k vectores) del conjunto de datos para construir una matriz de proyección, que establezca un nuevo espacio de características de k dimensiones.

\vspace{3mm}

No obstante, antes de aplicar el ajuste y la transformación al conjunto de datos a partir del \gls{pca}, se debe configurar el número de componentes a emplear. Para ello, se utiliza el ``método del codo" (del inglés \textit{Elbow Method}) con el fin de identificar el punto o ``codo" en el que el aumento del número de componentes no aporta mejoras significativas en el valor de la varianza. Como se representa en la Figura \ref{fig:pca}, en n=4 componentes, se produce el estancamiento más considerable de la varianza, en un valor del 6\% aproximadamente. Se puede visualizar que, para el caso de n=2 componentes, también ocurre pero en menor medida. Entonces, a modo comparativo, se considerará aplicar el \gls{pca} en función de ambos números.

\vspace{3mm}

\begin{figure}[H]
  \centering
  \includegraphics[width=0.9\textwidth]{img/desarrollo/pca.png}
  \caption{Análisis de la varianza en función del número de componentes empleadas en el \acrshort{pca}}
  \label{fig:pca}
\end{figure}


\subsubsection{Ejecución del modelo y evaluación de resultados}
\label{sec:rf4}

Por un lado, como se ha detallado en la Sección \ref{rf2}, el modelo que se considera más adecuado y, por tanto, sobre el que se decide trabajar, viene dado por la configuración de 25 estimadores y el criterio de la entropía. Por otro lado, en la Sección \ref{sec:rf3}, se han expuesto varias propuestas de selección de características o de reducción de dimensiones del conjunto de datos. 

\vspace{3mm}

\begin{lstlisting}[style=Python, caption={Clasificador RF seleccionado}]
  classifier = RandomForestClassifier(n_estimators = 25, criterion = 'entropy', random_state = 0) 
  classifier.fit(X_train, y_train)
\end{lstlisting}
  
\vspace{3mm}

La presente Sección viene dada por la necesidad de aplicar un proceso de evaluación sobre las diferentes implementaciones del modelo de \gls{rf} seleccionado con el fin de analizar el rendimiento y precisión que proporcionan cada una de ellas. En primera instancia, se va a proceder a ejecutar el modelo de \gls{rf}, tanto empleando las diferentes opciones de reducción de dimensiones del conjunto de datos, como sin aplicar ninguna de ellas con el objetivo de realizar la comparativa. A modo de proveer una mayor comprensión del entrenamiento que se produce, se incluye la lógica de representación de los estimadores o árboles de decisión del \gls{rf}, mediante el empleo del método de conversión \textit{export\_graphviz} \cite{graphviz2} y la herramienta de generación de grafos, \textit{WebGraphViz}. En la Figura \ref{fig:tree} se expone uno de los árboles que se construyen en este proceso.

\vspace{3mm}

Por consiguiente, una vez ejecutado, se puede llevar a cabo el proceso de evaluación, que consta de dos métodos: la matriz de confusión y el \textit{K-Fold Cross Validation}. La matriz de confusión \cite{cm} es un método de gran utilidad cuando se trabaja con técnicas de clasificación binaria, como se produce en este caso, para detectar y predecir los intercambios energéticos con errores. Permite analizar la precisión del modelo mediante la organización de las predicciones en cuatro categorías diferentes (ver Figura \ref{fig:confusion}): 

\begin{itemize}
  \item Verdaderos positivos (TP): Predicciones correctas de intercambios con errores.
  \item Falsos positivos (FP): Predicciones incorrectas de intercambios con errores.
  \item Verdaderos negativos (TN): Predicciones correctas de intercambios sin errores.
  \item Falsos negativos (FN): Predicciones incorrectas de intercambios sin errores.
\end{itemize}

\begin{sidewaysfigure}
  \centering
  \includegraphics[width=1\textwidth]{img/desarrollo/rf/tree2.png}
  \caption{Ejemplo gráfico de un estimador o árbol del \acrshort{rf} \cite{graphviz}}
  \label{fig:tree}
\end{sidewaysfigure}

\begin{figure}[H]
  \centering
  \includegraphics[width=0.45\textwidth]{img/desarrollo/rf/confusion.png}
  \caption{Matriz de confusión \cite{cm2}}
  \label{fig:confusion}
\end{figure}

\vspace{3mm}

A través de las categorías definidas en la matriz, se pueden obtener cuatro métricas de evaluación: 

\vspace{3mm}

\begin{itemize}
  \item \textit{Accuracy}: Mide la proporción de predicciones correctas en relación con el total y evalúa el rendimiento general de un modelo de clasificación.
  
  \begin{equation}
    \begin{aligned}
      \textit{Accuracy} = \frac{{TP + TN}}{{TP + TN + FP + FN}}
    \end{aligned}
  \end{equation}

\pagebreak 

  \item \textit{Precision}: Mide la proporción de verdaderos positivos entre todas las instancias clasificadas como positivas.
  
  \begin{equation}
    \begin{aligned}
      \text{Precision} = \frac{{TP}}{{TP + FP}}
    \end{aligned}
  \end{equation}

  \item \textit{Recall}: Mide la proporción de verdaderos positivos identificados correctamente entre todas las instancias que son realmente positivas. 
  
  \begin{equation}
    \begin{aligned}
      \text{Recall} = \frac{{TP}}{{TP + FN}}
    \end{aligned}
  \end{equation}

  \item \textit{F1 Score}: Combina las dos métricas anteriores, tomando en cuenta tanto los falsos positivos como los falsos negativos.
  
  \begin{equation}
    \begin{aligned}
      \text{F1 Score} = \frac{{2 \times \text{Precision} \times \text{Recall}}}{{\text{Precision} + \text{Recall}}}
    \end{aligned}
  \end{equation}
  
\end{itemize}

\begin{lstlisting}[style=Python, caption={Implementación de la matriz de confusión}]
  cm = confusion_matrix(y_test, y_pred)
  accuracy_score(y_test, y_pred)
\end{lstlisting}

\vspace{3mm}

Por tanto, con la aplicación de la matriz de confusión, se extraen los resultados expuestos en la Tabla \ref{tab:rfcm}. En la misma, se puede visualizar la cantidad de predicciones, en función de cada una de las categorías de la matriz y los valores obtenidos del cálculo de las métricas anteriores. A priori, si solo se pone el foco en la precisión global del modelo (\textit{Accuracy}), las opciones que proveen valores más altos son la que no aplica ningún método de reducción de dimensiones y que por tanto, opera con todas las características del conjunto, y la que emplea la técnica \gls{rfecv}. 

\vspace{3mm}

En el caso de la precisión de los valores positivos (\textit{Precision}) o, en otros términos, de las instancias que se predicen como erróneas, vuelve a ocurrir de la misma manera. A pesar de que para la métrica \textit{Accuracy} los valores de todas las opciones son relativamente parecidos, para \textit{Precision} se visualizan grandes diferencias. Por ejemplo, el empleo de un \gls{pca} con n=2 provee un buen valor de precisión global (97,55\%), pero no tiene un buen rendimiento en la clasificación de los errores (12,52\%). En consecuencia, el valor del \textit{Recall} también es muy pequeño en este caso (0,67\%). El resto de opciones, exceptuando las dos primeras, ya mencionadas anteriormente, presentan también porcentajes bajos, suponiendo un coste alto de predicción de errores. 

\vspace{3mm}

\begin{table}[H]
  \centering
  \begin{tabular}{|c|c|c|c|c|c|c|c|c|}
  \hline
  \rowcolor[HTML]{EFEFEF} 
  \textit{\begin{tabular}[c]{@{}c@{}}Matriz\\ de confusión\end{tabular}} & \cellcolor[HTML]{EFEFEF}\textit{TN} & \textit{FP} & \textit{FN} & \textit{TP} & \textit{Accuracy} & \textit{Precision} & \textit{Recall} & \textit{F1 Score} \\ \hline
  \cellcolor[HTML]{EFEFEF}\textit{Sin aplicar} & 376745 & 399 & 2332 & 6732 & 99,29 & 94,40 & 74,27 & 83,16 \\ \hline
  \cellcolor[HTML]{EFEFEF}\textit{RFECV} & 376699 & 445 & 1717 & 7347 & 99,44 & 94,28 & 81,05 & 87,17 \\ \hline
  \cellcolor[HTML]{EFEFEF}\textit{kbest (n=5)} & 376555 & 589 & 7927 & 1137 & 97,79 & 65,97 & 12,55 & 21,08 \\ \hline
  \cellcolor[HTML]{EFEFEF}\textit{kbest (n=8)} & 375027 & 2117 & 6583 & 2481 & 97,74 & 53,95 & 27,37 & 36,32 \\ \hline
  \cellcolor[HTML]{EFEFEF}\textit{PCA (n=2)} & 376718 & 426 & 9003 & 61 & 97,55 & 12,52 & 00,67 & 01,27 \\ \hline
  \cellcolor[HTML]{EFEFEF}\textit{PCA (n=4)} & 376684 & 460 & 7015 & 2049 & 98,06 & 81,66 & 22,60 & 35,41 \\ \hline
  \end{tabular}
  \caption{Resultados de aplicación de la matriz de confusión al \acrshort{rf}}
  \label{tab:rfcm}
\end{table}

\vspace{3mm}

De la misma forma, se puede visualizar que para la métrica \textit{F1 Score} se esperan resultados relativamente cercanos al \textit{Recall}. Al considerarse tanto los falsos positivos, como los falsos negativos, produce que la obtención de un porcentaje significativamente pequeño indique una baja \textit{Precision} y \textit{Recall} conjuntamente. 

\vspace{3mm}

Por otro lado, en cuanto al método \textit{K-Fold Cross Validation} \cite{kfold}, se emplea la misma operativa que en el \textit{Grid Search}. Como su nombre indica, se basa en un esquema de validación cruzada (\textit{cross validation}) y se divide el conjunto de datos en un total de 5 pliegues para evaluar el rendimiento del modelo. En la Tabla \ref{tab:rfk}, se determinan los resultados de su aplicación y, como se puede visualizar, los valores de precisión global son prácticamente iguales a los obtenidos anteriormente en la matriz de confusión. 

\vspace{3mm}

Por tanto, mediante el empleo de las dos técnicas y el análisis expuesto, se puede confirmar definitivamente que las opciones con mejores resultados hacen referencia a la que no se aplica ningún método de reducción de dimensiones y a la que emplea \gls{rfecv}, siendo esta última la que proporciona un rendimiento óptimo.

\vspace{3mm}

\begin{lstlisting}[style=Python, caption={Implementación del \textit{K-Fold Cross Validation}}]
accuracies = cross_val_score(estimator = classifier, X = X_train, y = y_train, cv = 5)
\end{lstlisting}

\vspace{3mm}
\clearpage

\begin{table}[H]
  \centering
  \begin{tabular}{|c|c|c|c|c|c|c|c|c|}
  \hline
  \rowcolor[HTML]{EFEFEF} 
  \textit{\begin{tabular}[c]{@{}c@{}}K-Fold\\ Cross Validation\end{tabular}} & \cellcolor[HTML]{EFEFEF}\textit{Accuracy (\%)} & \textit{Standard Deviation (\%)} \\ \hline
  \cellcolor[HTML]{EFEFEF}\textit{Sin aplicar} & 99,30 & 0,02 \\ \hline
  \cellcolor[HTML]{EFEFEF}\textit{RFECV} & 99,47 & 0,02 \\ \hline
  \cellcolor[HTML]{EFEFEF}\textit{kbest (n=5)} & 97,80 & 0,02 \\ \hline
  \cellcolor[HTML]{EFEFEF}\textit{kbest (n=8)} & 97,79 & 0,01 \\ \hline
  \cellcolor[HTML]{EFEFEF}\textit{PCA (n=2)} & 97,35 & 0,01 \\ \hline
  \cellcolor[HTML]{EFEFEF}\textit{PCA (n=4)} & 98,04 & 0,00 \\ \hline
  \end{tabular}
  \caption{Resultados de aplicación del \textit{K-Fold Cross Validation} al \gls{rf}}
  \label{tab:rfk}
\end{table}





%%%%%%%%%%%%%%%%%%%%%%%%%%%%%%%%%%%%%%%%%%%%%%%%%%%%%%%%%%%%%%%%%%
\subsection{Support Vector Machines (\acrshort{svm})}
\label{sec:svm}




\subsubsection{Diseño y ejecución del modelo de prueba}


\subsubsection{Puntuación de características}

\subsubsection{Optimización de hiperparámetros}


% C a fin de cuentas el hiperparámetro encargado de controlar el balance entre bias y varianza del modelo. En la práctica, su valor óptimo se identifica mediante validación cruzada. Esta es la razón por la que el parámetro  C controla el balance entre bias y varianza lo que permite un ajuste adecuado del modelo. 

%Cuando el valor de  Ces pequeño, el margen es más ancho, y más observaciones violan el margen, convirtiéndose en vectores soporte. El hiperplano está, por lo tanto, sustentado por más observaciones, lo que aumenta el bias pero reduce la varianza. 

%Cuando mayor es el valor de  C , menor el margen, menos observaciones son vectores soporte y el clasificador resultante tiene menor bias pero mayor varianza.

%Otra propiedad importante que deriva de que el hiperplano dependa únicamente de una pequeña proporción de observaciones (vectores soporte), es su robustez frente a observaciones muy alejadas del hiperplano. Esto hace al método de clasificación vector soporte distinto a otros métodos tales como Linear Discrimiant Analysis (LDA), donde la regla de clasificación depende de la media de todas las observaciones.

%%gird search tiempo
%https://datascience.stackexchange.com/questions/29495/how-to-estimate-gridsearchcv-computing-time

\subsubsection{Selección de características}

\subsubsection{Ejecución y validación de los modelos}

%%%%%%%%%%%%%%%%%%%%%%%%%%%%%%%%%%%%%%%%%%%%%%%%%%%%%%%%%%%%%%
\section{Técnicas de \gls{dl}}
\label{sec:tecnicasdl}

De la misma forma que se ha realizado para las técnicas de \gls{ml}, se debe indicar la secuencia de acciones a seguir en esta Sección referente al \gls{dl}. No obstante, antes que nada, es necesario volver a la Figura \ref{fig:features}, expuesta en la Sección \ref{sec:dl}. En la misma, se podían apreciar las diferencias estructurales que presentan las técnicas de \gls{ml} y de \gls{dl} entre sí. Como estas radican principalmente en el tratamiento de las características del conjunto de datos, para proceder al desarrollo de los diferentes modelos que se exponen en esta Sección, se deben eliminar los pasos de puntuación y selección comentados en la Sección \ref{sec:tecnicasml} (pasos 2 y 4). En cuanto al resto de la secuencia de acciones, se mantiene en el mismo orden.

%%%%%%%%%%%%%%%%%%%%%%%%%%%%%%%%%%%%%%%%%%%%%%%%%%%%%%%%%%%%%%%%%%
\subsection{Redes Neuronales Artificiales (\acrshort{ann})}
\label{sec:ann}

\subsubsection{Optimización de hiperparámetros}

Antes de entrar en detalle en la optimización de hiperparámetros de la \gls{ann}, es preciso indicar que esta Sección se divide en dos fases: primero, se centra la aplicación del método \textit{Grid Search} sobre el modelo de \gls{mlp} proporcionado por la librería \textit{sklearn} y, después, los resultados obtenidos de la búsqueda anterior se utilizan para configurar una nueva \gls{ann} óptima, basada en el módulo \textit{keras}. Esta secuencia de pasos se establece a modo de simplificar la comprensión y de justificar y comparar los resultados que se obtienen para ambas versiones de \gls{ann}s.

\vspace{3mm}

Por ello, en primera instancia, se determinan los hiperparámetros que se van a estudiar del \gls{mlp}. Estos hacen referencia a la configuración de capas ocultas y al número de neuronas que puede tener cada capa (\textit{hidden\_layer\_sizes}), a la función de activación que se aplica (\textit{activation}) y al algoritmo de optimización de los pesos de la red durante el proceso de entrenamiento (\textit{solver}). \cite{mlp}

\vspace{3mm}

\begin{lstlisting}[style=Python, caption={Cuadrícula de parámetros MLP}]
  param_grid = {
    'hidden_layer_sizes': [(5,), (8,), (10,), (50,), (100,), (5, 5), (8, 8), (10, 10), (50, 50)],  
    'activation': ['relu', 'tanh'],
    'solver': ['sgd', 'adam']
  }
\end{lstlisting}

\vspace{3mm}

En este caso, se configura para la búsqueda un modelo por defecto de \gls{mlp}, en el que se aplican un máximo de 100 iteraciones por cada combinación a probar. De la misma forma, para no introducir latencias innecesarias ni un sobreentrenamiento que perjudique a los resultados de clasificación, se determina una finalización temprana del entrenamiento, en el caso de que se se prooduzcan una cantidad de iteraciones seguidas sin mejoras significativas. En este caso, se deja el valor por defecto, que es 10 iteraciones y se modifica la tolerancia a un valor de 0.00001.

\vspace{3mm}

\begin{lstlisting}[style=Python, caption={Clasificador MLP por defecto}]
  mlp = MLPClassifier(max_iter=100, verbose=True, early_stopping=True, tol=0.00001)
\end{lstlisting}

\vspace{3mm}

La creación del objeto de la clase \textit{GridSearchCV()} con una configuración de 5 pliegues (\textit{cv=5}) y su entrenamiento sobre el \gls{mlp} anterior en un equipo de 32 procesadores, se estima en una duración de 2,34 horas. Cuando se lleva a cabo este proceso, se puede comprobar en el diccionario de resultados (\textit{cv\_results\_}) y, en particular, en los valores de precisión de la variable \textit{mean\_test\_score}, que se encuentran varias combinaciones de hiperparámetros que optimizan el rendimiento. En la Tabla \ref{tab:mlpaccuracy} se expone que el empleo del algoritmo de optimización del gradiente descendiente estocástico (\gls{sgd}) aporta mejores resultados que el \textit{adam} y que se consigue un valor de precisión máximo igual al 97,6\%. 

\vspace{3mm}

Teniendo esto en cuenta, se podría tomar cualquiera de las configuraciones de capas que proporcionan esta precisión. Sin embargo, también es importante poner el foco en los valores de desviación típica que se consiguen y en la duración del entrenamiento que supone cada opción para determinar cuál es la combinación de hiperparámetros más adecuada para el \gls{mlp}. 

\vspace{3mm}

\begin{table}[H]
    \centering
    \begin{tabular}{|
    >{\columncolor[HTML]{EFEFEF}}c |cc|cc|}
    \hline
    \textit{Solver} & \multicolumn{2}{c|}{\cellcolor[HTML]{EFEFEF}\textit{adam}} & \multicolumn{2}{c|}{\cellcolor[HTML]{EFEFEF}\textit{sgd}} \\ \hline
    \textit{\begin{tabular}[c]{@{}c@{}}Función de activación /\\ Capas ocultas\end{tabular}} & \multicolumn{1}{c|}{\cellcolor[HTML]{EFEFEF}\textit{relu}} & \cellcolor[HTML]{EFEFEF}\textit{tanh} & \multicolumn{1}{c|}{\cellcolor[HTML]{EFEFEF}\textit{relu}} & \cellcolor[HTML]{EFEFEF}\textit{tanh} \\ \hline
    (5,) & \multicolumn{1}{c|}{0,9017} & \multicolumn{1}{c|}{0,9761} & \multicolumn{1}{c|}{0,9766} & \multicolumn{1}{c|}{0,9754} \\ \hline
    (8,) & \multicolumn{1}{c|}{0,9203} & \multicolumn{1}{c|}{0,8973} & \multicolumn{1}{c|}{0,9765} & \multicolumn{1}{c|}{0,9765} \\ \hline
    (10,) & \multicolumn{1}{c|}{0,8700} & \multicolumn{1}{c|}{0,8990} & \multicolumn{1}{c|}{0,9748} & \multicolumn{1}{c|}{0,8976} \\ \hline
    (50,) & \multicolumn{1}{c|}{0,8689} & \multicolumn{1}{c|}{0,8982} & \multicolumn{1}{c|}{0,8953} & \multicolumn{1}{c|}{0,9765} \\ \hline
    (100,) & \multicolumn{1}{c|}{0,9095} & \multicolumn{1}{c|}{0,8971} & \multicolumn{1}{c|}{0,8456} & \multicolumn{1}{c|}{0,8972} \\ \hline
    (5, 5) & \multicolumn{1}{c|}{0,8980} & \multicolumn{1}{c|}{0,8972} & \multicolumn{1}{c|}{0,9766} & \multicolumn{1}{c|}{0,9765} \\ \hline
    (8, 8) & \multicolumn{1}{c|}{0,8849} & \multicolumn{1}{c|}{0,8972} & \multicolumn{1}{c|}{0,8989} & \multicolumn{1}{c|}{0,9765} \\ \hline
    (10, 10) & \multicolumn{1}{c|}{0,8168} & \multicolumn{1}{c|}{0,8970} & \multicolumn{1}{c|}{0,9030} & \multicolumn{1}{c|}{0,9765} \\ \hline
    (50, 50) & \multicolumn{1}{c|}{0,8191} & \multicolumn{1}{c|}{0,8204} & \multicolumn{1}{c|}{0,9766} & \multicolumn{1}{c|}{0,8973} \\ \hline
    \end{tabular}
    \caption{Resultados de precisión (\%) (\textit{mean\_test\_score}) extraídos del atributo \textit{cv\_results\_} del \acrshort{mlp}}
    \label{tab:mlpaccuracy}
\end{table}

\vspace{3mm}

En el caso de la desviación típica, como se observa en la Tabla \ref{tab:mlpstd}, no se producen grandes diferencias de valores entre las configuraciones en cuestión, por lo que, a priori, no es un factor determinante y que se deba tener en cuenta en la selección. No obstante, en el caso de los tiempos sí existen variaciones y esto se debe pricipalmente, al número de iteraciones que son necesarias para alcanzar la convergencia para cada versión de \gls{mlp} y a la duración que supone cada iteración. En la Tabla \ref{tab:mlptiempo}, se puede visualizar que, cuanto mayor sea el número de neuronas especificado, mayor será la latencia introducida. Esto presenta también, cierta correlación con los valores de la Tabla \ref{tab:mlpaccuracy}, ya que cuando el proceso de entrenamiento es relativamente largo, es posible que se produzca un sobreentrenamiento o \textit{overfitting} del modelo de \gls{mlp} y que la precisión obtenida sea más baja. Por ello, es importante configurar la tolerancia de optimización de forma correcta y parar el proceso cuando se detecte la convergencia del modelo (\textit{early stopping}).

\vspace{3mm}

\begin{table}[H]
    \centering
    \begin{tabular}{|
    >{\columncolor[HTML]{EFEFEF}}c |cc|cc|}
    \hline
    \textit{Solver} & \multicolumn{2}{c|}{\cellcolor[HTML]{EFEFEF}\textit{adam}} & \multicolumn{2}{c|}{\cellcolor[HTML]{EFEFEF}\textit{sgd}} \\ \hline
    \textit{\begin{tabular}[c]{@{}c@{}}Función de activación /\\ Capas ocultas\end{tabular}} & \multicolumn{1}{c|}{\cellcolor[HTML]{EFEFEF}\textit{relu}} & \cellcolor[HTML]{EFEFEF}\textit{tanh} & \multicolumn{1}{c|}{\cellcolor[HTML]{EFEFEF}\textit{relu}} & \cellcolor[HTML]{EFEFEF}\textit{tanh} \\ \hline
    (5,) & \multicolumn{1}{c|}{0,1475} & \multicolumn{1}{c|}{0,0010} & \multicolumn{1}{c|}{0,0001} & \multicolumn{1}{c|}{0,0022} \\ \hline
    (8,) & \multicolumn{1}{c|}{0,1133} & \multicolumn{1}{c|}{0,1584} & \multicolumn{1}{c|}{0,0000} & \multicolumn{1}{c|}{0,0000} \\ \hline
    (10,) & \multicolumn{1}{c|}{0,1554} & \multicolumn{1}{c|}{0,1551} & \multicolumn{1}{c|}{0,0033} & \multicolumn{1}{c|}{0,1578} \\ \hline
    (50,) & \multicolumn{1}{c|}{0,1558} & \multicolumn{1}{c|}{0,1574} & \multicolumn{1}{c|}{0,1109} & \multicolumn{1}{c|}{0,0000} \\ \hline
    (100,) & \multicolumn{1}{c|}{0,0940} & \multicolumn{1}{c|}{0,1590} & \multicolumn{1}{c|}{0,1665} & \multicolumn{1}{c|}{0,1587} \\ \hline
    (5, 5) & \multicolumn{1}{c|}{0,1583} & \multicolumn{1}{c|}{0,1587} & \multicolumn{1}{c|}{0,0001} & \multicolumn{1}{c|}{0,0000} \\ \hline
    (8, 8) & \multicolumn{1}{c|}{0,1219} & \multicolumn{1}{c|}{0,1587} & \multicolumn{1}{c|}{0,1552} & \multicolumn{1}{c|}{0,0000} \\ \hline
    (10, 10) & \multicolumn{1}{c|}{0,1977} & \multicolumn{1}{c|}{0,1589} & \multicolumn{1}{c|}{0,1471} & \multicolumn{1}{c|}{0,0000} \\ \hline
    (50, 50) & \multicolumn{1}{c|}{0,1952} & \multicolumn{1}{c|}{0,1920} & \multicolumn{1}{c|}{0,0001} & \multicolumn{1}{c|}{0,1585} \\ \hline
    \end{tabular}
    \caption{Resultados de desviación típica (\%) (\textit{std\_test\_score}) extraídos del atributo \textit{cv\_results\_} del \acrshort{mlp}}
    \label{tab:mlpstd}
\end{table}

\vspace{3mm}

\begin{table}[H]
    \centering
    \begin{tabular}{|
    >{\columncolor[HTML]{EFEFEF}}c |cc|cc|}
    \hline
    \textit{Solver} & \multicolumn{2}{c|}{\cellcolor[HTML]{EFEFEF}\textit{adam}} & \multicolumn{2}{c|}{\cellcolor[HTML]{EFEFEF}\textit{sgd}} \\ \hline
    \textit{\begin{tabular}[c]{@{}c@{}}Función de activación /\\ Capas ocultas\end{tabular}} & \multicolumn{1}{c|}{\cellcolor[HTML]{EFEFEF}\textit{relu}} & \cellcolor[HTML]{EFEFEF}\textit{tanh} & \multicolumn{1}{c|}{\cellcolor[HTML]{EFEFEF}\textit{relu}} & \cellcolor[HTML]{EFEFEF}\textit{tanh} \\ \hline
    (5,) & \multicolumn{1}{c|}{88} & \multicolumn{1}{c|}{37} & \multicolumn{1}{c|}{63} & \multicolumn{1}{c|}{35} \\ \hline
    (8,) & \multicolumn{1}{c|}{96} & \multicolumn{1}{c|}{43} & \multicolumn{1}{c|}{35} & \multicolumn{1}{c|}{36} \\ \hline
    (10,) & \multicolumn{1}{c|}{94} & \multicolumn{1}{c|}{40} & \multicolumn{1}{c|}{56} & \multicolumn{1}{c|}{43} \\ \hline
    (50,) & \multicolumn{1}{c|}{157} & \multicolumn{1}{c|}{168} & \multicolumn{1}{c|}{143} & \multicolumn{1}{c|}{70} \\ \hline
    (100,) & \multicolumn{1}{c|}{236} & \multicolumn{1}{c|}{223} & \multicolumn{1}{c|}{170} & \multicolumn{1}{c|}{131} \\ \hline
    (5, 5) & \multicolumn{1}{c|}{136} & \multicolumn{1}{c|}{50} & \multicolumn{1}{c|}{52} & \multicolumn{1}{c|}{44} \\ \hline
    (8, 8) & \multicolumn{1}{c|}{159} & \multicolumn{1}{c|}{54} & \multicolumn{1}{c|}{69} & \multicolumn{1}{c|}{46} \\ \hline
    (10, 10) & \multicolumn{1}{c|}{149} & \multicolumn{1}{c|}{55} & \multicolumn{1}{c|}{54} & \multicolumn{1}{c|}{48} \\ \hline
    (50, 50) & \multicolumn{1}{c|}{333} & \multicolumn{1}{c|}{319} & \multicolumn{1}{c|}{166} & \multicolumn{1}{c|}{155} \\ \hline
    \end{tabular}
    \caption{Resultados de tiempo (s) (\textit{mean\_fit\_time}) extraídos del atributo \textit{cv\_results\_} del \acrshort{mlp}}
    \label{tab:mlptiempo}
\end{table}

\vspace{3mm}

Tras analizar los resultados, se expone de forma concluyente que la combinación de hiperparámetros más adecuada y, que por tanto, debería ser la configurada en el modelo de \gls{mlp}, es la que determina una estructura de dos capas ocultas de 5 neuronas, una función de activación \textit{relu} y el algoritmo de optimización \gls{sgd}. En este caso, como se puede apreciar en las Figuras \ref{fig:1mlpbestacc} y \ref{fig:1mlpbestloss} el proceso de entrenamiento converge en la iteración 41 con una precisión del 97,66\% y una función de pérdidas con valor 0,0712.

\vspace{3mm}

\begin{lstlisting}[style=Python, caption={Clasificador MLP óptimo}]
  mlp = MLPClassifier(max_iter=100, verbose=True, early_stopping=True, tol=0.00001, hidden_layer_sizes=(5, 5), activation='relu', solver='sgd')
\end{lstlisting}

\vspace{3mm}

\begin{figure}[H]
  \centering
  \includegraphics[width=1\textwidth]{img/desarrollo/ann/1mlpbestacc.png}
  \caption{Representación del valor de precisión en función de las iteraciones para el modelo \acrshort{mlp} escogido}
  \label{fig:1mlpbestacc}
\end{figure}

\begin{figure}[H]
  \centering
  \includegraphics[width=1\textwidth]{img/desarrollo/ann/1mlpbestloss.png}
  \caption{Representación del valor de la función de pérdidas en función de las iteraciones para el modelo \acrshort{mlp} escogido}
  \label{fig:1mlpbestloss}
\end{figure}

\vspace{3mm}

Como se introducía en esta Sección, una vez seleccionada la combinación de hiperparámetros que produce un rendimiento óptimo del \gls{mlp}, el siguiente paso consiste en aplicar la configuración en cuestión a una nuevo modelo de \gls{ann}. El objetivo es realizar una comparativa entre dos \gls{ann}s proporcionadas por distintas librerías, como son en este caso \textit{sklearn} y el módulo \textit{keras} de \textit{tensorflow}, y comprobar que los resultados obtenidos anteriormente de aplicar el método \textit{Grid Search} son coherentes. 

\vspace{3mm}

Por ello, se define la estructura del modelo de \gls{ann} de \textit{keras} con las dos capas ocultas de 5 neuronas y una capa de una neurona a la salida, puesto que se trabaja con un conjunto de datos con etiquetas binarias. Además, es necesario especificar a la entrada el número de características (\textit{input\_shape}), ya que será igual al número de entradas de la red neuronal. La estructura del modelo definido se representa gráficamente en la Figura \ref{fig:neuronas}

\vspace{3mm}


%FIGURA NEURONAS

\begin{lstlisting}[style=Python, caption={Definición del modelo de ANN de Keras}]
  model = keras.Sequential([
    keras.layers.Dense(5, input_shape=(X.shape[1],), activation='relu'), 
    keras.layers.Dense(5, activation='relu'),
    keras.layers.Dense(1, activation='sigmoid') 
  ]) 
\end{lstlisting}

\vspace{3mm}



Por consiguiente, se 

\begin{lstlisting}[style=Python, caption={Entrenamiento del modelo de ANN de Keras}]
  model.compile(optimizer = 'sgd', loss = 'binary_crossentropy', metrics = ['accuracy'])
  History = model.fit(X_train, y_train, epochs = 100)
\end{lstlisting}



%History = model.fit(X_train, y_train, epochs=100, verbose=1, 
                            %callbacks=[keras.callbacks.EarlyStopping(monitor='val_loss', patience=5, verbose=1)])



\subsubsection{Ejecución del modelo y evaluación de resultados}

%loss functions -> solo el binary, los demás [-1,1]
%ciclo
%https://machinelearningmastery.com/5-step-life-cycle-neural-network-models-keras/






%%%%5

%diapo 36 + (batch)
%https://elvex.ugr.es/decsai/computational-intelligence/slides/N2%20Backpropagation.pdf




%overfitting
%https://machinelearningmastery.com/introduction-to-regularization-to-reduce-overfitting-and-improve-generalization-error/


