\chapter{Estado del arte}
\label{estadoArte}

En el presente capítulo se entrará en profundidad en los conceptos más relevantes relacionados con el \gls{tfm}. Por ello, el estado del arte se dividirá en una serie de Secciones enfocadas, por un lado, en el contexto de las \gls{sg}s y las redes \acrshort{iot} (ver Secciones \ref{sec:smartgrids} y \ref{sec:iot}) y, por otro lado, en la descripción teórica del concepto de \textit{big data} y de las técnicas de \gls{ml} y \gls{dl} (ver Secciones \ref{sec:bigdata} y \ref{sec:iateoria}) que se aplicarán en la etapa de desarrollo. 

\vspace{3mm}

De la misma forma, para aportar una buena compresión de los pasos de diseño que serán necesarios para llevar a cabo el planteamiento de escenarios de red, la generación de topologías y la ejecución de simulaciones, se añadirán las Secciones \ref{sec:den2ne} y \ref{sec:software}. La primera, expondrá las características que presenta el algoritmo \acrshort{den2ne} y la segunda, el funcionamiento de la plataforma \acrshort{brite}. Es por ello que se puede expresar que el propósito principal de este Capítulo se centra en establecer un marco teórico consistente y que, por tanto, permita conseguir una suficiente compresión del contexto, previamente a la exposición de las etapas de diseño y desarrollo del \gls{tfm}, en los Capítulos \ref{cha:analisis} y \ref{cha:desarrollo}.

\vspace{3mm}


% %%%%%%%%%%%%%%%%%%%%%%%%%%%%%%%%%%%%%%%%%%%%%%%%%%%%%%%%%%%%%%%%%%%%%%%%%%%%%%%%%%%%%%%%%%%%%%%%%%%%%
%%%%%%%%%%%%%%%%%%%%%%%%%%%%%%%%%%%%%%%%%%%%%%%%%%%%%%%%%%%%%%%%%%
\section{Smart Grids}
\label{sec:smartgrids}



% %%%%%%%%%%%%%%%%%%%%%%%%%%%%%%%%%%%%%%%%%%%%%%%%%%%%%%%%%%%%%%%%%%%%%%%%%%%%%%%%%%%%%%%%%%%%%%%%%%%%%
%%%%%%%%%%%%%%%%%%%%%%%%%%%%%%%%%%%%%%%%%%%%%%%%%%%%%%%%%%%%%%%%%%
\section{IoT}
\label{sec:iot}

El Internet de las Cosas (del inglés \acrfull{iot}) \cite{iot} \cite{iotamazon} se define como la integración de múltiples sensores o dispositivos físicos interconectados entre sí que se comunican a través de una red inalámbrica y recopilan información en tiempo real. 

\vspace{3mm}

La evolución del \gls{iot} tal y como se conoce en la actualidad tiene sus inicios en los años 90. Sin embargo, en ese entonces estas tecnologías se enfrentaban con la problemática del gran tamaño que presentaban los chips, por lo que se produjo un avance lento de las mismas. A medida que se permitió la reducción del volumen los dispositivos electrónicos y se mejoró la eficiencia y la capacidad de computación de los chips, se facilitó el progreso del \gls{iot}. En los últimos años, la integración de la tecnología 5G o de quinta generación móvil, ha supuesto el incremento de la velocidad de transmisión, procesamiento y análisis de los datos en los sistemas \gls{iot}. La capacidad de administrar a gran escala multitud de dispositivos físicos y la próxima implementación del 6G, depara un futuro en el que las tecnologías \gls{iot} estarán presentes en numerosos ámbitos y sectores.

\vspace{3mm}

Una de las grandes ventajas que proporcionan los sistemas \gls{iot} es la aparición de los entornos \gls{m2m} \cite{m2m}, en los cuales no es necesaria la intervención humana para posibilitar la transmisión y recepción de datos entre los elementos de la red. También, cabe destacar la reducción del coste de computación, la automatización de las tareas y la monitorización en tiempo real. Sin embargo, en cuanto a posibles desventajas, pueden presentar ciertas vulnerabilidades en cuestiones de privacidad y seguridad. Por ello, en la construcción de un sistema \gls{iot} se deben centrar esfuerzos en blindar los distintos elementos que lo componen frente a posibles ataques que puedan extraer información delicada o sabotear el funcionamiento. En cuanto a la estructura de un sistema \gls{iot}, se pueden diferenciar cuatro componentes que contribuyen al funcionamiento del mismo~\cite{iot}~\cite{iotamazon}:

\pagebreak

\begin{itemize}
    \item \textbf{Dispositivos inteligentes}: Se trata de cualquier elemento físico al que se le pueda asignar una dirección IPv6 y que tenga como mínimo capacidad de computación para recopilar datos del entorno o del usuario y transmitirlos a través de la red. 
    \item \textbf{Conectividad}: En este campo entran varias posibilidades de conexión: Wi-Fi, Bluetooth, redes de área amplia y baja potencia (del inglés \gls{lpwan}) como LoRaWAN, entre otras.
    \item \textbf{Aplicación}: Se compone de un conjunto de servicios de nube que se encargan de integrar y procesar todos los datos recopilados por los dispositivos del sistema \gls{iot}. Además, en algunos ámbitos, emplea técnicas de \gls{ml} para analizar la información que recibe y optimizar la toma de decisiones en el sistema.
    \item \textbf{Interfaz gráfica}: Desde la misma, el usuario puede gestionar y controlar el conjunto de elementos de la red. La interfaz está conectada directamente a la nube y puede tratarse de una aplicación móvil o web.
\end{itemize}

\begin{figure}[h!]
    \centering
    \includegraphics[width=0.7\textwidth]{img/teoria/iot.png}
    \caption{Infraestructura de componentes de un sistema \acrshort{iot} \cite{iotscheme}}
    \label{fig:iot}
\end{figure}

\subsection{IoE}
\label{sec:ioe}

Considerando el contexto energético en el que se engloba este \gls{tfm}, y entrando con más detalle en el marco de las tecnologías \gls{iot}, es preciso destacar dentro de las mismas la subcategoría denominada como el Internet de la Energía (del ingles \gls{ioe}) \cite{ioe}. Como concepto, el \gls{ioe} comprende todo el paradigma de operación de los elementos que constituyen una red energética. Es decir, lleva a cabo la integración de todos los dispositivos, sensores o equipos informáticos en una misma estructura, la cual está basada en internet para poder monitorizar y controlar remotamente el estado de cada punto de la red.

\vspace{3mm}

Se puede expresar que, con la implementación de tecnologías basadas en \gls{ioe}, se buscan objetivos muy similares a las \gls{sg}s. No obstante, el \gls{ioe} va más allá y supone un control y una gestión de la red a mayor escala. En otros términos, no solo se constituye por los elementos de la propia infraestructura eléctrica, sino que también permite una gestión energética a nivel interna de los hogares, mediante la inclusión de electrodomésticos y otros dispositivos electrónicos.




% %%%%%%%%%%%%%%%%%%%%%%%%%%%%%%%%%%%%%%%%%%%%%%%%%%%%%%%%%%%%%%%%%%%%%%%%%%%%%%%%%%%%%%%%%%%%%%%%%%%%%
%%%%%%%%%%%%%%%%%%%%%%%%%%%%%%%%%%%%%%%%%%%%%%%%%%%%%%%%%%%%%%%%%%
\section{DEN2NE}
\label{sec:den2ne}

aqui seria explicar el protocolo, esto en analisis o en estado del arte?

\cite{den2ne}
\cite{gitden2ne}

% %%%%%%%%%%%%%%%%%%%%%%%%%%%%%%%%%%%%%%%%%%%%%%%%%%%%%%%%%%%%%%%%%%%%%%%%%%%%%%%%%%%%%%%%%%%%%%%%%%%%%
%%%%%%%%%%%%%%%%%%%%%%%%%%%%%%%%%%%%%%%%%%%%%%%%%%%%%%%%%%%%%%%%%%
\section{Big Data}
\label{sec:bigdata}


El \textit{Big Data} \cite{stab} se puede definir como el manejo y el análisis de un conjunto extremadamente grande de datos, los cuales provienen de fuentes diferentes y no correlacionadas. Debido a su complejidad y volumen, no pueden ser procesados con software o herramientas tradicionales de gestión de bases de datos y se requieren soluciones tecnológicas más avanzadas. El concepto de \textit{Big Data} se basa en tres componentes principales, denominadas como las 5 Vs \cite{5vs} \cite{5vs2}:

\begin{itemize}
    \item \textbf{Variedad}: Hace referencia a la heterogeneidad de la información. Existen distintos tipos de datos a procesar de diferentes fuentes, formas y resoluciones. Se puede simplificar su gestión si estos son estructurados (en forma de tablas de bases de datos) o dificultar, si estos no tienen una estructura definida (en formato de texto o imagen). También, pueden ser semiestructurados (en formato JSON o XML).
    \item \textbf{Velocidad}: Es un parámetro crítico, ya que algunos entornos requieren de una toma de decisiones a tiempo real y los datos deben de ser generados, procesados y analizados rápidamente. 
    \item \textbf{Volumen}: Es imprescindible tener la capacidad de gestionar una gran cantidad de datos (en el orden de los TB) que además, es creciente en el tiempo. El almacenamiento y la minería de datos deben de ser eficientes para manejarlos de forma correcta 
    \item \textbf{Veracidad}: Mide la calidad y confiabilidad de la información que se adquiere. Se comprueba su autenticidad e integridad para asegurar que proviene de una fuente legítima, por lo que se deben aplicar mecanismos de seguridad como la encriptación o herramientas de detección y mitigación de ataques a la red (ver Sección \ref{sec:seg}). También, se verifica que la información no contenga errores y que sea lo más precisa posible.
    \item \textbf{Valor}: La masividad de datos introduce mucho ruido y confusión en la información y debe de seleccionarse solamente aquella de utilidad, aplicando minería de datos y procesamiento adicional.
\end{itemize}

\vspace{1mm}

Durante los últimos años el \textit{Big Data} se ha adoptado en multitud de sectores y campos como en las telecomunicaciones, las finanzas, el comercio, la medicina, el transporte o la investigación científica. Como se ha expuesto en los apartados anteriores (ver Sección \ref{sec:smartgrids} y, en particular, \ref{sec:consumo}) y, poniendo enfoque en los objetivos de este \gls{tfm} (ver Sección \ref{sec:obj}), el \textit{Big Data} tiene un gran potencial en el ámbito de las \gls{sg}s.

\vspace{3mm}

En este contexto, la efectividad del empleo de la información reside en la búsqueda de la correlación entre los datos adquiridos de la red eléctrica y el resto de características que pueden influir en los mismos, como son el comportamiento de los usuarios finales, las condiciones de estabilidad, la disponibilidad de recursos, el estado de las cargas o las condiciones climáticas en un determinado instante temporal. La coordinación de las mediciones que se realizan en la red junto con la información que se dispone del entorno es crucial para la monitorización, el control y la operativa llevada a cabo dentro de la \gls{sg}~\cite{stab}.

\vspace{3mm}

En otros términos, un análisis efectivo del \textit{Big Data} conducirá a una buena toma de decisiones en el sistema y, en consecuencia, a su optimización. Este análisis se constituye del empleo de herramientas dedicadas al procesamiento distribuido, el almacenamiento en la nube, la minería de datos y a técnicas y algoritmos de aprendizaje automático (\gls{ml}). 

\vspace{3mm}

\subsection{Computación en nube}

En la Figura \ref{fig:bigdata2} se representa la infraestructura de procesos dedicados al \textit{Big Data} en un ámbito de \gls{sg}s. Como se puede visualizar, se compone de tres capas operativas: la superior está enfocada al almacenamiento y computación de datos, la intermedia, a la gestión, compartición e integración de datos de diferentes aplicaciones o fuentes y la inferior, a todos los procesos y técnicas que se encargan del procesamiento, minería, clusterización y clasificación final \cite{stab}.

\begin{figure}[h!]
  \centering
  \includegraphics[width=0.9\textwidth]{img/teoria/bigdata2.png}
  \caption{Infraestructura de \textit{Big Data} enfocada al ámbito de las \acrshort{sg}s \cite{stab}}
  \label{fig:bigdata2}
\end{figure}

Para implementar esta infraestructura, es preciso hacer uso de un entorno de computación en nube. Dentro de este contexto, las grandes tecnológicas como Google, Amazon y Microsoft han puesto sus esfuerzos en los últimos años en desarrollar y optimizar sus propios entornos de nube con el fin de integrar toda la gestión y análisis del \textit{Big Data} en una plataforma. A modo de aportar una mayor comprensión de su funcionamiento, se representa en la Figura \ref{fig:bigdata} un esquema de los servicios de nube disponibles.

\begin{figure}[h!]
  \centering
  \includegraphics[width=1\textwidth]{img/teoria/bigdata.png}
  \caption{Plataforma de análisis de \textit{Big Data} en el entorno de \acrshort{sg}s \cite{bigdata}}
  \label{fig:bigdata}
\end{figure}

\vspace{3mm}

La computación en la nube se basa en el concepto de virtualización, el cual se puede definir como la tecnología que permite la creación de diferentes entornos virtuales aislados a partir de los recursos hardware disponibles y con independencia de este. Los proveedores de servicios emplean la virtualización para crear múltiples \gls{vm} y ofrecer recursos, como pueden ser el almacenamiento, la potencia de procesamiento o las aplicaciones de la nube, como diferentes servicios virtualizados. Esto aporta las siguientes ventajas a la computación en nube \cite{bigdata} \cite{virt}:

\begin{itemize}
  \item \textbf{Escalabilidad}: Se pueden aumentar o reducir recursos de la nube según las necesidades en cada instante temporal. Los usuarios consiguen un autoprovisionamiento de los recursos, ya que pueden desplegar instancias virtuales cuando lo deseen de forma manual o automática.
  \item \textbf{Flexibilidad}: Se ofrecen muchas posibilidades de configuración de los servicios de la nube.
  \item \textbf{Interoperabilidad}: Se emplean Interfaces de Programación de Aplicaciones (del ingles \gls{api}) para integrar plataformas y aplicaciones heterogéneas. El objetivo es soportar la comunicación entre las mismas y que se produzca una interpretación de los datos de forma correcta. Por ello, también se estandarizan los protocolos para que sean comunes. 
  \item \textbf{Seguridad}: Las plataformas de servicios de nube ofrecen un gran nivel de seguridad e implementan medidas como la encriptación de los datos y restricción de acceso o de permisos. Además, el aislamiento de los recursos y de las aplicaciones en diferentes \gls{vm} permite coonfigurar e implementar diferentes políticas de seguridad según las necesidades.
  \item \textbf{Ahorro de costes}: Se implementa un modelo de pago por uso en función de los recursos que se han consumido y se evita la inversión inicial necesaria para la instalación de una infraestructura física, además de su mantenimiento en el tiempo (CAPEX y OPEX).
  \item \textbf{Disponibilidad}: Se garantiza el acceso a los servicios y la confiabilidad de los mismos. Para ello, implementa técnicas de redundancia y aporta una buena toleracia a fallos.
\end{itemize}









\vspace{3mm}









%hablar de mineria de datos -> 3.2 datos inciertos (vacios y tal), 3.4 y 3.5



%stab

%Over the past few years, big data tools have been adopted by
% many major companies in the power industry such as IBM, Siemens,
% General Electric and Oracle (Arenas-Martinez et al., 2010). According to Li et al. (2012), about 85% of the processing tasks of big
% data can be delayed by a day. Hence, even if the energy output is
% time-varying and intermittent, it can be leveraged for processing the
% delayed datasets. Every day, more Terabytes of data emerge at the
% energy data center. Hence, it has become crucial to adopt big data
% technology to extract information from the multiple and diverse data
% sources through novel data centers (Liu et al., 2012). Through BDA, it
% would be possible to understand the behavior of energy consumption,
% which would help to improve energy efficiency and also promote the
% concepts of sustainability (Koseleva & Ropaite, 2017; Marmaras et al.,
% 2017; Mostafa & Negm, 2018).





%%bigdata

% 2.2 Industry perspective
% Even though the information technology related companies have
% achieved substantial success in the field of big data analytics,
% electrical industries are at the beginning stage to deploy big data. A
% few industries including Siemens, GE, ABB, OSI-Soft, and so on
% are developing big data platform and analytics for power grids. An
% account of a few commercially available platforms is provided here
% as a sample only and by no means is intended to be exhaustive.
% Siemens has developed a big data platform, called EnergyIP
% Analytics, which adds big data to smart grid application and
% provides insights on the management of big data for providing
% various grid services to electric utilities and grid operators [40].
% Siemens is currently integrating utility operations and data
% management technologies that could potentially be tapped for grid
% data analytics. This grid analytics platform can allow utilities to
% utilise big data for multiple functionalities, including home energy
% management, grid energy management, and predictive/corrective
% controls [40]. EnergyIP Analytics has already been used by more
% than 50 utilities with a total of 28 million installed smart devices
% [41].
% Similarly, GE has developed an industrial IoT platform, called
% PREDIX, to consolidate data from existing grid management
% systems, smart meters, and grid sensors [42]. In addition, Grid IQ
% Insight, a cloud-based big data analytics architecture, which utilises
% PREDIX platform, is developed to integrate big data analytics to
% grid applications [43]. Native data collected from Grid IQ Insight
% are stored in Data Lake that could be tapped for several grid
% applications [44]. In fact, these initiatives and developments
% support multiple grid applications ranging from real-time grid
% monitoring, distribution automation, home energy management,
% and ancillary services. As illustrated in Fig. 7, a concept of edgcomputing, whereby computational intelligence is connected at theedge of the data source, has been introduced by GE in its Grid IQInsight.
%ABB is integrating cloud computing and big data analytics
% intended for future power grid applications. ABB has developed an
% intelligent big data platform, called ABB Asset Health Center,
% which provides solutions for processing big data for smart grid
% applications [45]. In fact, ABB's Asset Health Center embeds
% equipment monitoring and systems expertise to establish end-toend asset management, business processes for reducing costs,
% minimising risks, improving reliability, and optimising operations
% across the electric utility [45]. In addition, OSI-Soft PI system,
% which is one of the most widely deployed database and analytics
% system, has been contributing to unveil the power of big data
% analytics to electric utilities. Smart asset management platform has
% introduced by OSI-Soft for the purpose of real-time monitoring of
% asset health [46].
% The aforementioned industries are offering utilities a way to
% gain a core understanding of what is the state of grid devices, and
% developing a launching pad for smart grid big data analytics
% applications over time. The next step for the industries is to
% effectively integrate prognosis and diagnosis into big data analytics
% framework so as to facilitate utilities to provide situational
% awareness, informed predictive decisions, condition monitoring,
% health management of critical grid infrastructure, and supporting
% grid functionalities.





















%ver (teoria)%20Smart-grid-based-big-data-analytics-using-machine-learning-and-artificial-intelligence-a-survey.pdf





% %%%%%%%%%%%%%%%%%%%%%%%%%%%%%%%%%%%%%%%%%%%%%%%%%%%%%%%%%%%%%%%%%%%%%%%%%%%%%%%%%%%%%%%%%%%%%%%%%%%%%
%%%%%%%%%%%%%%%%%%%%%%%%%%%%%%%%%%%%%%%%%%%%%%%%%%%%%%%%%%%%%%%%%%
\section{Inteligencia Artificial}
\label{sec:iateoria}

La inteligencia artificial (\gls{ia}) \cite{iagov} \cite{iaazure} se constituye como la disciplina de diseño y creación de sistemas de software o hardware de imitación de la inteligencia humana para la realización de tareas. Generalmente, estos sistemas obtienen la capacidad de aprendizaje, razonamiento, planificación y toma de decisiones a partir del entrenamiento basado en grandes cantidades de información. En otros términos, adquieren e interpretan datos del entorno o de un contexto específico y, teniendo en cuenta un objetivo determinado, procesan la información contenida para decidir la siguiente acción a realizar. Adicionalmente, son capaces de adaptar su comportamiento ante el análisis o detección de cambios en el entorno. 

\vspace{3mm}

La \gls{ia}~\cite{iagov} se postula como una de las tecnologías más revolucionarias y transformadoras de la actualidad y del futuro cercano por su potencial de aplicación en numerosos ámbitos. No obstante, el origen del término se remonta al año 1956, cuando el científico John McCarthy introdujo la idea de crear máquinas inteligentes tomando como base las teorías de computación de los matemáticos Norbert Wiener y John von Neumann en los años 40.

\vspace{3mm}

En la última década, el impulso de la \gls{ia} y los grandes avances que se han producido en este campo vienen dados principalmente por el aumento de las capacidades de los equipos informáticos y el acceso a grandes cantidades de recursos computacionales en las herramientas especializadas e infraestructuras de nube. En otros términos, se posibilita un almacenamiento masivo de datos y, en consecuencia, mejoras significativas en el desarrollo de algoritmos y técnicas, que cada vez se vuelven más complejas y precisas. Además, cabe destacar la contribución que han supuesto las tecnologías \gls{iot} (ver Sección \ref{sec:iot}) en la generación de grandes volúmenes de datos y, por tanto, en el avance de la \gls{ia} al proporcionar información de valor con la que entrenar estos algoritmos.

\vspace{3mm}

Entrando en estas técnicas, se establecen dos divisiones dentro del campo de la \gls{ia} en función de la naturaleza del aprendizaje: el automático (\acrfull{ml}) y el profundo (\acrfull{dl}). En las siguientes Secciones (ver Secciones \ref{sec:ml} y \ref{sec:dl}) se expondrán las características que presentan cada una de estas ramas, además de entrar en profundidad en el funcionamiento de los modelos que se emplearán en el desarrollo de este \gls{tfm}.

\subsection{Machine Learning (\acrshort{ml})}
\label{sec:ml}

La rama de aprendizaje automático (\acrfull{ml}) se enfoca en el desarrollo de modelos que sean capaces de aprender y tomar decisiones en base a la introducción de datos. Durante el proceso se realizan observaciones a la información existente y, en función de la técnica a emplear, se aplican algoritmos estadísticos para identificar patrones en los datos. Por lo tanto, mediante un entrenamiento progresivo en el tiempo, un modelo de \gls{ml} puede llegar a realizar predicciones sobre datos futuros sin la intervención humana y con una alta precisión. Como se representa en la Figura \ref{fig:ml}, las técnicas de \gls{ml} pueden pertenecer a tres categorías diferentes en función del enfoque de la información y de los objetivos que se pretenden conseguir con su aplicación \cite{mlcat} \cite{iageeks} \cite{mltlf}:

\vspace{3mm}

\begin{figure}[h!]
    \centering
    \includegraphics[width=1\textwidth]{img/teoria/ml.jpeg}
    \caption{Categorías de \acrshort{ml} \cite{metal}}
    \label{fig:ml}
\end{figure}

\begin{itemize}
    \item \textbf{Aprendizaje no supervisado}: Se parte de datos sin etiquetar para el entrenamiento de los modelos. Es decir, se parte del conocimiento de unos datos de entrada, pero no de la salida, por lo que el objetivo de los modelos no supervisados se basa en explorar posibles patrones en los datos mediante \textit{clustering}.
    \item \textbf{Aprendizaje supervisado}: Se manejan conjuntos de datos con un conocimiento de las posibles salidas, las cuales se proporcionan en forma de etiquetas o de valores numéricos. Las técnicas de aprendizaje supervisado buscan una función óptima que consiga, dadas unas determinadas variables o características de entrada, predecir la salida con cierta precisión. Son aplicables a dos tipos de problemas: de regresión, si a partir de las variables de entrada se desea predecir un valor numérico continuo a la salida, o de clasificación, si se predefinen diferentes clases y se asignan las variables de entrada a una de ellas. 
    \item \textbf{Aprendizaje por refuerzo}: Las técnicas englobadas en este tipo aprenden por prueba y error, por lo que su entrenamiento a lo largo del tiempo y la experiencia permiten optimizar los resultados sin la necesidad de disponer de un gran volumen de datos.
\end{itemize}

Una vez expuestas las diferentes categorías donde se engloban las técnicas de \gls{ml}, es preciso indicar que se va a centrar el estudio en el \textbf{aprendizaje supervisado} y, en particular, en la resolución de problemas de \textbf{clasificación binaria}. El motivo principal viene dado porque el objetivo que se persigue con la realización de este \gls{tfm} se basa en el desarrollo de modelos que permitan detectar y predecir posibles errores que se pueden producir en una \gls{sg} durante el proceso de distribución energética. 

\vspace{3mm}

Como se detallará más adelante en la Sección \ref{sec:cambiosden2ne}, se creará un conjunto de datos etiquetados en función de la existencia de error o no para entrenar los modelos. Teniendo en cuenta esto, a continuación, se incluyen dos Secciones (ver Secciones \ref{sec:mlsvm} y \ref{sec:mlrf}) para presentar el funcionamiento de las dos técnicas de \gls{ml} que se desarrollarán posteriormente.

\subsubsection{Random Forest (\acrshort{rf})}
\label{sec:mlrf}

El modelo de Bosques Aleatorios (del inglés \acrfull{rf}) está fundamentado en la aplicación de un conjunto de árboles de decisión. Cada uno de estos árboles se definen como estimadores y crecen mediante un proceso de entrenamiento sobre un subconjunto de datos extraído de forma aleatoria del conjunto completo. A la salida del esquema de estimadores se obtiene una predicción final basada en la votación mayoritaria de los árboles. No obstante, es un método que se puede emplear también en problemas de regresión y, en este caso, se proporcionaría un valor numérico promedio, calculado a partir de todas las salidas. En la Figura \ref{fig:rf}, se representa de forma gráfica el diagrama de flujo que expone el funcionamiento del modelo \cite{rfmedium}.

\vspace{3mm}

El proceso de construcción de los árboles de decisión se constituye por operaciones de división que van formando nodos y ramificaciones de forma iterativa. El objetivo es lograr la mayor ganancia de información posible a través de la valoración de la importancia que presentan cada una de las características del conjunto de datos. En otros términos, en cada nodo \textit{m} del árbol se cuantifica la ganancia de información que presenta cada característica y, después, se selecciona la que tenga el mayor valor (\textit{j}). Se aplica una división de los datos existentes en dos subconjuntos homogéneos de menor tamaño \cite{rfmedium2}:

\begin{equation}
    \begin{aligned}
        \theta &= (j, t_m) \\
        Q_m^{\text{left}}(\theta) &= \{(x, y) \mid x_j \leq t_m\} \\
        Q_m^{\text{right}}(\theta) &= Q_m \setminus Q_m^{\text{left}}(\theta)
    \end{aligned}
\end{equation}
    

Donde:
\begin{itemize}
    \renewcommand{\labelitemi}{}
    \item \( \theta \) es la operación de división en un nodo \textit{m}.
    \item \( j\) es la característica seleccionada.
    \item \( t_m\) es el nivel de partición de los datos.
    \item \( Q_m^{left}\) es el subconjunto de datos en la ramificación izquierda.
    \item \( Q_m^{right}\) es el subconjunto de datos en la ramificación derecha.
\end{itemize}

\vspace{3mm}

\begin{figure}[h!]
    \centering
    \includegraphics[width=0.95\textwidth]{img/teoria/rf.png}
    \caption{Modelo \acrshort{rf} \cite{rfmedium}}
    \label{fig:rf}
\end{figure}

\vspace{3mm}

Este proceso se va repitiendo en todos los nodos hasta llegar al final del árbol y, en cuanto al cálculo de la ganancia de información, es preciso introducir el concepto de impureza. Este se define como el procedimiento de evaluación de la calidad de las divisiones que se producen en los nodos. En otros términos, se puede expresar que indica el nivel de ruido que hay en los datos de los subconjuntos creados mediante una función de impureza \textit{H()}:

\vspace{3mm}

\begin{equation}
    \begin{aligned}
        G(Q_m, \theta) = \frac{n_m^{left}}{n_m} H(Q_m^{left}(\theta)) + \frac{n_m^{right}}{n_m} H(Q_m^{right}(\theta))
    \end{aligned}
\end{equation}

\vspace{3mm}

Tomando esto en consideración, se pueden emplear dos funciones o métodos de medición de la impureza:~\cite{rfmedium2} \cite{scikitrf}

\begin{itemize}
    \item Entropía: Su objetivo está orientado a establecer divisiones de los datos de forma que la entropía en los nodos inferiores o hijos sea menor que la del nodo superior o padre. Para ello, se aplica la teoría de la entropía de Shannon \cite{rfmedium2} para escoger las características que minimicen la entropía y, en consecuencia, la incertidumbre y la función de pérdidas. De forma contraria, en un nodo se obtiene una entropía máxima cuando las clases están representadas homogéneamente en el conjunto de datos.
    
    \begin{equation}
        \begin{aligned}
            H(Q_m) = - \sum_k p_{mk} \log(p_{mk})
        \end{aligned}
    \end{equation}
    
    \item Gini: La impureza es definida a partir del cálculo de la probabilidad de una característica determinada que es clasificada erróneamente cuando se selecciona aleatoriamente. Un índice Gini nulo expresa una clasificación pura donde todos los datos corresponden a una clase específica, en cambio, un índice Gini igual a 1, representa una distribución aleatoria de los datos.
    
    \begin{equation}
        \begin{aligned}
            H(Q_m) = \sum_k p_{mk} (1 - p_{mk})
        \end{aligned}
    \end{equation}
\end{itemize}

Donde:
\begin{itemize}
    \renewcommand{\labelitemi}{}
    \item \(p_{mk}\) es la proporción de datos de entrenamiento que pertenecen a una clase determinada.
    \item \(H(Q_m)\) es la función de impureza.
\end{itemize}


\subsubsection{Support Vector Machines (\acrshort{svm})}
\label{sec:mlsvm}

El modelo de Máquinas de Vector Soporte (del inglés \acrfull{svm}) \cite{svmmedium2} \cite{svmciencia} es una técnica que se emplea tanto en problemas de regresión como de clasificación. Se basa en el concepto del \textit{Maximal Margin Classifier} o \textit{Hard Margin Classifier}, puesto que tiene el fin principal de encontrar el hiperplano óptimo que consiga una máxima separación entre dos clases diferentes dentro de un espacio de características. Esta separación se define como margen y se mide como la distancia perpendicular que existe desde el hiperplano hacia los vectores de soporte, que son las muestras más cercanas a la frontera de separación de clases. En la Figura \ref{fig:svm}, se representa el cálculo del mejor hiperplano entre dos clases de datos.

\vspace{3mm}

Por ello, se puede expresar que los vectores de soporte son los puntos más críticos del plano para asignar una de las clases predefinidas a los nuevos vectores de datos a la entrada. Este proceso de clasificación viene dado por la siguiente expresión: \cite{svmmedium} 

\begin{equation}
    \begin{aligned}
        y_i(\mathbf{w} \cdot \mathbf{x_i} + b) \geq M
    \end{aligned}
\end{equation} 

    Donde:
\begin{itemize}
    \renewcommand{\labelitemi}{}
    \item \(y_i\) es la etiqueta de la clase. 
    \item \(x_i\) es el vector de características.
    \item \(w\) es el vector de pesos o coeficientes del hiperplano.
    \item \(b\) es el sesgo ajustado en el modelo.
    \item \(M\) es el ancho definido para el margen.
\end{itemize}

\vspace{3mm}

\begin{figure}[h!]
    \centering
    \includegraphics[width=0.65\textwidth]{img/teoria/svm.png}
    \caption{Cuantificación del hiperplano óptimo entre dos clases en el \acrshort{svm} \cite{svmmedium2}}
    \label{fig:svm}
\end{figure}

\vspace{3mm}

No obstante, en la práctica es complicado determinar un hiperplano si no se trata de un problema ideal y basado en datos linearmente separables \cite{matlab} \cite{svmciencia}. Generalmente, en los casos reales si se intenta forzar un máximo margen, se pueden llegar a producir problemas de sobreentrenamiento, ya que nuevos datos pueden suponer grandes variaciones en el hiperplano. En la Figura \ref{fig:svmerror}, se representa la opción alternativa que se lleva a cabo en la mayoría de casos, basada en obtener un margen máximo, permitiendo cierto grado de error (\textit{Soft Margin Classifier}). 

\vspace{3mm}

Esto tiene como consecuencia que existan una serie de vectores clasificados de forma errónea en el plano. En este proceso se introduce el concepto de variables de holgura (\textit{slack variables}) a partir de la siguiente expresión:

\begin{equation}
    \begin{aligned}
        \xi_i = \max(0, 1 - y_i (\mathbf{w} \cdot \mathbf{x}_i + b))
    \end{aligned}
\end{equation} 

\vspace{3mm}

\begin{figure}[h!]
    \centering
    \includegraphics[width=0.65\textwidth]{img/teoria/svm2.png}
    \caption{Representación de vectores clasificados erróneamente dentro de los márgenes establecidos en el \acrshort{svm} \cite{svmmedium}}
    \label{fig:svmerror}
\end{figure}

\vspace{3mm}

Estas variables determinan si un vector está correctamente clasificado o si se encuentra en una zona errónea del margen o del hiperplano, respectivamente. En el caso de posición incorrecta, el valor de la variable define la distancia al margen o el error cometido:

\begin{equation}
    \begin{aligned}
        \xi_i = 0 & \text{ si } 1 - y_i (\mathbf{w} \cdot \mathbf{x}_i + b) \leq 0 \\
        0 \text{<} \xi_i \text{<} 1 & \text{ si } 0 < 1 - y_i (\mathbf{w} \cdot \mathbf{x}_i + b) \leq 1 \\
        \xi_i \text{>} 1 & \text{ si } 1 - y_i (\mathbf{w} \cdot \mathbf{x}_i + b) > 1 \\
    \end{aligned}
\end{equation} 

\pagebreak

El sumatorio de todas las variables de holgura que existen en el plano vienen incluidas en el objetivo de la función de pérdida del modelo para que el error de clasificación sea mínimo, a la vez que el margen es máximo. Como es preciso encontrar un cierto equilibrio entre ambos, el \gls{svm} se determina como una técnica de optimización convexa que se fundamenta en el hiperparámetro \textit{C}. En la Figura \ref{fig:parametroc}, se representa de forma gráfica cómo su valor establece el control de forma inversamente proporcional de la cantidad de muestras clasificadas erróneamente, por lo que se define como el parámetro de regularización del modelo. Es decir, cuanto más alto sea su valor, menores violaciones del margen y del hiperplano serán permitidas y más se enfocará el modelo en un \textit{Maximal Margin Classifier}~\cite{svmciencia}.

\vspace{3mm}

\begin{figure}[h!]
    \centering
    \includegraphics[width=1\textwidth]{img/teoria/parametroc.png}
    \caption{Configuración del parámetro de regularización \textit{C} \cite{velocity}}
    \label{fig:parametroc}
\end{figure}

\vspace{3mm}

Además de permitir cierto grado de error, como se ha expuesto anteriormente, para enfrentar problemas con datos no linearmente separables, es imprescindible incrementar las dimensiones del espacio original de características. En este caso, se introduce el concepto de kernel como función para obtener un nuevo espacio dimensional diferente al original donde exista una mayor probabilidad de que los datos sí sean linearmente separables. La aplicación de los diferentes tipos de kernels que permiten las \gls{svm} se caracteriza a partir de las expresiones expuestas a continuación~\cite{svmciencia}~\cite{velocity}:

\begin{equation}
    \begin{aligned}
        \text{Kernel lineal: } K(\mathbf{x}_i, \mathbf{x}_j) = \mathbf{x}_i \cdot \mathbf{x}_j \\
        \text{Kernel polinómico: } K(\mathbf{x}_i, \mathbf{x}_j) = (\gamma \mathbf{x}_i \cdot \mathbf{x}_j + r)^d \\
        \text{Kernel radial o gaussiano: } K(\mathbf{x}_i, \mathbf{x}_j) = \exp \left( -\gamma \| \mathbf{x}_i - \mathbf{x}_j \|^2 \right) \\
        \text{Kernel sigmoid: } K(\mathbf{x}_i, \mathbf{x}_j) = \tanh(\gamma \mathbf{x}_i \cdot \mathbf{x}_j + r)
    \end{aligned}
\end{equation}


\vspace{3mm}

Para obtener una mayor compresión, se representa de forma adicional la Figura \ref{fig:rbf}. En la misma se puede apreciar gráficamente cómo se produce la transformación dimensional de las características, especificando para el caso del kernel gaussiano o \gls{rbf}.

\vspace{3mm}

\begin{figure}[H]
    \centering
    \includegraphics[width=0.9\textwidth]{img/teoria/rbf.png}
    \caption{Aplicación del kernel \acrshort{rbf} \cite{rbf}}
    \label{fig:rbf}
\end{figure}

\subsection{Deep Learning (\acrshort{dl})}
\label{sec:dl}

El campo del aprendizaje profundo (\acrfull{dl}) se enfoca principalmente en la creación de redes neuronales que imiten el comportamiento y la estructura lógica que tiene el cerebro humano para buscar patrones en los datos. A diferencia de las técnicas de \gls{ml}, el \gls{dl} se vuelve mucho más eficiente en el análisis de grandes volúmenes de datos, puesto que generalmente el \gls{ml} presenta ciertas limitaciones \cite{metal} en este aspecto.

\vspace{3mm}

De la misma forma, el \gls{dl} presenta un mejor funcionamiento en la identificación de patrones cuando se manejan características complejas o un gran número de ellas, aportando resultados con una mayor precisión que el \gls{ml}. Es por ello que las técnicas de \gls{dl} cobran una gran importancia en entornos donde los datos no son estructurados, como se produce en el caso de las imágenes, texto y audio. En este caso, se pueden introducir en aplicaciones de reconocimiento de voz, procesado de lenguaje natural (del inglés \gls{nlp}) o visión artificial para la detección de objetos y clasificación de imágenes. \cite{iageeks}

\vspace{3mm}

De la misma forma, a nivel estructural, las diferencias entre las técnicas de \gls{ml} y las de \gls{dl} radican principalmente en el tratamiento de las características o variables de los datos. En el caso de emplear \gls{ml}, si se manejan conjuntos con un gran número de características, sería necesario introducir un paso previo de diseño y selección de las más relevantes. Esto, como se puede apreciar en la Figura \ref{fig:features}, no ocurre en el desarrollo de un modelo de \gls{dl}, ya que el tratamiento se produce internamente dentro de la red reuronal \cite{valohai}.

\vspace{3mm}

Una vez presentada la rama de aprendizaje profundo, se introduce una Sección dedicada a la exposición del modelo de \gls{dl} que se desarrollará para lograr los objetivos de este \gls{tfm} (ver Sección \ref{sec:dlann}). 

\vspace{3mm}

\begin{figure}[h!]
    \centering
    \includegraphics[width=0.8\textwidth]{img/teoria/mlvsdl.png}
    \caption{Diferencias estructurales entre el \acrshort{ml} y el \acrshort{dl} \cite{valohai}}
    \label{fig:features}
\end{figure}

\subsubsection{Red Neuronal Artificial (\acrshort{ann}) y Perceptrón Multicapa (\acrshort{mlp})}
\label{sec:dlann}

Las redes neuronales artificiales (\gls{ann}) y, en particular, los perceptrones multicapa (del inglés \gls{mlp}) \cite{ibmann}, se construyen a partir de distintas capas de nodos y conexiones que replican la estructura neuronal que tiene el cerebro humano. Como se representa en la Figura \ref{fig:ann}, se define una capa de entrada (\textit{input layer}), una o múltiples capas ocultas (\textit{hidden layer}) y una capa de salida (\textit{output layer}). El entrenamiento de la arquitectura de neuronas viene dado por las conexiones que se van estableciendo entre los nodos de las capas y los pesos y los valores de umbral que se van especificando durante el proceso. 

\vspace{3mm}

En otros términos, se toma cada nodo como una aplicación del modelo de regresión lineal, donde se parte de unos datos en la capa de entrada, que son ponderados con unos determinados pesos en cada capa oculta de la siguiente forma:

\begin{equation}
    \begin{aligned}
        z = \mathbf{w}^T \cdot \mathbf{x} + b
    \end{aligned}
\end{equation} 

Donde:
\begin{itemize}
    \renewcommand{\labelitemi}{}
    \item \(z\) es la salida de cada capa oculta. 
    \item \(w\) es el vector de pesos. 
    \item \(b\) es el sesgo.
    \item \(x\) es el vector de entrada
\end{itemize}

\vspace{3mm}

\begin{figure}[h!]
    \centering
    \includegraphics[width=0.75\textwidth]{img/teoria/ann.jpg}
    \caption{Arquitectura de un \acrshort{mlp} \cite{ann}}
    \label{fig:ann}
\end{figure}

\vspace{3mm}

De la misma forma, en cada capa oculta se aplica una función de activación o de umbral, que impone un valor límite a la salida final del nodo en cuestión y, en consecuencia, a la entrada del nodo siguiente al que se encuentre conectado. Las funciones de activación principales que se pueden emplear se representan en la Figura \ref{fig:functions} y vienen dadas por las siguientes expresiones \cite{factiv} \cite{functions}:

\begin{equation}
    \begin{aligned}
        \text{Función Sigmoidal (Logística):} \quad f(x) = \frac{1}{1 + e^{-x}} \\
        \text{Función ReLU (Rectified Linear Unit):} \quad f(x) = \max(0, x) \\
        \text{Función tanh (Tangente Hiperbólica):} \quad f(x) = \tanh(x) \\
        \text{Función Softmax:} \quad \text{softmax}(x_i) = \frac{e^{x_i}}{\sum_{j} e^{x_j}} \\
    \end{aligned}
\end{equation}

\pagebreak

Teniendo todo el procedimiento anterior en cuenta, se puede expresar que, a partir de una serie de características, el objetivo es escoger aquellas que contribuyan a obtener una mayor precisión en la clasificación de datos de entrada futuros, puesto que se van estableciendo las conexiones óptimas entre los nodos de las capas ocultas con un entrenamiento progresivo. Es preciso indicar que el funcionamiento del modelo se basa en una red de propagación hacia delante, lo que supone que se defina un flujo unidireccional, desde la entrada a la salida. 

\vspace{3mm}

\begin{figure}[h!]
    \centering
    \includegraphics[width=0.75\textwidth]{img/teoria/functions.png}
    \caption{Funciones de activación \cite{functions}}
    \label{fig:functions}
\end{figure}

\vspace{3mm}

No obstante, para ajustar las ponderaciones se aporta una retroalimentación, que depende directamente de la función de coste y de la salida del modelo en cada ejecución o \textit{epoch}. Principalmente, se pueden utilizar dos métodos para minimizar la función de coste: el algoritmo adaptativo \textit{adam} (del inglés \textit{Adaptive Moment Estimation}), que presenta una tasa de aprendizaje adaptativa para cada parámetro, y el gradiente descendiente (del inglés \gls{sgd}), que actualiza los pesos de la red en función del gradiente de la función de pérdida con respecto a los pesos \cite{ibmann}.

\pagebreak

% %%%%%%%%%%%%%%%%%%%%%%%%%%%%%%%%%%%%%%%%%%%%%%%%%%%%%%%%%%%%%%%%%%%%%%%%%%%%%%%%%%%%%%%%%%%%%%%%%%%%%
%%%%%%%%%%%%%%%%%%%%%%%%%%%%%%%%%%%%%%%%%%%%%%%%%%%%%%%%%%%%%%%%%%
\section{Herramientas software}
\label{sec:software}

\subsection{BRITE}

\gls{brite} \cite{brite} se presenta como una plataforma dedicada a la generación de topologías de red. Fue desarrollada en la Universidad de Boston y se caracteriza por su gran flexibilidad, ya que soporta múltiples modelos de topología. Cada uno de estos modelos viene determinado por los parámetros de entrada que se definen. Es por ello que \gls{brite} sigue la siguiente secuencia de acciones para diseñar las topologías:

\begin{enumerate}
    \item Definición del posicionamiento de los nodos:
    \item Nivel de Interconexión de los nodos y configuración de enlaces:
    \item Asignación de atributos y características de la red y de los dispositivos: Se determina el delay y el ancho de banda de los enlaces.
    \item Especificación del formato de salida: Se indica un formato .brite para todas las topologías generadas.
\end{enumerate}

\footnote{https://github.com/NETSERV-UAH/BRITE}








% 6. Generador_brite.py:
% -----------------------
% Como se necesita crear un archivo inicial con la configuración de la topología he creado un programa que los crea automaticamente
% Este recibe por linea de comandos los siguientes parámetros:
% 	+Nombre del archivo de configuración que vamos a generar
% 	+Modelo de la topología (Solo da soporte a los modelos RTWaxman = 1, ASWaxman = 3, RTBarbasi = 2 y ASBarbasi)
% 	+Numero de nodos 
% 	+NodePlacement
% 	+GrowthTYpe (solo para Waxman)
% 	+Numero de enlaces por nodo (m)
% 	+Alpha (para Waxman)
% 	+Beta (para Waxman)


% 7. Autogenerador.sh
% --------------------
% Para automatizar la generación de los archivos de configuración y los archivos brite propiamente dichos, se ha creado un script "autogenerador.sh"
% En este script podemos ver que los parametros de entrada que hemos explicado en este documento están declarados al principio con los valores que pueden tomar para crear la topología que deseamos.
% Después ejecutamos el generador_brite.py con los parámetros de entrada correspondientes.
% Y por último, para cada archivo de configuración creado (determina una configuración de topologias brite), creamos 10 escenarios brite que siendo iguales en parámetros brite(numero de nodos, m, nodePlcement...) son escenarios distintos por el donde están posicionados cada nodo, o las conexiones entre nodos.



\subsubsection{Parser}

Función que transforma la salida de BRITE en dos archivos, Nodos.txt y Enlaces.txt para ser más legible




