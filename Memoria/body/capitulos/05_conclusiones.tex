\chapter{Conclusiones y trabajo futuro}
\label{conclusiones}

En este Capítulo final, se establecerán las conclusiones principales que han surgido de la realización de este proyecto realizado y se valorará el cumplimiento de los objetivos expuestos inicialmente en el Capítulo \ref{ch:intro}. De la misma forma, se explorarán las posibles vías de trabajo futuro que se presentan como opciones de continuación de este \gls{tfm}.


\section{Conclusiones del \glsentryshort{tfm}}

En este \gls{tfm}, se han desarrollado diferentes modelos de \gls{ml} y \gls{dl} para la detección y predicción precisa de errores en \gls{sg}s durante el proceso de distribución energética que lleva a cabo el algoritmo \gls{den2ne}. 

\vspace{3mm}

Este proceso se ha estructurado en seis etapas, comenzando por el estudio del estado del arte y la investigación de las posibles fuentes de datos de implementaciones \gls{sg}s reales a emplear. Por consiguiente, ha sido necesario analizar y evaluar el volumen de información útil aplicando un procesamiento exhaustivo sobre los datos reales. Este paso ha sido fundamental y, aunque ha supuesto un análisis más extenso y detallado de lo estimado inicialmente, se ha logrado un conjunto de datos reales coherente, riguroso y simplificado que abarque muestras tomadas por hora durante un intervalo de un año.

\vspace{3mm}

No obstante, después de este procesamiento, se han requerido dos pasos adicionales para extraer el conjunto de datos final sobre el que desarrollar y entrenar los modelos. Por un lado, se ha aplicado la herramienta BRITE para el planteamiento y generación de topologías de red, lo que ha derivado en la creación de 180 en total. Estas topologías se han importado a un simulador propietario realizado en python y han supuesto la base de las simulaciones de distribución energética en \gls{den2ne}. 

\vspace{3mm}

De la misma forma, también ha sido necesario modelar y realizar una serie de modificaciones en la estructura y en el funcionamiento del algoritmo. Este proceso ha sido imprescindible para permitir la importación al mismo de los datos reales, ejecutar múltiples simulaciones de las topologías generadas en el paso anterior y exportar el conjunto de datos final con las instancias ya etiquetadas. Cabe destacar que, para obtener suficientes muestras útiles o, en otros términos, un número significante de simulaciones a ejecutar en el algoritmo, se han seleccionado una serie de instantes temporales sobre los que trabajar. Entonces, en base a la configuración propuesta, se han realizado un total de 47.304.000 simulaciones en \gls{den2ne}.

\vspace{3mm}

El siguiente paso se ha correspondido con el desarrollo y entrenamiento de los diferentes modelos de \gls{ml} y \gls{dl} a partir de los patrones de error etiquetados y exportados del algoritmo. En este caso, se han planteado, diseñado y probado tres técnicas diferentes: \gls{rf}, \gls{svm} y \gls{ann}s. En este caso, el proceso de optimización de los hiperparámetros característicos de cada una de ellas y la aplicación de diferentes métodos de reducción de las dimensiones del conjunto de datos han derivado en el desarrollo de múltiples opciones a probar. Además, la evaluación del rendimiento de las mismas ha sido crucial para llevar a cabo un análisis completo del rendimiento de los modelos y para definir el caso óptimo que permita detectar los errores de forma precisa. En este caso, se puede expresar que el modelo que ha proporcionado el mejor rendimiento en términos globales es el \gls{rf} con la aplicación del método \gls{rfecv}.

\vspace{3mm}

Como síntesis, los resultados obtenidos del desarrollo de este \gls{tfm} han sido fruto de realizar un trabajo amplio y riguroso, tanto en el estudio, análisis y procesamiento de los datos, como en la creación de modelos de detección y predicción de errores con una alta efectividad. Este trabajo contribuye significativamente a la mejora del funcionamiento del algoritmo \gls{den2ne} y a la eficiencia y la fiabilidad de las \gls{sg}s en el contexto de la distribución energética. Adicionalmente, la adaptabilidad de los modelos se extiende al contexto de implementación en entornos de redes de dispositivos \gls{iot}, lo que destaca la notable flexibilidad e importancia que aporta este proyecto.

\vspace{3mm}

%Personalmente, este \gls{tfm} ha sido un reto, por su complejidad a nivel técnico, y por lo transversal que es. Se ha estudiado las bondades y las tecnologías habilitantes del \gls{6g}, el entorno radio, las nuevas configuraciones de \textit{arrays} de antenas, las arquitecturas propuestas en la nueva generación. Asimismo, se ha ido trabajando herramienta por herramienta del mundo \gls{sdn}, se ha programado desde alto nivel (Python) a bajo nivel (BOFUSS), se ha trabajado desde el espacio de usuario a espacio de kernel, incluso se ha tenido que diseñar una plataforma de emulación inalámbrica. Por todo ello, por el carácter polímata y transversal del proyecto, creo que este \gls{tfm} ilustra bastante bien las competencias y habilidades que un Ingeniero de Telecomunicación debería tener, que es por ello, por lo que somos reconocidos tanto en la industria, como en la sociedad.




%----

% \\
% En la primera fase del proyecto, se realizó un exhaustivo estudio y documentación de las principales herramientas y tecnologías relacionadas, sentando así una sólida base teórica antes de abordar el diseño y prototipado del protocolo de control in-band. Posteriormente, se analizaron las necesidades y características de las distintas tecnologías para seleccionar las herramientas más adecuadas para la implementación del protocolo de control. Se llevó a cabo un estudio de estas herramientas con el objetivo de lograr una implementación optimizada en la medida de lo posible.  El proyecto se completó mediante la validación del protocolo desarrollado a través de la emulación, lo que permitió realizar pruebas de funcionamiento del protocolo implementado en el entorno \gls{bofus} sobre interfaces inalámbricas emuladas. De este modo, se pudo comprobar el correcto desempeño del protocolo en diferentes casos de uso. Esta validación resulta fundamental para asegurar que el protocolo cumple con los requisitos de escalabilidad y control en entornos \gls{iot} y \gls{sdn}. De forma adicional, se ha estudiado como validarlo en un entorno real, no siendo posible, dado que hay problemas en la traducción de símbolos a la arquitectura ARM.\\
% \\
% Durante todo el proceso de ejecución del proyecto, se generaron diversas contribuciones no asociadas directamente a los objetivos del proyecto, pero sí a la comunidad \gls{sdn}. Se aportó claridad al documentar el funcionamiento interno de herramientas como ``dpctl" y su interfaz de comandos, lo que permitió resolver problemas relacionados con ellas. Se exploró a fondo el funcionamiento de Mininet-WiFi, documentando su jerarquía de clases y los módulos que utiliza en el kernel de Linux para emular el entorno de radio, lo que facilitó la identificación de posibles fuentes de errores debido a desvíos en las interfaces inalámbricas emuladas. También se profundizó en el funcionamiento interno de \gls{bofus}, solucionando problemas de compilación para las últimas versiones de distribuciones Linux y actualizando documentación incorrecta sobre su funcionamiento.\\
% \\
% Además, se crearon escenarios replicables mediante Vagrant, que incluyen todas las herramientas relacionadas con entornos \gls{sdn} (Mininet, Mininet-WiFi, Ryu, \gls{onos}, \gls{bofus}) en la última versión de Ubuntu 22.04. Esto ha permitido mantener actualizadas estas herramientas y proyectos que habían sido abandonados. Por ejemplo, el proyecto \gls{bofus} fue retomado después de casi cuatro años sin recibir ninguna contribución (\textit{commit}), gracias a la colaboración entre el autor y el autor de este \gls{tfm}, quien asumió un rol de mantenedor en el proyecto.\\
% \\
% En conclusión, este \gls{tfm} ha logrado desarrollar un mecanismo de control in-band para dispositivos \gls{iot} en entornos \gls{6g}. Además, se ha generado una cadena de valor al aportar conocimiento y mejoras a herramientas y proyectos previamente abandonados, lo que ha contribuido a revitalizarlos.\\
% \\
% Personalmente, este proyecto ha sido un reto, por su complejidad a nivel técnico, y por lo transversal que es. Se ha estudiado las bondades y las tecnologías habilitantes del \gls{6g}, el entorno radio, las nuevas configuraciones de \textit{arrays} de antenas, las arquitecturas propuestas en la nueva generación. Asimismo, se ha ido trabajando herramienta por herramienta del mundo \gls{sdn}, se ha programado desde alto nivel (Python) a bajo nivel (BOFUSS), se ha trabajado desde el espacio de usuario a espacio de kernel, incluso se ha tenido que diseñar una plataforma de emulación inalámbrica. Por todo ello, por el carácter polímata y transversal del proyecto, creo que este \gls{tfm} ilustra bastante bien las competencias y habilidades que un Ingeniero de Telecomunicación debería tener, que es por ello, por lo que somos reconocidos tanto en la industria, como en la sociedad.

% \newpage


\section{Líneas de trabajo futuro}
\label{trabajoFuturo}


%extender el estudio hacia otros algoritmos de encaminamiento tipo den2ne o a otras ubicaciones (aqui madeira) para estudiar comportamientos de consumo
%extender el analisis a otros ml o ann
%extender a estudio de balance energetico o comportamiento de consumo



% Tras concluir y revisar el alcance conseguido de los objetivos propuestos para la ejecución del proyecto, podemos apuntar a próximas líneas de trabajo futuro basadas en los siguientes puntos:
% \begin{itemize}
%     \item Realizar un despliegue en real utilizando Raspberry Pi (RPi) que tienen una interfaz WiFi nativa para evaluar el funcionamiento del win-BOFUSS con varias RPis. Esto permitirá verificar si el rendimiento es similar al observado durante el proyecto.

%     \item Realizar la integración del switch BOFUSS en la jerarquía de clases de Mininet-WiFi. Esto facilitará la realización de pruebas y evaluaciones más consistentes al combinar las funcionalidades de ambos proyectos.

%     \item Añadir seguridad mediante el uso de TLS en las conexiones OpenFlow in-band desde el UserAP hacia el controlador. Esto garantizará la confidencialidad y la integridad de las comunicaciones, protegiendo la información transmitida entre el UserAP y el controlador.

%     \item Realizar simulaciones de pérdida utilizando el protocolo desarrollado y evaluar cómo afecta al funcionamiento intrínseco de la operativa programada. Esto ayudará a comprender el impacto de la pérdida de paquetes en el rendimiento del sistema y permitirá optimizar el protocolo en función de los resultados obtenidos.

%     \item Modificar la interfaz del \gls{bofus} con la herramienta de gestión ``dpctl'' para que implemente una interfaz estándar como JSON.

%     \item Arreglar el trazado de símbolos a la arquitectura ARM. Este punto tiene que ver con el primero a líneas a futuro, porque si no se porta correctamente a ARM según se ha explicado, será imposible desplegarlo en una RPi, o en dispositivos ARM que porten Openwrt, por ejemplo.
% \end{itemize}

% Estas líneas de trabajo futuro proporcionan oportunidades para mejorar y ampliar el proyecto, explorando nuevas funcionalidades, mejorando la seguridad y evaluando su rendimiento en diferentes escenarios y condiciones. Como se ha indicado en las conclusiones, aparte de apuntar a las posibles líneas de trabajo a futuro, también hay que mencionar la realidad inmediata, que después de estar trabajando con el \textit{software switch} BOFUSS, conseguir hacerlo funcionar en las últimas versiones se tomará el relevo a Eder como mantenedor del proyecto en una nueva organización agnóstica.


% %%%%%%%%%%%%%%%%%%%%%%%%%%%%%%%%%%%%%%%%%%%%%%%%%%%%%%%%%%%%%%%%%%%%%%%%%%%%%%%%%%%%%%%%%%%%%%%%%
