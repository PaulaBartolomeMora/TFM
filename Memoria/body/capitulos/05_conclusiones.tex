\chapter{Conclusiones y trabajo futuro}
\label{conclusiones}

En este Capítulo final, se establecerán las conclusiones principales a las que se ha llegado con la realización de este proyecto de investigación y se valorará el cumplimiento de los objetivos expuestos inicialmente en el Capítulo \ref{ch:intro}. De la misma forma, se explorarán las posibles vías de trabajo futuro que se presentan como opciones de continuación de este \gls{tfm}.

\section{Conclusiones del \glsentryshort{tfm}}

En este \gls{tfm}, se han desarrollado diferentes modelos de \gls{ml} y \gls{dl} para la detección y predicción precisa de errores en \gls{sg}s durante el proceso de distribución energética que lleva a cabo el algoritmo \gls{den2ne}. 

\vspace{3mm}

Este proceso se ha estructurado en seis etapas, comenzando por el estudio del estado del arte y la investigación de las posibles fuentes de datos de implementaciones \gls{sg}s reales a emplear. Por consiguiente, ha sido necesario evaluar el volumen de información útil de los datos reales, provenientes de la implementación de una \gls{sg} en la ciudad de Funchal, Madeira. Cabe destacar que se ha aplicado un procesamiento exhaustivo, además de que ha sido necesario incorporar un análisis en profundidad de dos herramientas (\textit{Global Solar Atlas} y \textit{PVWatts}) para simular los datos de producción. Este paso de tratamiento de datos ha sido fundamental y, aunque ha supuesto una mayor extensión en el tiempo de lo estimado inicialmente, se ha logrado un conjunto de datos reales coherente, riguroso y simplificado que abarque muestras tomadas por hora durante un intervalo de un año.

\vspace{3mm}

No obstante, después de este procesamiento, se han requerido dos pasos adicionales para extraer el conjunto de datos final sobre el que desarrollar y entrenar los modelos. Por un lado, se ha aplicado la herramienta BRITE para el planteamiento y generación de topologías de red, lo que ha derivado en la creación de 180 en total. Estas topologías se han importado a un simulador propietario realizado en python y han supuesto la base de las simulaciones de distribución energética en \gls{den2ne}. 

\pagebreak

Por otro lado, también ha sido necesario modelar y realizar una serie de modificaciones en la estructura y en el funcionamiento del algoritmo. Este proceso ha sido imprescindible para permitir la importación al mismo de los datos reales, ejecutar múltiples simulaciones de las topologías generadas en el paso anterior y exportar el conjunto de datos final con las instancias ya etiquetadas. En base a la configuración de los parámetros de \gls{den2ne} se ha establecido la posibilidad de realizar hasta 47.304.000 simulaciones, teniendo en cuenta todo el rango temporal de medidas, que como se ha comentado es de un año. En el caso de este \gls{tfm}, para obtener suficientes muestras útiles y variadas, se han seleccionado 12 instantes temporales sobre los que trabajar. Esta selección de datos se ha aplicado a la entrada de \gls{den2ne}, derivando a su salida en un conjunto de datos final compuesto por un total de 1.931.040 instancias etiquetadas. 

\vspace{3mm}

El siguiente paso ha consistido en el desarrollo y entrenamiento de los diferentes modelos de \gls{ml} y \gls{dl}, a partir de los patrones de error etiquetados y exportados del algoritmo anteriormente. En este caso, se han planteado, diseñado y probado tres técnicas diferentes: \gls{rf}, \gls{svm} y \gls{ann}s, siendo la última desarrollada en dos librerías diferentes (\textit{sklearn} y \textit{keras}), a modo comparativo. El proceso de optimización de los hiperparámetros característicos de cada una de ellas y la aplicación de diferentes métodos de reducción de las dimensiones del conjunto de datos, en el caso del \gls{rf} y \gls{svm}, han derivado en el desarrollo de múltiples opciones a probar. 

\vspace{3mm}

De la misma forma, la evaluación del desempeño de los modelos ha sido fundamental para realizar un análisis exhaustivo de su rendimiento y para definir el caso óptimo que permita detectar los errores de forma precisa. A modo de sintetizar los resultados finales del desarrollo de este \gls{tfm}, y de observar las métricas obtenidas, se representa la Tabla \ref{tab:resumen}. A partir de la misma, se determina de forma concluyente que el modelo que proporciona el mejor rendimiento en términos globales es el \gls{rf} con la aplicación del método \gls{rfecv}.

\vspace{3mm}

Finalmente, tomando en consideración todo el trabajo realizado, se puede expresar como conclusión que este \gls{tfm} ha supuesto un gran reto por su complejidad a nivel técnico y por su extensión temporal requerida. Los resultados obtenidos del desarrollo de este \gls{tfm} han sido fruto de realizar un trabajo amplio y riguroso, tanto en el estudio, análisis y procesamiento de los datos, como en la creación de modelos de detección y predicción de errores con una alta efectividad. Por ello, este trabajo contribuye significativamente a la mejora del funcionamiento del algoritmo \gls{den2ne} y a la eficiencia y la fiabilidad de las \gls{sg}s en el contexto de la distribución energética. Adicionalmente, la adaptabilidad de los modelos se extiende al contexto de implementación en entornos de redes de dispositivos \gls{iot}, lo que incrementa la importancia de este proyecto y destaca su notable flexibilidad.

\begin{sidewaystable}
    \centering
    \begin{tabular}{
    >{\columncolor[HTML]{EFEFEF}}c 
    >{\columncolor[HTML]{EFEFEF}}c |llll|ll|}
    \cline{3-8}
    \multicolumn{2}{c|}{\cellcolor[HTML]{FFFFFF}\textit{}} & \multicolumn{4}{c|}{\cellcolor[HTML]{EFEFEF}\textit{Matriz de confusión}} & \multicolumn{2}{c|}{\cellcolor[HTML]{EFEFEF}\textit{\begin{tabular}[c]{@{}c@{}}K-Fold Cross Validation \\ *score() / **evaluate()\end{tabular}}} \\ \cline{3-8} 
    \multicolumn{2}{c|}{\cellcolor[HTML]{FFFFFF}\textit{}} & \multicolumn{1}{c|}{\cellcolor[HTML]{EFEFEF}\textit{Accuracy}} & \multicolumn{1}{c|}{\cellcolor[HTML]{EFEFEF}\textit{Precision}} & \multicolumn{1}{c|}{\cellcolor[HTML]{EFEFEF}\textit{Recall}} & \multicolumn{1}{c|}{\cellcolor[HTML]{EFEFEF}\textit{F1 Score}} & \multicolumn{1}{c|}{\cellcolor[HTML]{EFEFEF}\textit{Accuracy}} & \multicolumn{1}{c|}{\cellcolor[HTML]{EFEFEF}\textit{Standard Deviation / (*,**)Loss}} \\ \hline
    \multicolumn{1}{|c|}{\cellcolor[HTML]{EFEFEF}} & \textit{Sin aplicar} & \multicolumn{1}{c|}{99,29} & \multicolumn{1}{c|}{94,40} & \multicolumn{1}{c|}{74,27} & \multicolumn{1}{c|}{83,16} & \multicolumn{1}{c|}{99,30} & \multicolumn{1}{c|}{0,02} \\ \cline{2-8} 
    \multicolumn{1}{|c|}{\cellcolor[HTML]{EFEFEF}} & \textit{RFECV} & \multicolumn{1}{c|}{99,44} & \multicolumn{1}{c|}{94,28} & \multicolumn{1}{c|}{81,05} & \multicolumn{1}{c|}{87,17} & \multicolumn{1}{c|}{99,47} & \multicolumn{1}{c|}{0,02} \\ \cline{2-8} 
    \multicolumn{1}{|c|}{\cellcolor[HTML]{EFEFEF}} & \textit{kbest (n=5)} & \multicolumn{1}{c|}{97,79} & \multicolumn{1}{c|}{65,97} & \multicolumn{1}{c|}{12,55} & \multicolumn{1}{c|}{21,08} & \multicolumn{1}{c|}{97,80} & \multicolumn{1}{c|}{0,02} \\ \cline{2-8} 
    \multicolumn{1}{|c|}{\cellcolor[HTML]{EFEFEF}} & \textit{kbest (n=8)} & \multicolumn{1}{c|}{97,74} & \multicolumn{1}{c|}{53,95} & \multicolumn{1}{c|}{27,37} & \multicolumn{1}{c|}{36,32} & \multicolumn{1}{c|}{97,79} & \multicolumn{1}{c|}{0,01} \\ \cline{2-8} 
    \multicolumn{1}{|c|}{\cellcolor[HTML]{EFEFEF}} & \textit{PCA (n=2)} & \multicolumn{1}{c|}{97,55} & \multicolumn{1}{c|}{12,52} & \multicolumn{1}{c|}{00,67} & \multicolumn{1}{c|}{01,27} & \multicolumn{1}{c|}{97,35} & \multicolumn{1}{c|}{0,01} \\ \cline{2-8} 
    \multicolumn{1}{|c|}{\multirow{-6}{*}{\cellcolor[HTML]{EFEFEF}\textit{\begin{tabular}[c]{@{}c@{}}Random Forest \\ {[}25 estimadores, \\ criterio de entropia{]}\end{tabular}}}} & \textit{PCA (n=4)} & \multicolumn{1}{c|}{98,06} & \multicolumn{1}{c|}{81,66} & \multicolumn{1}{c|}{22,60} & \multicolumn{1}{c|}{35,41} & \multicolumn{1}{c|}{98,04} & \multicolumn{1}{c|}{0,00} \\ \hline
    \multicolumn{1}{|c|}{\cellcolor[HTML]{EFEFEF}} & \textit{Sin aplicar} & \multicolumn{1}{c|}{97,23} & \multicolumn{1}{c|}{89,32} & \multicolumn{1}{c|}{06,95} & \multicolumn{1}{c|}{12,83} & \multicolumn{1}{c|}{97,79} & \multicolumn{1}{c|}{0,01} \\ \cline{2-8} 
    \multicolumn{1}{|c|}{\cellcolor[HTML]{EFEFEF}} & \textit{kbest (n=5)} & \multicolumn{1}{c|}{97,11} & \multicolumn{1}{c|}{65,25} & \multicolumn{1}{c|}{12,29} & \multicolumn{1}{c|}{20,73} & \multicolumn{1}{c|}{97,80} & \multicolumn{1}{c|}{0,01} \\ \cline{2-8} 
    \multicolumn{1}{|c|}{\cellcolor[HTML]{EFEFEF}} & \textit{kbest (n=8)} & \multicolumn{1}{c|}{97,05} & \multicolumn{1}{c|}{71,19} & \multicolumn{1}{c|}{09,30} & \multicolumn{1}{c|}{16,46} & \multicolumn{1}{c|}{97,79} & \multicolumn{1}{c|}{0,01} \\ \cline{2-8} 
    \multicolumn{1}{|c|}{\cellcolor[HTML]{EFEFEF}} & \textit{PCA (n=2)} & \multicolumn{1}{c|}{97,61} & \multicolumn{1}{c|}{-} & \multicolumn{1}{c|}{0} & \multicolumn{1}{c|}{-} & \multicolumn{1}{c|}{97,76} & \multicolumn{1}{c|}{0,00} \\ \cline{2-8} 
    \multicolumn{1}{|c|}{\multirow{-5}{*}{\cellcolor[HTML]{EFEFEF}\textit{\begin{tabular}[c]{@{}c@{}}SVM \\ {[}C=1, kernel=rbf{]}\end{tabular}}}} & \textit{PCA (n=4)} & \multicolumn{1}{c|}{97,61} & \multicolumn{1}{c|}{-} & \multicolumn{1}{c|}{0} & \multicolumn{1}{c|}{-} & \multicolumn{1}{c|}{97,76} & \multicolumn{1}{c|}{0,00} \\ \hline
    \multicolumn{1}{|c|}{\cellcolor[HTML]{EFEFEF}} & \multicolumn{1}{c|}{\cellcolor[HTML]{EFEFEF}\textit{sklearn}} & \multicolumn{1}{c|}{97,84} & \multicolumn{1}{c|}{69,03} & \multicolumn{1}{c|}{14,98} & \multicolumn{1}{c|}{24,66} & \multicolumn{1}{c|}{97,83} & \multicolumn{1}{c|}{*-} \\ \cline{2-8} 
    \multicolumn{1}{|c|}{\multirow{-2}{*}{\cellcolor[HTML]{EFEFEF}\textit{\begin{tabular}[c]{@{}c@{}}ANN\\ {[}(5, 5), relu, sgd{]}\end{tabular}}}} & \multicolumn{1}{c|}{\cellcolor[HTML]{EFEFEF}\textit{keras}} & \multicolumn{1}{c|}{97,84} & \multicolumn{1}{c|}{72,17} & \multicolumn{1}{c|}{13,49} & \multicolumn{1}{c|}{22,74} & \multicolumn{1}{c|}{97,85} & \multicolumn{1}{c|}{**0,0690} \\ \hline
    \end{tabular}
    \caption{Síntesis de resultados obtenidos del desarrollo de los modelos de \acrshort{ml} y \acrshort{dl}}
    \label{tab:resumen}
\end{sidewaystable}


\section{Líneas de trabajo futuro}
\label{trabajoFuturo}

Tras exponer el alcance logrado de los objetivos del presente \gls{tfm}, se introducen las posibles vías de trabajo futuro para la continuación de este proyecto de investigación.

\begin{itemize}
    \item Ampliar el estudio del comportamiento del algoritmo \gls{den2ne} y de los patrones de error a partir de otros conjuntos de datos reales. De esta forma, se podría realizar diversas evaluaciones, a partir de otras fuentes de implementaciones reales de \gls{sg}s en ubicaciones diferentes.
    \item Desarrollar técnicas adicionales de \gls{ml} y \gls{dl} y evaluar el rendimiento de los nuevos modelos. En este caso, se propone plantear el empleo de Máquinas de Refuerzo de Gradiente (del inglés \gls{gbm}), como \textit{XGBoost}, ya que permiten manejar grandes cantidades de datos. Sería de interés comprobar si se mejoran los resultados de precisión obtenidos en este proyecto.
    \item Profundizar el empleo de los modelos para estudiar y predecir otros parámetros relativos al comportamiento de la distribución energética que simula \gls{den2ne}. Se podría plantear un estudio de la predicción del balance de cargas, lo que seguiría mejorando las funcionalidades del algoritmo y proporcionaría una visión más amplia de su funcionamiento.
    \item Simular en \gls{den2ne} otros instantes temporales del conjunto de datos. En este proyecto se han seleccionado 12 instantes temporales para comprender un día por cada mes del año, suponiendo datos suficientes para entrenar los modelos desarrollados de forma rigurosa. Como línea futura, se podrían seleccionar y probar otras muestras temporales.
    \item Extender el desarrollo realizado en este \gls{tfm} al contexto de redes de dispositivos \gls{iot}. Por su caracter flexible, sería posible implementar la detección y predicción de errores. En este entorno, se podría basar la fuente de error en las posibles sobrecargas computacionales de los dispositivos.
\end{itemize}

En definitiva, las líneas de trabajo futuro presentadas proporcionan diversas oportunidades de ampliación del proyecto de investigación realizado, lo que indica el gran alcance que permitiría su continuación.
