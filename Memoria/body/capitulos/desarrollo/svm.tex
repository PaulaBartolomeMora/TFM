%%%%%%%%%%%%%%%%%%%%%%%%%%%%%%%%%%%%%%%%%%%%%%%%%%%%%%%%%%%%%%%%%%
\section{Support Vector Machines (\acrshort{svm})}
\label{sec:svm}






% C a fin de cuentas el hiperparámetro encargado de controlar el balance entre bias y varianza del modelo. En la práctica, su valor óptimo se identifica mediante validación cruzada. Esta es la razón por la que el parámetro  C controla el balance entre bias y varianza lo que permite un ajuste adecuado del modelo. 

%Cuando el valor de  Ces pequeño, el margen es más ancho, y más observaciones violan el margen, convirtiéndose en vectores soporte. El hiperplano está, por lo tanto, sustentado por más observaciones, lo que aumenta el bias pero reduce la varianza. 

%Cuando mayor es el valor de  C , menor el margen, menos observaciones son vectores soporte y el clasificador resultante tiene menor bias pero mayor varianza.

%Otra propiedad importante que deriva de que el hiperplano dependa únicamente de una pequeña proporción de observaciones (vectores soporte), es su robustez frente a observaciones muy alejadas del hiperplano. Esto hace al método de clasificación vector soporte distinto a otros métodos tales como Linear Discrimiant Analysis (LDA), donde la regla de clasificación depende de la media de todas las observaciones.