%%%%%%%%%%%%%%%%%%%%%%%%%%%%%%%%%%%%%%%%%%%%%%%%%%%%%%%%%%%%%%%%%%
\section{Análisis de los conjuntos de datos}
\label{sec:sustdata}

En ámbitos como el Procesado de Lenguaje Natural (del inglés \gls{nlp}) o el reconocimiento facial, la disponibilidad de numerosas fuentes de datos ha sido imprescindible para acelerar el desarrollo de técnicas de minería de datos y de aprendizaje automático (\gls{ml}).

\vspace{3mm}

A diferencia de estos campos, el contexto energético experimenta un menor grado de disponiblidad de datasets. La protección de la privacidad de los usuarios en el proceso de medición y análisis de su comportamiento energético supone que exista un menor número de datasets públicos enfocados a implementaciones reales. No obstante, la búsqueda de la optimización de la distribución energética y de la reducción del consumo de los usuarios, aparte de otras motivaciones ambientales, ha instado en los últimos años a múltiples ingenieros y científicos a lo largo del mundo a crear datasets públicos enfocados a la investigación (ver Tabla \ref{tab:datasets}). 

\vspace{3mm}

En este contexto han cobrado gran importancia las tecnologías de Monitorización de Cargas no Intrusiva (del inglés \gls{nilm}). Su fin principal reside en la desagregación energética para poder estimar de una forma precisa el consumo individual de cada uno de los dispositivos y electrodomésticos que hay en una vivienda. Para ello, se instalan medidores en cada circuito y después, se analizan, tanto a nivel interno como a nivel externo de la vivienda, los cambios de los parámetros eléctricos en cada instante temporal.~\cite{nilm} \cite{greend}

\vspace{3mm}

Teniendo esto en cuenta, se puede expresar que las técnicas \gls{nilm} se caracterizan por su bajo coste y por su facilidad y flexibilidad de despliegue. No obstante, el proceso de medición y recolección de datos puede llegar a requerir mucho tiempo y esfuerzo. Por ello, en el contexto de la investigación es importante que los datos tengan una disponibilidad pública para progresar en el desarrollo de técnicas de minería de datos y de aprendizaje automático en el ámbito energético. 

\vspace{3mm}

En la Tabla \ref{tab:datasets} se exponen los múltiples datasets residenciales que han sido estudiados en esta fase de diseño, junto con información relativa a la implementación real sobre la que se ha basado cada uno de ellos, como es la ubicación de la misma, el número de viviendas total, el período temporal del despliegue, la frecuencia de las muestras tomadas y los parámetros eléctricos medidos. Tomando en consideración las motivaciones y objetivos del presente \gls{tfm}, es importante tener en cuenta ciertos requisitos para seleccionar el conjunto de datos más apropiado:

\vspace{1mm}

\begin{itemize}
    \item Cantidad: En primera instancia, es imprescindible revisar la cantidad de datos que aporta cada uno. En otros términos, para llevar a cabo un análisis estadístico del comportamiento energético de una ubicación se requiere partir de un conjunto de datos que agrupe las mediciones de un gran número de edificios residenciales. Adicionalmente, este análisis debe comprender extensos períodos temporales de medición para obtener una perspectiva real del impacto de las estaciones y las condiciones climáticas en el comportamiento de los usuarios.
    \item Calidad: Se debe evaluar la calidad de los datos recogidos, la cual responde en parte con la resolución de las medidas que se toman. Como se puede observar en la Tabla \ref{tab:datasets}, algunos datasets como \textit{REDD} y \textit{BLUED} se centran en monitorizar un pequeño número de casas a una frecuencia de muestreo alta. En el contexto de las tecnologías \gls{nilm}, esto aporta una vista más representativa del comportamiento energético a nivel interno en una vivienda y permite una desagregación energética más precisa. Por lo tanto, cuanto mayor sea la frecuencia de adquisición de medidas, mayor será la resolucion de los datos.
    \item Ubicación: En términos de la ubicación de la implementación real sobre la que se han adquirido las mediciones, es preciso tener en cuenta a la hora de analizar los datos las diferencias de voltaje que se manejan en cada país. Por ejemplo, en el caso de los datasets \textit{BLUED}, \textit{REDD} y \textit{Smart*}, provenientes de Estados Unidos, trabajarán a tensiones menores de 120V, mientras que otros como \textit{ECO} o \textit{GREEND}, que se han basado en países europeos, lo harán con tensiones de hasta 230V. \cite{greend}
    \item Parámetrización: Por lo general, un dataset dedicado a las tecnologías \gls{nilm} estará constituido por una colección de muestras de voltaje (V), corriente (I) y potencia activa (P) y reactiva (Q). Cada una de estas muestras vendrá asociada con su marca de tiempo y con la vivienda o medidores a los que corresponde. ------No obstante, también podrá incluir información adicional relativa a parámetros ambientales, lo que será de gran relevancia en el caso de las \gls{sg}s.
\end{itemize}



\vspace{3mm}



\vspace{3mm}




En contraste, el hecho de buscar un conjunto de datos 











\begin{sidewaystable}
    \centering 
    \begin{tabularx}{\textheight}{|X|X|X|X|X|X|}
        \hline
        \rowcolor[HTML]{EFEFEF} 
        Nombre & Localización & Nº de residencias & Período de medición (días) & Resolución de medición & Parámetros \\ \hline
        \textit{BLUED} \cite{blued} & Pittsburg (Estados Unidos) & 1 & 8 & 12KHz &  I, V, eventos de switch \\ \hline
        \textit{ECO} \cite{eco} & Thun (Suiza) & 6 & 244 & 1Hz &  \\ \hline
        \textit{GREEND} \cite{greend} & Italia y Austria & 9 & 310 & 1Hz & P \\ \hline
        \textit{iAWE} \cite{iawe} & Nueva Delhi (La India) & 1 & 73 & 1Hz & V, I, P, S \\ \hline
        \textit{REDD} \cite{redd} & Boston (Estados Unidos) & 6 & 19 & 15KHz & V, P \\ \hline
        \textit{Smart*} \cite{smart*} & Massachussets (Estados Unidos) & 3 & 90 & 1min & P, S \\ \hline
        \textit{SustDataED} \cite{sustdata} & Madeira (Portugal) & 50 & 504 & 1min & I, V, P, Q \\ \hline
        \textit{UK-DALE} \cite{ukdale} & Reino Unido & 5 & 499 & 16KHz & P, estado de switch \\ \hline
    \end{tabularx}
    \caption{Comparación entre datasets públicos en el ámbito \acrshort{nilm} \cite{greend} \cite{intrusive} \cite{tabladatasets}}
    \label{tab:datasets}
\end{sidewaystable}















%cinco conjuntos de datos públicos creados específicamente para la investigación NILM, a saber, el
% Conjunto de datos de desagregación de energía de referencia (REDD) [10], el
% Conjunto de datos totalmente etiquetados a nivel de edificio para electricidad
% Desagregación (AZUL) [11], el Almanaque de Minuto
Conjunto de datos de % de potencia (AMPds) [12] Pecan Street Research
% Institute (PSRI) [13] y, por último, el conjunto de datos “UK-Dale” [14]. Todo
% de estos conjuntos de datos proporcionan datos sobre el consumo de toda la casa y
% de información veraz sobre el terreno de aparatos individuales ya sea por
% de etiquetado de los cambios de energía en el consumo de toda la casa.
% de señal (el caso de BLUED) o proporcionando el valor agregado
% consumo de electrodomésticos o circuitos individuales de la casa
% (REDD, AMPds, PSRI y “UK-Dale”). Además, dos de
% estos conjuntos de datos complementan los datos de consumo de energía con
% otras mediciones, concretamente el consumo de gas y agua en
% de AMPds y generación fotovoltaica en la calle Pecan
% conjunto de datos. Se espera que estos desempeñen papeles importantes en la
% creación de nuevas estrategias de desagregación que incluyan datos de
% sensores diferentes, es decir, fusión de sensores.




% combina ambos NILM específicos
% de información con consumo a largo plazo y eco-feedback
% de información, proporcionando así una oportunidad perfecta para probar NILM
% de enfoques que impactan los estudios a largo plazo de hogares y
% eco-retroalimentación. Creemos que SustData también puede desempeñar un papel
% papel importante en los campos de la energía, el medio ambiente y
% sostenibilidad. Él
% también puede desempeñar un papel en campos emergentes como las redes inteligentes, donde
Los modelos % precisos de demanda de energía son cruciales para la planificación
% redes de distribución eléctrica y producción óptima
% capacidad








































\subsection{SustDataED}
\cite{sustdata}


% The work presented in this paper emerges from a large 
% multi-disciplinary research project (SINAIS1
%  – Sustainable 
% Interactions with Social Networks, context Awareness and 
% Innovative Services) looking at how social networks and 
% context-awareness can be used to promote sustainable
% behaviors. The research team involving civil and electrical 
% engineers, computer scientists, psychologists and designers was 
% responsible for designing, implementing and deploying several 
% low-cost non-intrusive electric energy monitoring systems [5]. 
% These systems were sensing electrical circuits in many 
% households with the goal of providing eco-feedback to the 
% families. The research involved several deployments of 
% different eco-feedback systems, including both qualitative and 
% quantitative evaluation of the user interactions with the 
% deployed systems. The overall goal was to raise the 
% understanding and the awareness towards motivating people to 
% consume more sustainably.
% During this period of almost five years we have collected 
% and stored large quantities of valuable data, including energy 
% consumption, user interactions with the eco-feedback systems 
% and electricity production from renewable sources, which we 
% are now making publicly available to other researchers in the 
% Information and Communication Technologies for 
% Sustainability (ICT4S) community

% Motivated by the increasing importance of public datasets 
% for ICT4S research we decided to make our dataset public. Our 
% data is unique because it combines both NILM specific 
% information with long-term consumption and eco-feedback 
% information thus providing a perfect opportunity to test NILM 
% approaches that impact long-term studies of households and 
% eco-feedback.






\subsection{Global Solar Atlas}
































%hacer intro en relacion con el apartado de big data

%%para intro -->

%cite stab

%There are two main types of renewable energy data: geospatial and
% temporal data. Geospatial data is concerned with the locations, while
% temporal data is concerned with data time characteristics. For renewable energy, Geospatial data may include the location of transmission
% infrastructure, cities, factories, hospitals, schools, roads, etc. (Shekhar
% et al., 2012); this data is based mainly on Geographical Information Systems (GIS) tools. Temporal data may include the consumption patterns
% with respect to time (annually, monthly, weekly, daily, and hourly)
% besides the amount of energy (e.g., sunshine) during different times
% of day or year.
% A third type user classification data can be the social classification,
% users can be classified into categories not only according to geographic
% areas but also to their social stratums that can be an indicator for daily
% consumption curves (Zhou et al., 2016a). The weather data (e.g., angle
% of the sun rays, wind speed and direction, temperature, pressure, cloud
% cover, humidity, etc.) play a basic role in decision-making support in
% power stations (Zhou et al., 2016b). Hence, the integration between
% supply and demand data, spatial data, and temporal data can support
% strategic decisions such as location selection for renewable energy
% stations to improve output, productivity and efficiency. For a comprehensive review on big data and its techniques for energy systems, the
% reader is referred to the works by Jiang et al. (2016), Molina-Solana
% et al. (2017), Ma et al. (2017).