<<<<<<< HEAD
%%%%%%%%%%%%%%%%%%%%%%%%%%%%%%%%%%%%%%%%%%%%%%%%%%%%%%%%%%%%%%%%%%
\section{Análisis de fuentes de datos}
\label{sec:sustdata}

%aqui explicar los datasets que se han visto
%enfocarlo mas a todos lo datasets o a sustdata?
%aqui habria que explicar tambien la creacion de los datos de produccion


\cite{sustdata}





%hacer intro en relacion con el apartado de big data

%%para intro -->

%cite stab

%There are two main types of renewable energy data: geospatial and
% temporal data. Geospatial data is concerned with the locations, while
% temporal data is concerned with data time characteristics. For renewable energy, Geospatial data may include the location of transmission
% infrastructure, cities, factories, hospitals, schools, roads, etc. (Shekhar
% et al., 2012); this data is based mainly on Geographical Information Systems (GIS) tools. Temporal data may include the consumption patterns
% with respect to time (annually, monthly, weekly, daily, and hourly)
% besides the amount of energy (e.g., sunshine) during different times
% of day or year.
% A third type user classification data can be the social classification,
% users can be classified into categories not only according to geographic
% areas but also to their social stratums that can be an indicator for daily
% consumption curves (Zhou et al., 2016a). The weather data (e.g., angle
% of the sun rays, wind speed and direction, temperature, pressure, cloud
% cover, humidity, etc.) play a basic role in decision-making support in
% power stations (Zhou et al., 2016b). Hence, the integration between
% supply and demand data, spatial data, and temporal data can support
% strategic decisions such as location selection for renewable energy
% stations to improve output, productivity and efficiency. For a comprehensive review on big data and its techniques for energy systems, the
% reader is referred to the works by Jiang et al. (2016), Molina-Solana
=======
%%%%%%%%%%%%%%%%%%%%%%%%%%%%%%%%%%%%%%%%%%%%%%%%%%%%%%%%%%%%%%%%%%
\section{Análisis de fuentes de datos}
\label{sec:sustdata}

%aqui explicar los datasets que se han visto
%enfocarlo mas a todos lo datasets o a sustdata?
%aqui habria que explicar tambien la creacion de los datos de produccion


\cite{sustdata}





%hacer intro en relacion con el apartado de big data

%%para intro -->

%cite stab

%There are two main types of renewable energy data: geospatial and
% temporal data. Geospatial data is concerned with the locations, while
% temporal data is concerned with data time characteristics. For renewable energy, Geospatial data may include the location of transmission
% infrastructure, cities, factories, hospitals, schools, roads, etc. (Shekhar
% et al., 2012); this data is based mainly on Geographical Information Systems (GIS) tools. Temporal data may include the consumption patterns
% with respect to time (annually, monthly, weekly, daily, and hourly)
% besides the amount of energy (e.g., sunshine) during different times
% of day or year.
% A third type user classification data can be the social classification,
% users can be classified into categories not only according to geographic
% areas but also to their social stratums that can be an indicator for daily
% consumption curves (Zhou et al., 2016a). The weather data (e.g., angle
% of the sun rays, wind speed and direction, temperature, pressure, cloud
% cover, humidity, etc.) play a basic role in decision-making support in
% power stations (Zhou et al., 2016b). Hence, the integration between
% supply and demand data, spatial data, and temporal data can support
% strategic decisions such as location selection for renewable energy
% stations to improve output, productivity and efficiency. For a comprehensive review on big data and its techniques for energy systems, the
% reader is referred to the works by Jiang et al. (2016), Molina-Solana
>>>>>>> 0d7b4e19ce4dbb9c918f8bc78eb546b2d1448fdb
% et al. (2017), Ma et al. (2017).