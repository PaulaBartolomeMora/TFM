%%%%%%%%%%%%%%%%%%%%%%%%%%%%%%%%%%%%%%%%%%%%%%%%%%%%%%%%%%%%%%%%%%
\section{Preprocesado de los datos}
\label{sec:preprocesado}

%meter graficas del paper de sustdata
%meter graficas de sustdata (ej. consumo de un hogar y tal, correlacion prod-clima)
%definir que es lo que se va a utilizar

%ESTO ES LO QUE ESTA PUESTO ARRIBA --> (PARA ACORDARME DE REFERENCIARLO AQUI Y DETALLARLO)
% El procesamiento que será requerido para estos datos se detallará en la Sección \ref{sec:preprocesado}. No obstante, es importante destacar las características de los datos de producción eléctrica dados por \textit{SustDataED}. Como se ha expuesto anteriormente en la Sección \ref{prodsustdata}, el dataset proporciona esta información en términos globales y después, desagrega los valores energéticos según su fuente de procedencia. Como se expondrá en la Sección \ref{sec:preprocesado}, será imprescindible determinar si estos datos son precisos en un entorno de \gls{sg}s como se requiere para cumplir con los objetivos de este \gls{tfm}


% 7 y 7.1 del AI-based FDI Countermeasure for IoE Smart Grids


%The SustData dataset contains, at the time of this writing
% (March 10, 2014), over 50 million individual records of electric 
% energy related data, spanning a total of 1144 distinct days since 
% the 29th of July 2010.

%hacerme una idea de la estructura -> ver pag 37 AI-based-FDI-Countermeasure-for-IoE-Smart-Grids











%ESTO PARA CUANDO SE HABLE DE BRITE

%En la Sección \ref{sec:brite_eje}, referente a la ejecución de la herramienta \gls{brite} se había expuesto la posibilidad de generar un total de 1200 topologías para probar sobre el algoritmo \gls{den2ne}. Ahora, teniendo en cuenta también el número de nodos disponibles (25) y la cantidad de instantes temporales comprendidos en los datos (24*365=8760), existe la posibilidad de realizar hasta 262.800.000 simulaciones únicas.